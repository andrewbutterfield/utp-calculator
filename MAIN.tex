\documentclass[12pt,fleqn]{article}
\usepackage[a4paper, margin=1.1in]{geometry}
\usepackage{amssymb}
\usepackage{amsmath}
\usepackage{stmaryrd}
\usepackage{graphicx}
\usepackage{xcolor}
\usepackage[utf8]{inputenc}
\usepackage{listings}
\usepackage{lstlhs}
\usepackage{doc/views}
\usepackage{doc/RGAlg}
\usepackage{doc/UTPCalc}
\usepackage{mathpartir}
\usepackage{epstopdf}
\usepackage{doc/doclevels}
\usepackage{hyperref}
\allowdisplaybreaks[2]
% ----------------------------------------------------------------

\parindent=0pt % I hate first line indentation
\parskip=3pt   % I like a visual white gap between paragraphs

\setcounter{tocdepth}{4}
\setcounter{secnumdepth}{4}


\newif\ifDraft
% \Drafttrue  % comment out to turn off note/draft-mode
\ifDraft
  \def\NOTE#1{\textbf{Note: }\textsl{#1}}
  \def\DRAFT#1{\textbf{Draft: }\textit{#1}}
\else
  \def\NOTE#1{}
  \def\DRAFT#1{}
\fi

\bibliographystyle{alpha}

\title{UTPCalc --- A calculator for UTP Predicates}%
\author{
   Andrew Butterfield
      \thanks{\noindent
         This work was supported,
         in part,
         by Science Foundation Ireland grants 10/CE/I1855 and 13/RC/2094
        to Lero
         ---
         the Irish Software Engineering Research Centre (www.lero.ie)}%
}%
\date{\today}%

\parindent=0pt % I hate first line indentation
\parskip=3pt   % I like a visual white gap between paragraphs

\mathchardef\spot="320F
\mathcode`\@=\spot
%\mathcode`\|=\mid

\setcounter{tocdepth}{4}

\begin{document}

%
% Basic Predicates
%
\def\mTrue{true} \def\tTrue{\mbox{Always True}}
\def\mFalse{false} \def\tFalse{\mbox{Never True}}
\def\mPV{P} \def\tPV{\mbox{Predicate Variable}}
\def\mEq{e_1 = e_2} \def\tEq{\mbox{Equality}}
\def\mAtm{e} \def\tAtm{\mbox{Atomic Predicate}}
\def\mComp{P(p_1,\ldots,p_n)} \def\tComp{\mbox{Composite}}
\def\mSubs{p[e_1,\ldots,e_n/v_1,\ldots,v_n]} \def\tSubs{\mbox{Substitution}}

\def\BASIC{\RLEQNS{ 
   P &\in& Var & \mbox{Pred-Vars}
\\ p \in Pred &::=&
       \mTrue & \tTrue
\\ &|&\mFalse & \tFalse
\\ &|& \mPV & \tPV
\\ &|& \mEq & \tEq
\\ &|& \mAtm & \tAtm
\\ &|& \mComp & \tComp
\\ &|& \mSubs & \tSubs
}}

%
% Standard Predicates
%
\def\mNot{\lnot p} \def\tNot{\mbox{Negation}}
\def\mAnd{p_1 \land p_2 \land \ldots \land p_n} \def\tAnd{\mbox{Conjunction}}
\def\mOr{p_1 \lor p_2 \lor \ldots \lor p_n} \def\tOr{\mbox{Disjunction}}
\def\mImp{p_1 \implies p_2} \def\tImp{\mbox{Implication}}
\def\mEqv{p_1 \equiv p_2} \def\tEqv{\mbox{Equivalence}}

\def\STANDARD{\RLEQNS{
   p \in P &::=& \ldots &
\\ &|& \mNot & \tNot
\\ &|& \mAnd & \tAnd
\\ &|& \mOr & \tOr
\\ &|& \mImp & \tImp
\\ &|& \mEqv & \tEqv
}}


%
% UTP Standard Predicates
%
\def\mTop{\top} \def\tTop{\mbox{(Lattice) Top, a.k.a. miracle}}
\def\mBot{\bot}  \def\tBot{\mbox{(Lattice) Bottom, a.k.a. abort/Chaos}}
\def\mNDC{p_1 \ndc p_2 \ndc \ldots \ndc p_n} \def\tNDC{\mbox{Non-Det. Choice}}
\def\mRby{p_1 \refinedby p_2} \def\tRby{\mbox{Refinement}}
\def\mCond{p_1 \cond{p_2} p_3} \def\tCond{\mbox{Conditional}}
\def\mSkip{\Skip} \def\tSkip{\mbox{Skip}}
\def\mSeq{p_1 \seq p_2} \def\tSeq{\mbox{Sequencing}}
\def\mIter{p_1 * p_2} \def\tIter{\mbox{Iteration}}

\def\UTPSTANDARD{\RLEQNS{
   p \in P &::=& \ldots &
\\ &|& \mTop & \tTop
\\ &|& \mBot & \tBot
\\ &|& \mNDC & \tNDC
\\ &|& \mRby & \tRby
\\ &|& \mCond & \tCond
\\ &|& \mSkip & \tSkip
\\ &|& \mSeq & \tSeq
\\ &|& \mIter & \tIter
}}


\maketitle
\tableofcontents

\newpage
\HDRa{Introduction}\label{ha:intro}

to be done


% \newpage
% \input{src/UTCPSemantics.lhs}
\newpage
\input{src/UTCPLaws.lhs}
\newpage
\input{src/UTCPCReduce.lhs}
\newpage
\input{src/UTCPCalc.lhs}
\newpage
\HDRa{Math support for Views}\label{ha:mViews}

We go for a new formulation
\RLEQNS{
   atm(a) &=& \true * (\Skip \lor (in\arr a out) \land [in|out|g])
\\ C &=& \bigvee_{i=0}^\infty (actions(C)) \land [in|out|g] \land I_C
}
where $I_C$ is some more $C$-specific invariants.

\HDRb{Basic Actions}

We define the basic action as:
\RLEQNS{
   E \arr a N
   &\defs&
   ls(E) \land s' \in \sem a s \land ls' = (ls\setminus E)\cup N
}
and calculate its self-sequential composition as:
\RLEQNS{
  && E_1 \arr a N_1 \seq E_2 \arr b N_2
\EQ{Defns of $\seq$ and $\arr{-}$}
\\&& \exists s_m,ls_m @
\\&& \quad ls(E_1) \land s_m \in \sem a s \land ls_m = (ls\setminus E_1)\cup N_1
     \land {}
\\&& \quad ls_m(E_2) \land s' \in \sem b s_m \land ls' = (ls_m\setminus E_2)\cup N_2
\EQ{1pt-rule, $ls_m = (ls\setminus E_1)\cup N_1$}
\\&& \exists s_m @
\\&& \quad ls(E_1) \land s_m \in \sem a s \land s' \in \sem b s_m \land  {}
\\&& \quad ((ls\setminus E_1)\cup N_1)(E_2)
           \land ls' = (((ls\setminus E_1)\cup N_1)\setminus E_2)\cup N_2
\EQ{push scope in}
\\&& ls(E_1) \land
     (\exists s_m @ s_m \in \sem a s \land s' \in \sem b s_m) \land  {}
\\&& ((ls\setminus E_1)\cup N_1)(E_2) \land
     ls' = (((ls\setminus E_1)\cup N_1)\setminus E_2)\cup N_2
\EQ{Defn $\seq$, re-arrange}
\\&& ls(E_1) \land
     ((ls\setminus E_1)\cup N_1)(E_2) \land {}
\\&& s' \in \sem{a\seq b} s) \land
     ls' = (((ls\setminus E_1)\cup N_1)\setminus E_2)\cup N_2
\EQ{simplify $ls'$}
\\&& ls(E_1) \land
     ((ls\setminus E_1)\cup N_1)(E_2) \land {}
\\&& s' \in \sem{a\seq b} s) \land
     ls' = (ls\setminus (E_1\setminus N_1)\cup E_2))
           \cup (N_1\setminus E_2 \cup N_2
\EQ{GUESS $ls$}
\\&& ls(E_1 \cup (E_2 \setminus N_1))\land 
     s' \in \sem{a\seq b} s) \land {}
\\&& ls' = (ls\setminus (E_1\setminus N_1\cup E_2))
           \cup (N_1\setminus E_2 \cup N_2
\EQ{fold defn $\arr{-}$}
\\&& (~E_1 \cup (E_2 \setminus N_1)
     \arr{a\seq b}
     (N_1\setminus E_2) \cup N_2~)
}

Check $(E \arr a N)^2$
\RLEQNS{
  && (E \arr a N)^2
\EQ{prev. calc.}
}

\newpage
\HDRb{Working with the Invariant}

We have introduced the following notation:
\[
  [ L_1 | L_2 | \dots | L_n ]
\]
Its first intended meaning is to assert that
all the $L_i$ are mutually disjoint:
\RLEQNS{
   \forall i,j \in 1\dots n @ i \neq j \implies L_i \cap L_j = \emptyset
}
It also states that if any one of its members overlaps with $ls$,
then none of the others do (and similarly for $ls'$):
\RLEQNS{
  &&   \forall i,j \in 1\dots n @
        i \neq j \land L_i \cap ls \neq \emptyset
        \implies L_j \cap ls = \emptyset
\\&&   \forall i,j \in 1\dots n @
        i \neq j \land L_i \cap ls' \neq \emptyset
        \implies L_j \cap ls' = \emptyset
}

By \emph{design}, our labels and generator scheme
establishes the following (weakest) invariant:
\[ [in|out|g]\]
and for any generator expression $G$ we can split it into
four disjoint parts (also by design) to get
\[  [\ell_G|G_{:}|G_1|G_2] . \]
So we can always strengthen out invariant by splitting some $G$
in this way:
\RLEQNS{
  && [in|out|g]
\EQ{split $g$}
\\&& [in|out|\ell_g|\g:|\g1|\g2]
\EQ{split $\g:$}
\\&& [in|out|\ell_g| \ell_{g:}|\g{::}|\g{:1}|\g{:2}|\g1|\g2]
}
However, sometimes we don't wait to do this.
The hope is that by requiring $[in|out|g]$ at each level,
that we get the right level of exclusivity vs. sharing of $ls$
by generated labels.

\newpage
\HDRb{Semantic Definitions}

For ``wheels-within-wheels'' to work,
we need to ensure we spin up as long as any constituent
atomic action does not have its $out$ label in $ls$.
\RLEQNS{
  C &=& ls(\B{OUT}) * \bigvee_{i=1}^n in_i \arr{a_i} out_i
\\ && \textbf{where }OUT = \setof{out_1,\dots,out_n}
}
Now this looks a little non-compositional,
but perhaps we can get this outcome by only looking one level down\dots

We assume the generator invariant $[in|out|g]$
and all its strengthening hold throughout.



\RLEQNS{
   E \arr a N &\defs& ls(E) \land s'\in\sem a \land ls'=(ls\setminus E)\cup N
\\ atm(a) &\defs& ls(\B{out}) * in \arr a out
\\ C \cseq D
   &\defs&
   ls(\B{\ell_g,out}) *
     ( C[\ell_g,\g{:1}/out,g] \lor
       D[\ell_g,\g{:2}/in,g] )
\\ C + D
   &\defs&
   ls(\B{out}) *
     ( C[g_{1:},\ell_{g1}/g,in] \lor
       D[g_{2:},\ell_{g2}/g,in] )
\\ C \parallel D
   &\defs&
   ls(\B{\ell_{g1},\ell_{g2},\ell_{g1:},\ell_{g2:},out}) * {}
\\&& \quad (~
      (in \arr{ii} \ell_{g1},\ell{g2}) \lor
      (\ell_{g1:},\ell{g2:} \arr{ii} out) \lor {}
\\&& \quad ~~
       C[g_{1::},\ell_{g1},\ell_{g1:}/g,in,out] \lor
       D[g_{2::},\ell_{g2},\ell_{g2:}/g,in,out] ~)
\\ C^*
   &\defs&
   ls(\B{out}) *
     (~ (in \arr{ii} out) \lor
       (in \arr{ii} \ell_g) \lor
       C[\g:,\ell_g,in/g,in,out] ~)
}


\newpage
\HDRb{Semantic Calculations}

We need to only investigate the combinators
applied to atoms, but also some nesting to see if everything
does keep going.

The basic principle relies on the following properties of
standard UTP iteration:
\RLEQNS{
  && c*P
\EQ{loop unroll}
\\&& P \seq c*P \cond c \Skip
\EQ{defn. of $\cond{}$}
\\&& \lnot c \land \Skip
     \lor
     c \land (P \seq c*P)
\EQ{push $c$ into $\seq$}
\\&& \lnot c \land \Skip
     \lor
     c \land P \seq c*P
\EQ{repeat above steps several times with $\lor$-$\seq$-distr.}
\\&& \lnot c \land \Skip
     \lor
     c \land P \seq \lnot c \land \Skip
     \lor
     (c \land P)^2 \seq \lnot c \land \Skip
     \lor \dots \lor
     (c \land P)^n \seq c*P
\EQ{pushed to the limit}
\\&& \bigvee_{i=0}^\infty (c \land P)^i \seq \lnot c \land \Skip
}

We introduce further shorthand:
\RLEQNS{
   D &\defs& \lnot c \land \Skip
\\ S &\defs& c \land P
\\ c*P &=& \bigvee_{i=0}^\infty S^i \seq D
}
So the nature of $S^i\seq D$ is the critical thing to compute.


SOME OLD STUFF THAT MAY NEED REWRITING:

Now, for our basic primitives some easy calculations show:
\RLEQNS{
   D(T) &\implies& ls(T) \land ls'(T)
\\ E \arr a N &\implies& ls(E) \land ls'(N)
}
If we keep in mind that,
\RLEQNS{
  P \land c' \seq Q &=& P \seq c \land Q
}
then we can very easily read-off whether or not sequential compositions
of these two violate the invariant, e.g, given $[in|out|\ell_g]$
\RLEQNS{
   D(in) \seq D(\ell_g) &=& \false
\\ (in \arr a \ell_g) \seq (\ell_g \arr b out)
    &=& (in \arr{a \seq b} out)
\\ (in \arr a \ell_g) \seq (in \arr a \ell_g) &=& \false
\\ (in \arr a \ell_g) \seq D(out) &=& \false
\\ (in \arr a \ell_g) \seq D(\ell_g) &=& (in \arr a \ell_g)
}

\newpage
\HDRa{Math support for Views}\label{ha:mViews}

We go for a new formulation
\RLEQNS{
   atm(a) &=& \true * (\Skip \lor (in\arr a out) \land [in|out|g])
\\ C &=& \bigvee_{i=0}^\infty (actions(C)) \land [in|out|g] \land I_C
}
where $I_C$ is some more $C$-specific invariants.

\HDRb{Basic Actions}

We define the basic action as:
\RLEQNS{
   E \arr a N
   &\defs&
   ls(E) \land s' \in \sem a s \land ls' = (ls\setminus E)\cup N
}
and calculate its self-sequential composition as:
\RLEQNS{
  && E_1 \arr a N_1 \seq E_2 \arr b N_2
\EQ{Defns of $\seq$ and $\arr{-}$}
\\&& \exists s_m,ls_m @
\\&& \quad ls(E_1) \land s_m \in \sem a s \land ls_m = (ls\setminus E_1)\cup N_1
     \land {}
\\&& \quad ls_m(E_2) \land s' \in \sem b s_m \land ls' = (ls_m\setminus E_2)\cup N_2
\EQ{1pt-rule, $ls_m = (ls\setminus E_1)\cup N_1$}
\\&& \exists s_m @
\\&& \quad ls(E_1) \land s_m \in \sem a s \land s' \in \sem b s_m \land  {}
\\&& \quad ((ls\setminus E_1)\cup N_1)(E_2)
           \land ls' = (((ls\setminus E_1)\cup N_1)\setminus E_2)\cup N_2
\EQ{push scope in}
\\&& ls(E_1) \land
     (\exists s_m @ s_m \in \sem a s \land s' \in \sem b s_m) \land  {}
\\&& ((ls\setminus E_1)\cup N_1)(E_2) \land
     ls' = (((ls\setminus E_1)\cup N_1)\setminus E_2)\cup N_2
\EQ{Defn $\seq$, re-arrange}
\\&& ls(E_1) \land
     ((ls\setminus E_1)\cup N_1)(E_2) \land {}
\\&& s' \in \sem{a\seq b} s) \land
     ls' = (((ls\setminus E_1)\cup N_1)\setminus E_2)\cup N_2
\EQ{simplify $ls'$}
\\&& ls(E_1) \land
     ((ls\setminus E_1)\cup N_1)(E_2) \land {}
\\&& s' \in \sem{a\seq b} s) \land
     ls' = (ls\setminus (E_1\setminus N_1)\cup E_2))
           \cup (N_1\setminus E_2 \cup N_2
\EQ{GUESS $ls$}
\\&& ls(E_1 \cup (E_2 \setminus N_1))\land 
     s' \in \sem{a\seq b} s) \land {}
\\&& ls' = (ls\setminus (E_1\setminus N_1\cup E_2))
           \cup (N_1\setminus E_2 \cup N_2
\EQ{fold defn $\arr{-}$}
\\&& (~E_1 \cup (E_2 \setminus N_1)
     \arr{a\seq b}
     (N_1\setminus E_2) \cup N_2~)
}

Check $(E \arr a N)^2$
\RLEQNS{
  && (E \arr a N)^2
\EQ{prev. calc.}
}

\newpage
\HDRb{Working with the Invariant}

We have introduced the following notation:
\[
  [ L_1 | L_2 | \dots | L_n ]
\]
Its first intended meaning is to assert that
all the $L_i$ are mutually disjoint:
\RLEQNS{
   \forall i,j \in 1\dots n @ i \neq j \implies L_i \cap L_j = \emptyset
}
It also states that if any one of its members overlaps with $ls$,
then none of the others do (and similarly for $ls'$):
\RLEQNS{
  &&   \forall i,j \in 1\dots n @
        i \neq j \land L_i \cap ls \neq \emptyset
        \implies L_j \cap ls = \emptyset
\\&&   \forall i,j \in 1\dots n @
        i \neq j \land L_i \cap ls' \neq \emptyset
        \implies L_j \cap ls' = \emptyset
}

By \emph{design}, our labels and generator scheme
establishes the following (weakest) invariant:
\[ [in|out|g]\]
and for any generator expression $G$ we can split it into
four disjoint parts (also by design) to get
\[  [\ell_G|G_{:}|G_1|G_2] . \]
So we can always strengthen out invariant by splitting some $G$
in this way:
\RLEQNS{
  && [in|out|g]
\EQ{split $g$}
\\&& [in|out|\ell_g|\g:|\g1|\g2]
\EQ{split $\g:$}
\\&& [in|out|\ell_g| \ell_{g:}|\g{::}|\g{:1}|\g{:2}|\g1|\g2]
}
However, sometimes we don't wait to do this.
The hope is that by requiring $[in|out|g]$ at each level,
that we get the right level of exclusivity vs. sharing of $ls$
by generated labels.

\newpage
\HDRb{Semantic Definitions}

For ``wheels-within-wheels'' to work,
we need to ensure we spin up as long as any constituent
atomic action does not have its $out$ label in $ls$.
\RLEQNS{
  C &=& ls(\B{OUT}) * \bigvee_{i=1}^n in_i \arr{a_i} out_i
\\ && \textbf{where }OUT = \setof{out_1,\dots,out_n}
}
Now this looks a little non-compositional,
but perhaps we can get this outcome by only looking one level down\dots

We assume the generator invariant $[in|out|g]$
and all its strengthening hold throughout.



\RLEQNS{
   E \arr a N &\defs& ls(E) \land s'\in\sem a \land ls'=(ls\setminus E)\cup N
\\ atm(a) &\defs& ls(\B{out}) * in \arr a out
\\ C \cseq D
   &\defs&
   ls(\B{\ell_g,out}) *
     ( C[\ell_g,\g{:1}/out,g] \lor
       D[\ell_g,\g{:2}/in,g] )
\\ C + D
   &\defs&
   ls(\B{out}) *
     ( C[g_{1:},\ell_{g1}/g,in] \lor
       D[g_{2:},\ell_{g2}/g,in] )
\\ C \parallel D
   &\defs&
   ls(\B{\ell_{g1},\ell_{g2},\ell_{g1:},\ell_{g2:},out}) * {}
\\&& \quad (~
      (in \arr{ii} \ell_{g1},\ell{g2}) \lor
      (\ell_{g1:},\ell{g2:} \arr{ii} out) \lor {}
\\&& \quad ~~
       C[g_{1::},\ell_{g1},\ell_{g1:}/g,in,out] \lor
       D[g_{2::},\ell_{g2},\ell_{g2:}/g,in,out] ~)
\\ C^*
   &\defs&
   ls(\B{out}) *
     (~ (in \arr{ii} out) \lor
       (in \arr{ii} \ell_g) \lor
       C[\g:,\ell_g,in/g,in,out] ~)
}


\newpage
\HDRb{Semantic Calculations}

We need to only investigate the combinators
applied to atoms, but also some nesting to see if everything
does keep going.

The basic principle relies on the following properties of
standard UTP iteration:
\RLEQNS{
  && c*P
\EQ{loop unroll}
\\&& P \seq c*P \cond c \Skip
\EQ{defn. of $\cond{}$}
\\&& \lnot c \land \Skip
     \lor
     c \land (P \seq c*P)
\EQ{push $c$ into $\seq$}
\\&& \lnot c \land \Skip
     \lor
     c \land P \seq c*P
\EQ{repeat above steps several times with $\lor$-$\seq$-distr.}
\\&& \lnot c \land \Skip
     \lor
     c \land P \seq \lnot c \land \Skip
     \lor
     (c \land P)^2 \seq \lnot c \land \Skip
     \lor \dots \lor
     (c \land P)^n \seq c*P
\EQ{pushed to the limit}
\\&& \bigvee_{i=0}^\infty (c \land P)^i \seq \lnot c \land \Skip
}

We introduce further shorthand:
\RLEQNS{
   D &\defs& \lnot c \land \Skip
\\ S &\defs& c \land P
\\ c*P &=& \bigvee_{i=0}^\infty S^i \seq D
}
So the nature of $S^i\seq D$ is the critical thing to compute.


SOME OLD STUFF THAT MAY NEED REWRITING:

Now, for our basic primitives some easy calculations show:
\RLEQNS{
   D(T) &\implies& ls(T) \land ls'(T)
\\ E \arr a N &\implies& ls(E) \land ls'(N)
}
If we keep in mind that,
\RLEQNS{
  P \land c' \seq Q &=& P \seq c \land Q
}
then we can very easily read-off whether or not sequential compositions
of these two violate the invariant, e.g, given $[in|out|\ell_g]$
\RLEQNS{
   D(in) \seq D(\ell_g) &=& \false
\\ (in \arr a \ell_g) \seq (\ell_g \arr b out)
    &=& (in \arr{a \seq b} out)
\\ (in \arr a \ell_g) \seq (in \arr a \ell_g) &=& \false
\\ (in \arr a \ell_g) \seq D(out) &=& \false
\\ (in \arr a \ell_g) \seq D(\ell_g) &=& (in \arr a \ell_g)
}

\newpage
\section{Root}\label{ha:Root}

We do a quick run-down of the Commands\cite{conf/popl/Dinsdale-YoungBGPY13}.

\subsection{Syntax}

\def\Atm#1{\langle#1\rangle}
\def\rr#1{r{\scriptstyle{#1}}}

\begin{eqnarray*}
   a &\in& \Atom
\\ C &::=&
 \Atm a \mid \cskip \mid C \cseq C \mid C+C \mid C \parallel C \mid C^*
\\ g &:& Gen
\\ \ell &:& Lbl
\\ G &::=&  g \mid G_{:} \mid G_1 \mid G_2
\\ L &::=& \ell_G
\end{eqnarray*}


\subsection{Domains}


\subsubsection{Rooted Paths}

This Root file is a re-work of the Views semantics
replacing label generators by ``rooted'' label-paths.

We start by defining three basic ways to transform a rooted path:
``step'' ($:$);
``split-one'' ($1$);
and ``split-two'' ($2$).
\RLEQNS{
   S &\defs& \setof{:,1,2}
}

We now define a rooted path as an expression of the form
of the variable $r$ followed by zero or more $S$ transforms.
\RLEQNS{
   \sigma,\varsigma &\defs& S^*
\\ R &::=& r | R S
\\   & = & r\sigma
}
These have to be expressions as we shall want to substitute for $r$
in them.

\subsubsection{Path Ordering}
\RLEQNS{
   R &\le& R\sigma
\\ R1\sigma &<& R\!:\!\varsigma
\\ R2\sigma &<& R\!:\!\varsigma
}
\RLEQNS{
   r &\le& r\sigma
}
\RLEQNS{
   R1\sigma &<& R\!:\!\varsigma
}
\RLEQNS{
   R2\sigma &<& R\!:\!\varsigma
}
\RLEQNS{
   R &\le& R\sigma
}

\subsection{Shorthands}

We support a shorthand (that views a set as its own collection
of corresponding $n$-ary characteristic functions)
that denotes $x \in S$ by $S(x)$ and $ X \subseteq S$ by $S(X)$,
and omits $\{$ and $\}$ from around enumerations when context makes
it clear a set is expected

\begin{eqnarray*}
   ls(\ell) &\defs& \ell \in ls
\\ ls(L) &\defs& L \subseteq ls
\\ ls(\B\ell) &\defs& \ell \notin ls
\\ ls(\B L) &\defs& L \cap ls = \emptyset
\end{eqnarray*}


\paragraph{Set Swapping}

We update a set by removing some elements
and replacing them with new ones:
\RLEQNS{
   A \ominus (B,C) &\defs& (A \setminus B) \cup C
}

\subsection{Alphabet}

\begin{eqnarray*}
   s, s' &:& \mathcal S
\\ ls, ls' &:& \mathcal P (R)
\\ r &:& R
\end{eqnarray*}

We define our dictionary alphabet entries,
and also declare that the predicate variables $a$, $b$ and $c$
will refer to atomic state-changes,
and so only have alphabet $\setof{s,s'}$.

\subsection{``Standard'' UTP Constructs}

\begin{eqnarray*}
   P \cond c Q
   &\defs&
   c \land P \lor \lnot c \land Q
\\ P ; Q
   &\defs&
   \exists s_m,ls_m \bullet P[s_m,ls_m/s',ls'] \land Q[s_m,ls_m/s,ls]
\\ c * P
   &=&
   P ; c * P \cond c \Skip
\end{eqnarray*}

We need to update some definitions from standard UTP as follows:

\paragraph{Updating UTP Skip ($\Skip$)}\label{hd:updating-UTP-II}

We know have a concrete definition for $\Skip$,
as well as a known alphabet.
\RLEQNS{
   \Skip &=& ls'=ls \land s'=s
\\ \alpha \Skip &=& \setof{ls,ls',s,s'}
}

\subsection{WwW Basic Shorthands}

Atomic actions have a basic behaviour that is described by
\[
ls(E) \land s' \in \sem a s \land ls' = (ls\setminus E) \cup N
\]
where $E$ is the set of necessary enabling labels,
$a$ is a relation over shared state,
and $N$ is the set of new labels deposited upon completion.
We shall abstract the above as
\[
A(E|a|N)
\]
There is a slightly more general form that removes labels
that may differ from those doing the enabling:
\[
ls(E) \land s' \in \sem a s \land ls' = (ls\setminus R) \cup A
\]
where $R$ and $A$ are sets of labels respectively removed, then added
on the action is complete.
We abstract this as
\[
  X(E|a|R|A)
\]
and note that
\[
  A(E|a|N) = X(E|a|E|N).
\]


\subsubsection{Generic Atomic Behaviour}

\begin{eqnarray*}
   X(E|ss'|R|A)
   &\defs&
   ls(E) \land [ss'] \land ls'=(ls\setminus R)\cup A
\end{eqnarray*}

We do want the following simplification:
\begin{eqnarray*}
  X(E|a|E|N) &=& A(E|a|N)
\end{eqnarray*}

\begin{eqnarray*}
   A(E|ss'|N)
   &\defs&
   X(E|ss'|E|N)
\end{eqnarray*}


\subsubsection{Coding Label-Set Invariants}

We have a key invariant as part of the healthiness
condition associated with every semantic predicate,
namely that the labels $r$ and $\rr:$ never occur in  $ls$ at
the same time:
\[
 ( r \in ls \implies \rr: \notin ls )
 \land
 ( \rr: \in ls \implies r \notin ls )
\]
This is the Label Exclusivity invariant.

Given the way we shall use substitution of $r$ by
other rooted paths, for sub-components,
we shall see that we will get a number of instances of this.
We adopt a shorthand notation,
so that the above invariant is simply
\[
  [r|\rr:]
\]
So we define the following general shorthand:
\RLEQNS{
   ~[L_1|L_2|\dots|L_n]
   &\defs&
   \forall_{i,j \in 1\dots n}
    @
    i \neq j \implies
     ( L_i \cap ls \neq \emptyset \implies L_j \cap ls = \emptyset )
\\ \multicolumn{3}{c}{\elabel{short-lbl-exclusive}}
}
What needs to be kept in mind regarding this shorthand notation
is that $ls$ is mentioned under the hood,
and it is really all about what can be present in the global label-set
at any instant in time.


Now, we need to define invariant satisfaction.
Our invariant applies to $A$ and $X$ atomic actions:
\RLEQNS{
   A(E|a|N) \textbf{ sat } I
   &\defs&
   E \textbf{ lsat } I \land N \textbf{ lsat } I
\\ X(E|a|R|A)  \textbf{ sat } I
   &\defs&
   E \textbf{ lsat } I \land A \textbf{ lsat } I
}

Now we have to define \textbf{lsat}:
\RLEQNS{
   \textbf{lsat}_S [L_1|\dots|L_n]
   &\defs&
   \#(filter ~\textbf{lsat'}_S \seqof{L_1,\ldots,L_n}) < 2
\\ \textbf{lsat'}_S \setof{\ell_1,\dots,\ell_n}
  &\defs&
  \exists i @ \textbf{lsat''}_S \ell_i
\\ \textbf{lsat''}_S \ell &\defs& \ell \in S
}

\subsubsection{Invariant Trimming}

We can use the invariant to trim removal sets,
given an enabling label or label-set.
\RLEQNS{
   I \textbf{ invTrims } X(E|a|E,L|A)
   &\defs&
   \lnot(\setof{E,L} \textbf{ lsat } I)
}

\subsubsection{Wheels within Wheels}\label{hc:WwW}

The wheels-within-wheels healthiness condition
insists that $r$ and $\rr:$ are never simultaneously in
the label-set $ls$,
and that our semantic predicates are closed under mumbling.
\RLEQNS{
   \W(C)
   &\defs&
   [r|\rr:] \land \left(~\bigvee_{i\in 0\dots} C^i~\right)
\\ ii &\defs& s'=s
}

Unrolling $\W(C)$, using $I$ for $[r|\rr:]$,
and noting that
$
\bigvee_{i \in \Nat} C^i
= \Skip \lor (C\seq\bigvee_{i \in \Nat} C^i)
$.
\RLEQNS{
   \W(C) &=& I \land \bigvee C^i
\\ &=& I \land (\Skip \lor C\seq\bigvee C^i)
\\ &=& I \land (\Skip \lor C\seq((\Skip \lor C\seq\bigvee C^i)))
\\ &=& I \land (\Skip \lor C \lor C^2\seq\bigvee C^i)
\\ &=& I \land (\Skip \lor C \lor C^2\seq((\Skip \lor C\seq\bigvee C^i)))
\\ &=& I \land (\Skip \lor C \lor C^2 \lor C^3\seq\bigvee C^i)
\\ &\vdots&
\\ &=& I \land (\Skip \lor C \lor \dots C^{k-1} \lor C^k\seq\bigvee C^i)
\\ &=& I \land (~(\bigvee_{i < k} C^i) \lor (\bigvee_{i \geq k} C^i)~)
}
We assume $\seq$ binds tighter than $\lor$.

\subsection{WwW Semantic Definitions}

The definitions, using the new shorthands:
\RLEQNS{
   \W(C) &\defs& [r|\rr:]
                 \land
                 \left(\bigvee_{i\in 0\dots} C^i\right)
\\ ii &\defs& s'=s
\\
\\ \Atm a &\defs&\W(A(r|a|\rr:))
\\
\\ \cskip
   &\defs&
   \Atm{ii}
}
The following are all under review (they lack sufficient invariants).
\RLEQNS{
   C \cseq D
   &\defs&
   \W(~    A(r|ii|\rr1)
      \lor C[\rr1/r]
      \lor A(\rr{1:}|ii|\rr2)
      \lor D[\rr2/r]
      \lor A(\rr{2:}|ii|\rr:) ~)
\\
\\ C + D
   &\defs&
   \W(\quad {}\phlor A(r|ii|\rr1) \lor A(r|ii|\rr2)
\\ && \qquad {} \lor
   C[\rr1/r] \lor D[\rr2/r]
\\ && \qquad {} \lor A(\rr{1:}|ii|\rr:) \lor A(\rr{2:}|ii|\rr:) ~)
\\
\\ C \parallel D
   &\defs&
   \W(\quad\phlor A(r|ii|\rr1,\rr2)
\\ && \qquad {}\lor
   C[\rr1/r]
   \lor D[\rr2/r]
\\ && \qquad {}\lor
   A(\rr{1:},\rr{2:}|ii|\rr:)~)
\\
\\ C^*
   &\defs&
   \W(\quad  \phlor A(r|ii|\rr1) \lor A(\rr1|ii|\rr:)
\\ && \qquad {}\lor C[\rr1/r]    \lor A(\rr{1:}|ii|\rr1) ~)
}

\subsubsection{Coding Atomic Semantics}

\RLEQNS{
   \Atm a &\defs&\W(A(r|a|\rr:))
}

\newpage
\subsubsection{Coding Skip}

\RLEQNS{
   ii &\defs& s'=s
\\ \cskip
   &\defs&
   \Atm{ii}
}


\subsubsection{Coding Sequential Composition}

\RLEQNS{
   C \cseq D
   &\defs& [r|\rr1|\rr{1:}|\rr2|\rr{2:}|\rr:] \land {}
\\ && \W(\quad {}\phlor A(r|ii|\rr1)
\\ && \qquad {} \lor C[\rr1/r]
\\ && \qquad {} \lor A(\rr{1:}|ii|\rr2)
\\ && \qquad {} \lor D[\rr2/r]
\\ && \qquad {} \lor A(\rr{2:}|ii|\rr:)~)
}

\subsubsection{Coding (Non-Det.) Choice}

\RLEQNS{
   C + D
   &\defs& [r|\rr1|\rr{1:}|\rr2|\rr{2:}|\rr:] \land {}
\\&& \W(\quad {}\phlor A(r|ii|\rr1) \lor A(r|ii|\rr2)
\\ && \qquad {} \lor
   C[\rr1/r] \lor D[\rr2/r]
\\ && \qquad {} \lor A(\rr{1:}|ii|\rr:) \lor A(\rr{2:}|ii|\rr:) ~)
}

\subsubsection{Coding Parallel Composition}

\RLEQNS{
   C \parallel D
   &\defs& ~
   [r|\rr1,\rr2,\rr{1:},\rr{2:}|\rr:] \land
   [\rr1|\rr{1:}] \land
   [\rr2|\rr{2:}] \land {}
\\&& \W(\quad\phlor A(r|ii|\rr1,\rr2)
\\ && \qquad {}\lor
   C[\rr1/r]
   \lor D[\rr2/r]
\\ && \qquad {}\lor
   A(\rr{1:},\rr{2:}|ii|\rr:)~)
}

\subsubsection{Coding Iteration}

\RLEQNS{
   C^*
   &\defs& [r|\rr2|\rr1|\rr{1:}|\rr:] \land {}
\\&& \W(\quad  \phlor A(r|ii|\rr2)
\\ && \qquad {}\lor A(\rr2|ii|\rr1)
               \lor C[\rr1/r]
               \lor A(\rr{1:}|ii|\rr2)
\\ && \qquad {}\lor A(\rr2|ii|\rr:) ~)
}


\subsection{Reductions for WWW}\label{hb:WWW-reduce}

\subsubsection{Recognisers for WWW}\label{hc:v-recog}

\RLEQNS{
   ls'=ls\ominus(S_1,S_2)
   &\rightsquigarrow&
   \seqof{S_1,S_2}
   & \ecite{sswap-$;$-prop.}
}

\paragraph{$A$ then $A$}

\RLEQNS{
  && A(E_1|a|N_1) \seq A(E_2|b|N_2)
\EQ{proof in Views.tex }
\\&& X(E_1 \cup (E_2\setminus N_1)
       |a\seq b
       |E_1 \cup E_2
       |N_1 \setminus E_2 \cup  N_2)
       \land (E_2\setminus N_1) \cap E_1 = \emptyset
}

\paragraph{$X$ then $A$}

\RLEQNS{
  && X(E_1|a|R_1|A_1) \seq A(E_2|b|N_2)
\EQ{proof in Views.tex }
\\&& X(E_1 \cup (E_2\setminus A_1)
       |a\seq b
       |R_1 \cup E_2
       |A_1 \setminus E_2 \cup  N_2)
       \land (E_2\setminus A_1) \cap R_1 = \emptyset
}

\paragraph{$A$ then $X$}

\RLEQNS{
  && A(E_1|a|N_1) \seq X(E_2|b|R_2|A_2)
\EQ{proof in Views.tex}
\\&& X(E_1 \cup (E_2 \setminus N_1)
      | a\seq b
      | E_1 \cup R_2
      | N_1 \setminus R_2 \cup A_2)
     \land
     (E_2 \setminus N_1) \cap E_1 = \emptyset
}

\paragraph{$X$ then $X$}

\RLEQNS{
  && X(E_1|a|R_1|A_1) \seq X(E_2|b|R_2|A_2)
\EQ{proof in Views.tex }
\\&& X(E_1 \cup (E_2\setminus A_1)
       |a\seq b
       |R_1 \cup R_2
       |A_1 \setminus R_2 \cup  A_2)
       \land (E_2\setminus A_1) \cap R_1 = \emptyset
}

\paragraph{$I$ collapses $X$ to $A$}
\RLEQNS{
   \lnot(\setof{E,L} \textbf{ lsat } I)
   &\implies&  X(E|a|E,L|N) = A(E|a|N)
}
Given $I$ and $X(E|a|R|A)$, we proceed as follows:
\begin{enumerate}
  \item Let $D = R \setminus E$
  \item Compute
    $D' =
       \setof{  d | d \in D,
                    \lnot (E\cup\setof d \textbf{ lsat } I)}
    $
  \item
    If $D = D'$ then return $A(E|a|A)$
  \item
    Else, return $X(E|a|E\cup(D \setminus D')|A)$.
\end{enumerate}

\paragraph{General Stuff}~

If $a$ and $b$ have alphabet $\setof{s,s'}$,
then
$(a \seq b)[G,L,M/g,in,out) = a\seq b$.

\subsection{Loop Unrolling for Views}\label{hb:WWW-unroll}

Iteration  satisfies the loop-unrolling law:
\[
  c * P  \quad=\quad (P ; c * P ) \cond c \Skip
\]
But we also support several styles and degrees of unrolling:
\begin{eqnarray*}
   c*P
   &=_0& (P\seq c*P) \cond c \Skip
\\ &=_1& \lnot c \land \Skip
         \lor
         c \land P ; c * P
\\ &=_2& \lnot c \land \Skip
         \lor
         c \land P ; \lnot c \land \Skip
         \lor
         c \land P ; c \land P ; c * P
\\ &=_3& \lnot c \land \Skip
         \lor
         c \land P ; \lnot c \land \Skip
         \lor
         c \land P ; c \land P ; \lnot c \land \Skip
         \lor
         c \land P ; c \land P ; c \land P ; c *P
\\ && \vdots
\\ &=_n& \left(
           \bigvee_{i=0}^{n-1}  (c \land P)^i \seq \lnot c \land \Skip
         \right)
         \lor
         (c \land P)^n \seq c *P
\end{eqnarray*}


\subsection{Calculation Results}

Given the idiom $\true * (\Skip \lor Q)$
we look for pre-computed $Q$-cores,
noting that our final semantics will have the form
\[
  I \land ( \Skip \lor Q \lor Q^2 \lor \dots Q^k)
\]
where $I$ is the label-set invariant
and $Q^i = \false$ for all $i > k$.

When we change from:
\RLEQNS{
   \W(C) &\defs& \true * (\Skip \lor C)
\\ \Atm a &\defs&\W(A(in|a|out)) \land [in|out]
}
\noindent
to:
\RLEQNS{
   \W(C) &\defs& \true * C
\\ \Atm a &\defs&\W(\Skip \lor A(in|a|out)) \land [in|out]
}
\noindent
the calculation outcomes
for \texttt{athenb}, \texttt{aorb}, \texttt{awithb} and \texttt{itera} are unchanged,
and consequently so are all the other calculations.

% \newpage
% \HDRa{Rely-Guarantee Algebra Redux}\label{ha:RGAlg-redux}


\HDRb{From the FM2016 Tutorial}
\RLEQNS{
       & \top
\\ \nil &      & \alf
\\     & \chaos
\\ & \bot
}

\begin{center}
\begin{tabular}{|c|c|c|c|c|c|}
  \hline
    & assoc & comm & idem & unit & zero
  \\\hline
  $\sqcap$ & \checkmark & \checkmark & \checkmark & $\top$ & $\bot$
  \\\hline
  $\sqcup$ & \checkmark & \checkmark & \checkmark & $\bot$ & $\top$
  \\\hline
\end{tabular}
\end{center}

\RLEQNS{
   r \subseteq \Sigma \times \Sigma
\\ π(r) &=& \Pi(\sigma,\sigma'), (\sigma,\sigma') \in r
\\ ϵ(r) &=& \mathcal{E}(\sigma,\sigma'), (\sigma,\sigma') \in r
\\ \stutter &=& \pi(\id)
\\ \pi &=& \pi(\univ)
\\ \epsilon &=& \epsilon(\univ)
\\ p &\subseteq& \Sigma
\\ τ(p) &=& \mbox{if $p$ then terminate else $\top$}
\\ \pre~ t &=& t \sqcap \lnot t \bot
\\  &=& t \sqcap (\lnot t) \seq \bot
\\ \setof p &=& \pre~\tau(p)
\\ &=& \tau(p) \sqcap \tau(\overline{p})\bot
\\ !  && \mbox{not sure what this is}
\\ \assume~ a &=& a \sqcap (!a) \bot
\\ \pi(\emp) &=& \top
\\ \epsilon(\emp) &=& \top
\\ \tau(\emp) &=& \top
\\ \tau(\Sigma) &=& \nil
\\ \tau(p_1) \sqcap \tau(p_2) &=& \tau(p_1 \cup p_2)
\\ \tau(p_1) \sqcup \tau(p_2) &=& \tau(p_1 \cap p_2)
\\                            &=& \tau(p_1)\tau(p_2)
\\                            &=& \tau(p_1)\parallel\tau(p_2)
\\ \lnot\tau(p) &=& \tau(\overline p)
\\ \assume~\pi \sqcap \epsilon(r)
   &=&
   \pi \sqcap \epsilon(r) \sqcap \epsilon(\overline{r})\bot
}

\newpage
\HDRb{From the FM2016 (joint-Best) Paper}

\HDRc{Introduction}

Assume $a$, $b$ atomic, $c$, $d$ arbitrary processes.
\RLEQNS{
   (a;c)\parallel(b;d) &=& (a\parallel b);(c\parallel d)
\\ (a;c)\ileave(b;d) &=& a;(c\ileave b;d) \sqcap b;(a;c\ileave d)
}

\HDRc{Concurrent Refinement Algebra}~

Concurrent Refinement Algebra (CRA):
\[
(\mathcal C,\sqcap,\sqcup,;,\parallel,\bot,\top,\nil,\Skip)
\]
Complete, distributive lattice:
$
(\mathcal C,\sqcap,\sqcup,\bot,\top)
$.
\RLEQNS{
   c \sqsubseteq d &\defs& (c \sqcap d) = c
\\ \bot \quad \sqsubseteq &c& \sqsubseteq \quad \top
}
Monoid:
$
  (\mathcal C, ;, \nil)
$.
\RLEQNS{
   \top ; c &=& \top
\\ \bot ; c &=& \bot
\\ c ; \top &\neq& \top
\\ c ;\bot &\neq& \bot
\\ (\bigsqcap C) ; d &=& \bigsqcap_{c \in C}(c;d)
\\ c^0 &\defs& \nil
\\ c^{i+1} &\defs& c ; c^i
\\ c^\star &\defs& \nu x . \nil \sqcap c ; x
\\ c^\omega &\defs& \mu x . \nil \sqcap c ;x
\\ c^\infty &\defs& c^\omega ; \top
\\ c^\omega &=& \nil \sqcap c ; c^\omega
\\ c^\star &=& \nil \sqcap c ; c^\star
\\ c^\infty &=& c ; c^\infty ~=~ c^i ; c^\infty ~=~ c^\infty ; d
}
True in their relational model, but not generally in CCS or CSP:
\RLEQNS{
   D \neq \setof{} &\implies& c;(\bigsqcap D) = \bigsqcap_{d \in D}(c;d)
}
It says that ; is \emph{conjunctive}.
Needed for the following:
\RLEQNS{
   c^\omega &=& c^\star \sqcap c^\infty
\\ c^\star &=& \bigsqcap_{i \in \Nat} c^i
\\ c^\omega ; d &=& c^\star;d \sqcap c^\infty
\\ c;c^\omega;d &=& c;c^\star;d \sqcap c^\infty
}

\HDRc{The Boolean Sub-algebra of Tests}~

Test commands: $t \in \mathcal B \subseteq C$, extended algebra:
\[
(\mathcal C,\mathcal B,\sqcap,\sqcup,;,\parallel,\bot,\top,\nil,\Skip,\lnot)
\]
Test Boolean algebra --- sub-lattice of CRA:
$
(\mathcal B,\sqcap,\sqcup,\lnot,\top,\nil)
$

$\mathcal B$ closed under $\sqcap, \sqcup, ;, \parallel$.

Assume $t \in \mathcal B$, arbitrary test.
\RLEQNS{
   t;t' &=& t \sqcup t'
\\ t\parallel t' &=& t \sqcup t'
\\ (t;c) \parallel (t;d) &=& t;(c\parallel d)
\\ (t;c) \sqcup (t';d) &=& (t \sqcup t') ; (c \sqcup d)
\\ \Assert~t &\defs& t \sqcap \lnot t ; \bot
\\ \lnot \top &=& \nil
}

\HDRc{Abstract Atomic Steps}~

Atomic Steps commands: $a,b \in \mathcal A \subseteq C$.

Atomic Action Boolean algebra --- sub-lattice of CRA:
$
(\mathcal A,\sqcap,\sqcup,!,\top,\alf)
$
\RLEQNS{
   \alf \sqcup \nil &=& \top
}

$\mathcal A$ closed under $\sqcap, \sqcup, \parallel$, but not $;$.

\RLEQNS{
   a \parallel \wait &=& a
\\ a;c \parallel b;d &=& (a \parallel b);(c\parallel d)
\\ a;c \sqcup b;d &=& (a \sqcup b);(c \sqcup d)
\\ a;c \parallel \nil &=& \top
\\ a;c \sqcup \nil &=& \top
\\ a \sqcup !a &=& \top
\\ a \sqcap !a &=& \alf
\\ !\top &=& \alf
\\ \assume~a &\defs& a \sqcap (!a);\bot
}

Given any $c$ there are $t$, $t'$, $I$, $a_i$ and $c_i$ such that:
\RLEQNS{
   c &=& t \sqcap t';\bot \sqcap \bigsqcap_{i \in I}(a_i ; c_i)
\\ \Skip &\defs& \wait^\omega
\\ \wait^\omega \parallel c &=& c
\\ a^\star\parallel \nil &=& \nil
\\ a^\omega\parallel \nil &=& \nil
\\ a^\infty\parallel \nil &=& \top
\\ a^i;c \parallel b^i;d &=& (a\parallel b)^i ; (c \parallel d)
}
If ; is conjunctive:
\RLEQNS{
   a^\star \parallel b^\star &=& (a \parallel b)^\star
\\ a^\infty \parallel b^\infty &\defs?& (a \parallel b)^\infty
\\ a^\star;c \parallel b^\star;d
   &=&
   (a \parallel b)^\star
   ;
   ( (c \parallel d)
     \sqcap
     (c \parallel b;b^\star;d)
     \sqcap
     (a;a^\star;c \parallel d) )
\\ a^\star;c \parallel b^\infty
   &=&
   (a\parallel b)^\star; (c\parallel b^\infty)
\\ a^\omega;c \parallel b^\omega;d
   &=&
   (a \parallel b)^\omega
   ;
   ( (c \parallel d)
     \sqcap
     (c \parallel b;b^\omega;d)
     \sqcap
     (a;a^\omega;c \parallel d) )
\\ \action a &\defs& \wait^\omega ; a; \wait^\omega
\\ \action a \parallel \action b
   &=&
   \action{a\parallel b}
   \sqcap \action a ; \action b
   \sqcap \action b ; \action a
\\ a \ileave b &=& a;b \sqcap b;a
}


\HDRc{Relational Atomic Steps}~

\RLEQNS{
   \sigma &\in& \Sigma
\\ r &\in& \Set(\Sigma\times\Sigma)
\\ \pi &:& \Set(\Sigma\times\Sigma) \fun \mathcal A
\\ \epsilon &:& \Set(\Sigma\times\Sigma) \fun \mathcal A
\\ \pi(\emptyset) ~~= &\top& =~~ \epsilon(\emptyset)
\\ \pi(r_1) \sqcup \epsilon(r_2) &=& \top
}
For $s \in \setof{\pi,\epsilon}$:
\RLEQNS{
   r_1=r_2 &\Leftrightarrow& s(r_1)=s(r_2)
\\ s(r_1 \cup r_2) &=& s(r_1) \sqcap s(r_2)
\\ s(r_1 \cap r_2) &=& s(r_1) \sqcup s(r_2)
\\ r_1 \subseteq r_2 &\implies& s(r_2) \sqsubseteq s(r_1)
}

\RLEQNS{
   p &\in& \Set\Sigma
\\ \tau &:& \Set\Sigma \fun \mathcal B
\\ \tau(\emptyset) &=& \top
\\ \tau(\Sigma) &=& \nil
\\ \Pre p &\defs& \Assert~\tau(p)
\\     &  =  & \tau(p) \sqcap \tau(\lnot p);\bot
\\ \Pre\emptyset&=& \bot
\\ \Pre\Sigma &=& \nil
}

\HDRc{Relies and Guarantees}~

\RLEQNS{
   g &\in& \Set(\Sigma\times\Sigma)
\\ (\piRestrict~g) &\defs& \pi(g) \sqcap \wait
\\ \guar~g &\defs& (\piRestrict~g)^\omega
\\ g_1 \subseteq q_2 &\implies& (\piRestrict~g_2) \sqsubseteq (\piRestrict~g_1)
}

\RLEQNS{
   c \Cap \bot &=& \bot
\\ (c \Cap c') \Cap c'' &=& c \Cap (c' \Cap c'')
\\ c \Cap d &=& d \Cap c
\\ c \Cap c &=& c
\\ c \Cap (\bigsqcap D) &=& (\bigsqcap_{d \in D} c \Cap d), D \neq \setof{}
\\ a \Cap b &=& a \sqcup b
\\ t \Cap t' &=& t \sqcup t'
\\ (a;c) \Cap (b;d) &=& (a \Cap b);(c \Cap d)
\\ (a;c) \Cap \nil &=& \top
\\ a^\infty \Cap b^\infty &=& (a \Cap b)^\infty
\\ a \Cap \alf &=& a
\\ \chaos &\defs& \alf^\omega
\\ a^\omega \Cap b^\omega &=& (a \Cap b)^\omega
\\ (\piRestrict~ g_1) \Cap (\piRestrict~g_2) &=& (\piRestrict(g_1 \cap g_2))
\\ a^\omega \Cap (c;d) &=& (a^\omega \Cap c);(a^\omega \Cap d)
\\ (guar~g) \Cap (c;d) &=& (\guar~g \Cap c) ; (\guar~g \Cap d)
}

\RLEQNS{
   (\epsAssm~r) &\defs& \assume(!\epsilon(\overline r))
\\ &=& !\epsilon(\overline r) \sqcap \epsilon(\overline r);\bot
\\ \rely~r &\defs& (\epsAssm~r)^\omega
\\ \assume~a \Cap \assume~b &=& \assume(a \sqcup b)
\\ (\rely~r) \Cap (c;d) &=& (\rely~r \Cap c);(\rely~r \Cap d)
}

Rely-Guarantee quintuple: $\setof{p,r}c\setof{g,q}$

\RLEQNS{
   \term &\defs& \epsilon^\omega (\pi;\epsilon^\omega)^\star
\\ ~[q]
   &\defs&
   \bigsqcap_{\sigma\in\Sigma}
    \tau(\setof{\sigma})
    ; \term
    ; \tau(\setof{\sigma'\in\Sigma|(\sigma,\sigma')\in q})
\\ \setof{p,r}c\setof{g,q}
   &\defs&
   \Pre p ;(\rely~r \Cap \guar~g \Cap [q]) \sqsubseteq c
}

\HDRc{Abstract Communication in Process Algebras}~

% \newpage
% \HDRa{Rely-Guarantee Algebra Redux}\label{ha:RGAlg-redux}


\HDRb{From the FM2016 Tutorial}
\RLEQNS{
       & \top
\\ \nil &      & \alf
\\     & \chaos
\\ & \bot
}

\begin{center}
\begin{tabular}{|c|c|c|c|c|c|}
  \hline
    & assoc & comm & idem & unit & zero
  \\\hline
  $\sqcap$ & \checkmark & \checkmark & \checkmark & $\top$ & $\bot$
  \\\hline
  $\sqcup$ & \checkmark & \checkmark & \checkmark & $\bot$ & $\top$
  \\\hline
\end{tabular}
\end{center}

\RLEQNS{
   r \subseteq \Sigma \times \Sigma
\\ π(r) &=& \Pi(\sigma,\sigma'), (\sigma,\sigma') \in r
\\ ϵ(r) &=& \mathcal{E}(\sigma,\sigma'), (\sigma,\sigma') \in r
\\ \stutter &=& \pi(\id)
\\ \pi &=& \pi(\univ)
\\ \epsilon &=& \epsilon(\univ)
\\ p &\subseteq& \Sigma
\\ τ(p) &=& \mbox{if $p$ then terminate else $\top$}
\\ \pre~ t &=& t \sqcap \lnot t \bot
\\  &=& t \sqcap (\lnot t) \seq \bot
\\ \setof p &=& \pre~\tau(p)
\\ &=& \tau(p) \sqcap \tau(\overline{p})\bot
\\ !  && \mbox{not sure what this is}
\\ \assume~ a &=& a \sqcap (!a) \bot
\\ \pi(\emp) &=& \top
\\ \epsilon(\emp) &=& \top
\\ \tau(\emp) &=& \top
\\ \tau(\Sigma) &=& \nil
\\ \tau(p_1) \sqcap \tau(p_2) &=& \tau(p_1 \cup p_2)
\\ \tau(p_1) \sqcup \tau(p_2) &=& \tau(p_1 \cap p_2)
\\                            &=& \tau(p_1)\tau(p_2)
\\                            &=& \tau(p_1)\parallel\tau(p_2)
\\ \lnot\tau(p) &=& \tau(\overline p)
\\ \assume~\pi \sqcap \epsilon(r)
   &=&
   \pi \sqcap \epsilon(r) \sqcap \epsilon(\overline{r})\bot
}

\newpage
\HDRb{From the FM2016 (joint-Best) Paper}

\HDRc{Introduction}

Assume $a$, $b$ atomic, $c$, $d$ arbitrary processes.
\RLEQNS{
   (a;c)\parallel(b;d) &=& (a\parallel b);(c\parallel d)
\\ (a;c)\ileave(b;d) &=& a;(c\ileave b;d) \sqcap b;(a;c\ileave d)
}

\HDRc{Concurrent Refinement Algebra}~

Concurrent Refinement Algebra (CRA):
\[
(\mathcal C,\sqcap,\sqcup,;,\parallel,\bot,\top,\nil,\Skip)
\]
Complete, distributive lattice:
$
(\mathcal C,\sqcap,\sqcup,\bot,\top)
$.
\RLEQNS{
   c \sqsubseteq d &\defs& (c \sqcap d) = c
\\ \bot \quad \sqsubseteq &c& \sqsubseteq \quad \top
}
Monoid:
$
  (\mathcal C, ;, \nil)
$.
\RLEQNS{
   \top ; c &=& \top
\\ \bot ; c &=& \bot
\\ c ; \top &\neq& \top
\\ c ;\bot &\neq& \bot
\\ (\bigsqcap C) ; d &=& \bigsqcap_{c \in C}(c;d)
\\ c^0 &\defs& \nil
\\ c^{i+1} &\defs& c ; c^i
\\ c^\star &\defs& \nu x . \nil \sqcap c ; x
\\ c^\omega &\defs& \mu x . \nil \sqcap c ;x
\\ c^\infty &\defs& c^\omega ; \top
\\ c^\omega &=& \nil \sqcap c ; c^\omega
\\ c^\star &=& \nil \sqcap c ; c^\star
\\ c^\infty &=& c ; c^\infty ~=~ c^i ; c^\infty ~=~ c^\infty ; d
}
True in their relational model, but not generally in CCS or CSP:
\RLEQNS{
   D \neq \setof{} &\implies& c;(\bigsqcap D) = \bigsqcap_{d \in D}(c;d)
}
It says that ; is \emph{conjunctive}.
Needed for the following:
\RLEQNS{
   c^\omega &=& c^\star \sqcap c^\infty
\\ c^\star &=& \bigsqcap_{i \in \Nat} c^i
\\ c^\omega ; d &=& c^\star;d \sqcap c^\infty
\\ c;c^\omega;d &=& c;c^\star;d \sqcap c^\infty
}

\HDRc{The Boolean Sub-algebra of Tests}~

Test commands: $t \in \mathcal B \subseteq C$, extended algebra:
\[
(\mathcal C,\mathcal B,\sqcap,\sqcup,;,\parallel,\bot,\top,\nil,\Skip,\lnot)
\]
Test Boolean algebra --- sub-lattice of CRA:
$
(\mathcal B,\sqcap,\sqcup,\lnot,\top,\nil)
$

$\mathcal B$ closed under $\sqcap, \sqcup, ;, \parallel$.

Assume $t \in \mathcal B$, arbitrary test.
\RLEQNS{
   t;t' &=& t \sqcup t'
\\ t\parallel t' &=& t \sqcup t'
\\ (t;c) \parallel (t;d) &=& t;(c\parallel d)
\\ (t;c) \sqcup (t';d) &=& (t \sqcup t') ; (c \sqcup d)
\\ \Assert~t &\defs& t \sqcap \lnot t ; \bot
\\ \lnot \top &=& \nil
}

\HDRc{Abstract Atomic Steps}~

Atomic Steps commands: $a,b \in \mathcal A \subseteq C$.

Atomic Action Boolean algebra --- sub-lattice of CRA:
$
(\mathcal A,\sqcap,\sqcup,!,\top,\alf)
$
\RLEQNS{
   \alf \sqcup \nil &=& \top
}

$\mathcal A$ closed under $\sqcap, \sqcup, \parallel$, but not $;$.

\RLEQNS{
   a \parallel \wait &=& a
\\ a;c \parallel b;d &=& (a \parallel b);(c\parallel d)
\\ a;c \sqcup b;d &=& (a \sqcup b);(c \sqcup d)
\\ a;c \parallel \nil &=& \top
\\ a;c \sqcup \nil &=& \top
\\ a \sqcup !a &=& \top
\\ a \sqcap !a &=& \alf
\\ !\top &=& \alf
\\ \assume~a &\defs& a \sqcap (!a);\bot
}

Given any $c$ there are $t$, $t'$, $I$, $a_i$ and $c_i$ such that:
\RLEQNS{
   c &=& t \sqcap t';\bot \sqcap \bigsqcap_{i \in I}(a_i ; c_i)
\\ \Skip &\defs& \wait^\omega
\\ \wait^\omega \parallel c &=& c
\\ a^\star\parallel \nil &=& \nil
\\ a^\omega\parallel \nil &=& \nil
\\ a^\infty\parallel \nil &=& \top
\\ a^i;c \parallel b^i;d &=& (a\parallel b)^i ; (c \parallel d)
}
If ; is conjunctive:
\RLEQNS{
   a^\star \parallel b^\star &=& (a \parallel b)^\star
\\ a^\infty \parallel b^\infty &\defs?& (a \parallel b)^\infty
\\ a^\star;c \parallel b^\star;d
   &=&
   (a \parallel b)^\star
   ;
   ( (c \parallel d)
     \sqcap
     (c \parallel b;b^\star;d)
     \sqcap
     (a;a^\star;c \parallel d) )
\\ a^\star;c \parallel b^\infty
   &=&
   (a\parallel b)^\star; (c\parallel b^\infty)
\\ a^\omega;c \parallel b^\omega;d
   &=&
   (a \parallel b)^\omega
   ;
   ( (c \parallel d)
     \sqcap
     (c \parallel b;b^\omega;d)
     \sqcap
     (a;a^\omega;c \parallel d) )
\\ \action a &\defs& \wait^\omega ; a; \wait^\omega
\\ \action a \parallel \action b
   &=&
   \action{a\parallel b}
   \sqcap \action a ; \action b
   \sqcap \action b ; \action a
\\ a \ileave b &=& a;b \sqcap b;a
}


\HDRc{Relational Atomic Steps}~

\RLEQNS{
   \sigma &\in& \Sigma
\\ r &\in& \Set(\Sigma\times\Sigma)
\\ \pi &:& \Set(\Sigma\times\Sigma) \fun \mathcal A
\\ \epsilon &:& \Set(\Sigma\times\Sigma) \fun \mathcal A
\\ \pi(\emptyset) ~~= &\top& =~~ \epsilon(\emptyset)
\\ \pi(r_1) \sqcup \epsilon(r_2) &=& \top
}
For $s \in \setof{\pi,\epsilon}$:
\RLEQNS{
   r_1=r_2 &\Leftrightarrow& s(r_1)=s(r_2)
\\ s(r_1 \cup r_2) &=& s(r_1) \sqcap s(r_2)
\\ s(r_1 \cap r_2) &=& s(r_1) \sqcup s(r_2)
\\ r_1 \subseteq r_2 &\implies& s(r_2) \sqsubseteq s(r_1)
}

\RLEQNS{
   p &\in& \Set\Sigma
\\ \tau &:& \Set\Sigma \fun \mathcal B
\\ \tau(\emptyset) &=& \top
\\ \tau(\Sigma) &=& \nil
\\ \Pre p &\defs& \Assert~\tau(p)
\\     &  =  & \tau(p) \sqcap \tau(\lnot p);\bot
\\ \Pre\emptyset&=& \bot
\\ \Pre\Sigma &=& \nil
}

\HDRc{Relies and Guarantees}~

\RLEQNS{
   g &\in& \Set(\Sigma\times\Sigma)
\\ (\piRestrict~g) &\defs& \pi(g) \sqcap \wait
\\ \guar~g &\defs& (\piRestrict~g)^\omega
\\ g_1 \subseteq q_2 &\implies& (\piRestrict~g_2) \sqsubseteq (\piRestrict~g_1)
}

\RLEQNS{
   c \Cap \bot &=& \bot
\\ (c \Cap c') \Cap c'' &=& c \Cap (c' \Cap c'')
\\ c \Cap d &=& d \Cap c
\\ c \Cap c &=& c
\\ c \Cap (\bigsqcap D) &=& (\bigsqcap_{d \in D} c \Cap d), D \neq \setof{}
\\ a \Cap b &=& a \sqcup b
\\ t \Cap t' &=& t \sqcup t'
\\ (a;c) \Cap (b;d) &=& (a \Cap b);(c \Cap d)
\\ (a;c) \Cap \nil &=& \top
\\ a^\infty \Cap b^\infty &=& (a \Cap b)^\infty
\\ a \Cap \alf &=& a
\\ \chaos &\defs& \alf^\omega
\\ a^\omega \Cap b^\omega &=& (a \Cap b)^\omega
\\ (\piRestrict~ g_1) \Cap (\piRestrict~g_2) &=& (\piRestrict(g_1 \cap g_2))
\\ a^\omega \Cap (c;d) &=& (a^\omega \Cap c);(a^\omega \Cap d)
\\ (guar~g) \Cap (c;d) &=& (\guar~g \Cap c) ; (\guar~g \Cap d)
}

\RLEQNS{
   (\epsAssm~r) &\defs& \assume(!\epsilon(\overline r))
\\ &=& !\epsilon(\overline r) \sqcap \epsilon(\overline r);\bot
\\ \rely~r &\defs& (\epsAssm~r)^\omega
\\ \assume~a \Cap \assume~b &=& \assume(a \sqcup b)
\\ (\rely~r) \Cap (c;d) &=& (\rely~r \Cap c);(\rely~r \Cap d)
}

Rely-Guarantee quintuple: $\setof{p,r}c\setof{g,q}$

\RLEQNS{
   \term &\defs& \epsilon^\omega (\pi;\epsilon^\omega)^\star
\\ ~[q]
   &\defs&
   \bigsqcap_{\sigma\in\Sigma}
    \tau(\setof{\sigma})
    ; \term
    ; \tau(\setof{\sigma'\in\Sigma|(\sigma,\sigma')\in q})
\\ \setof{p,r}c\setof{g,q}
   &\defs&
   \Pre p ;(\rely~r \Cap \guar~g \Cap [q]) \sqsubseteq c
}

\HDRc{Abstract Communication in Process Algebras}~


\newpage
\input{/proglogcalc/src/DictAbstractions.lhs}

\newpage
\bibliography{UTPCALC}

\end{document}
