\HDRa{Math support for Views}\label{ha:mViews}

\HDRb{The Big Plan}

We go for a new formulation
\RLEQNS{
   X(E|a|R|A) &\defs&
   ls(E) \land s' \in \sem a s \land ls' = (ls\setminus R)\cup A
\\ A(E|a|N) &\defs& X(E|a|E|N)
\\ \W(P) &\defs& \true * (\Skip \lor P)
\\       &=& \bigvee_{i \in \Nat} \Skip \seq P^i
\\ atm(a) &=& \W(A(in|a|out) \land [in|out])
\\ C &=& \W(actions(C) \land [in|out|g] \land I_C)
}
where $I_C$ is some more $C$-specific invariants.

\HDRb{Set Inclusion/Membership}

An atomic action $A(E|a|N)$ is enabled if $E$ is contained
in the global label-set ($ls(E)$)
and results in $E$ being removed from that set, and new labels
$N$ being added ($ls'=(ls\setminus E)\cup N$).
We need a way to reason about containment in such an $ls'$
in terns of $E$ and $N$, and to compute sequential compositions
of such forms, which will take the more general form $X(E|a|R|A)$.

We find we get assertions of the form $(F(ls))(E)$,
asserting that $E$ is a subset of $F(ls)$ where $F$ is a set-function
composed of named/enumerated sets and standard set-operations.
We want to transform it into $ls(G) \land P$ where $G$ and $P$
do not involve $ls$.

We present the laws,
then the proofs
\RLEQNS{
   (ls \cup A)(S)      &=& ls(S\setminus A)
\\ (ls \setminus R)(S) &=& ls(S) \land S \cap R = \emptyset
\\ (ls \cap M)(S)      &=& ls(S) \land S \subseteq M
\\ ((ls \setminus R) \cup A)(S)
   &=& ls(S \setminus A) \land (S \setminus A) \cap R = \emptyset
\\ (((ls\setminus R_1) \cup A_1)\setminus R_2) \cup A_2
  &=& (ls \setminus (R_1 \cup R_2)) \cup ((A_1\setminus R_2) \cup A_2)
}

We do the proofs in ``classical'' set notation
\RLEQNS{
  && S \subseteq (ls \cup A)
\EQ{set definitions}
\\&& x \in S \implies x \in (ls \cup A)
\EQ{defn $\cup$}
\\&& x \in S \implies x \in ls \lor x \in A
\EQ{defn $\implies$}
\\&& x \notin S \lor x \in ls \lor x \in A
\EQ{rearrange}
\\&& x \notin S \lor x \in A \lor x \in ls
\EQ{De-Morgan, defn $\implies$}
\\&& (x \in S \land x \notin A) \implies x \in ls
\EQ{defn subset}
\\&& (S \setminus A) \subseteq ls
}

\RLEQNS{
  && S \subseteq (ls \setminus R)
\EQ{set definitions}
\\&& x \in S \implies x \in (ls \setminus R)
\EQ{set definition}
\\&& x \in S \implies x \in ls \land x \notin R
\EQ{defn $\implies$}
\\&& x \notin S \lor x \in ls \land x \notin R
\EQ{distribution}
\\&& (x \notin S \lor x \in ls)
     \land
     ( x \notin S \lor x \notin R)
\EQ{defn implies, de-morgan}
\\&& (x \in S \implies x \in ls)
     \land
     \lnot( x \in S \land x \in R)
\EQ{defn subset}
\\&&S \subseteq ls \land S \cap R = \emptyset
}

\RLEQNS{
  && S \subseteq (ls \cap M)
\EQ{set definitions}
\\&& x \in S \implies x \in ls \land x \in M
\EQ{defn $\implies$}
\\&& x \notin S \lor x \in ls \land x \in M
\EQ{distribution}
\\&& (x \notin S \lor x \in ls)
     \land
     (x \notin S \lor x \in M)
\EQ{defn $implies$}
\\&& (x \in S \implies x \in ls)
     \land
     (x \in S \implies x \in M)
\EQ{def subset}
\\&& S \subseteq ls \land S \subseteq M
}

\RLEQNS{
  && ((ls \setminus R) \cup A)(S)
\EQ{laws above}
\\&& (ls \setminus R)(S \setminus A)
\EQ{laws above}
\\&& ls(S \setminus A) \land (S \setminus A) \cap R = \emptyset
}

\RLEQNS{
  && x \in (ls\setminus R_1) \cup A_1
\EQ{defn $\cup$}
\\&& x \in (ls\setminus R_1) \lor x \in  A_1
\EQ{defn $\setminus$}
\\&& x \in ls \land x \notin R_1 \lor x \in  A_1
}

\RLEQNS{
  && x \in (((ls\setminus R_1) \cup A_1)\setminus R_2) \cup A_2
\EQ{above law}
\\&& x \in (ls\setminus R_1) \cup A_1) \land x \notin R_2 \lor x \in  A_2
\EQ{above law}
\\&& (x \in ls \land x \notin R_1 \lor x \in  A_1) \land x \notin R_2 \lor x \in  A_2
\EQ{distributivity}
\\&& (x \in ls \land x \notin R_1 \land x \notin R_2)
     \lor
     (x \in  A_1 \land x \notin R_2)
     \lor x \in  A_2
\EQ{de-Morgan, defn $\setminus$}
\\&& (x \in ls \land \lnot(x \in R_1 \lor x \in R_2))
     \lor
     x \in  A_1 \setminus R_2
     \lor x \in  A_2
\EQ{defn $\cup$, twice}
\\&& (x \in ls \land \lnot(x \in R_1 \cup R_2))
     \lor
     x \in  A_1 \setminus R_2 \cup  A_2
\EQ{tweak}
\\&& (x \in ls \land x \notin R_1 \cup R_2)
     \lor
     x \in  A_1 \setminus R_2 \cup  A_2
\EQ{definition of $\setminus$}
\\&& x \in (ls \setminus R_1 \cup R_2)
     \lor
     x \in  A_1 \setminus R_2 \cup  A_2
\EQ{definition of $cup$}
\\&& x \in (ls \setminus R_1 \cup R_2)
     \cup
     (A_1 \setminus R_2 \cup  A_2)
}

\HDRb{Basic Actions}

We define a basic operation form that is closed
under sequential composition, modulo some `ground' side-conditions.
\RLEQNS{
   X(E|a|R|A)
   &\defs&
   ls(E) \land [a] \land ls' = (ls\setminus R) \cup A
}
Composing these requires us decouple the enabling labels
from those removed---they are the same for a basic action,
but differ as they are sequentially composed.

The general composition:
\RLEQNS{
  && X(E_1|a|R_1|A_1)\seq X(E_2|b|R_2|A_2)
\EQ{Defn $X$}
\\&& ls(E_1) \land [a] \land ls' = (ls\setminus R_1) \cup A_1
     \quad\seq\quad
     ls(E_2) \land [b] \land ls' = (ls\setminus R_2) \cup A_2
\EQ{Defn $\seq$}
\\&& \exists s_m,ls_m @
\\&& \quad ls(E_1) \land
           [a][s_m/s'] \land
           ls_m = (ls\setminus R_1) \cup A_1 \land {}
\\&& \quad ls_m(E_2) \land
           [b][s_m/s] \land
           ls' = (ls_m\setminus R_2) \cup A_2
\EQ{1-pt rule ($ls_m$), re-arrange}
\\&& \exists s_m @
\\&& \quad ls(E_1) \land
           ((ls\setminus R_1) \cup A_1)(E_2) \land {}
\\&& \quad [a][s_m/s'] \land
           [b][s_m/s] \land {}
\\&& \quad ls' = (((ls\setminus R_1) \cup A_1)\setminus R_2) \cup A_2
\EQ{shrink $s_m$ scope, use $ls(-)$ laws above}
\\&& ls(E_1) \land
     ls(E_2\setminus A_1) \land
     (E_2\setminus A_1) \cap R_1 = \emptyset \land {}
\\&& (\exists s_m @ [a][s_m/s'] \land [b][s_m/s]) \land {}
\\&& ls' = (((ls\setminus R_1) \cup A_1)\setminus R_2) \cup A_2
\EQ{defn $\seq$, prop of $ls(-)$, simplification from above}
\\&& ls(E_1 \cup (E_2\setminus A_1)) \land
     (E_2\setminus A_1) \cap R_1 = \emptyset \land {}
\\&& [a;b] \land {}
\\&& ls' = (ls \setminus R_1 \cup R_2)
           \cup
           (A_1 \setminus R_2 \cup  A_2)
\EQ{defn of $X$}
\\&& X(E_1 \cup (E_2\setminus A_1)
       |a\seq b
       |R_1 \cup R_2
       |A_1 \setminus R_2 \cup  A_2)
       \land (E_2\setminus A_1) \cap R_1 = \emptyset
}
Here the `ground' side-condition is
 $(E_2\setminus A_1) \cap R_1 = \emptyset$.

Composing a basic operation with itself:
\RLEQNS{
  && X(E|a|R|A) \seq X(E|a|R|A)
\EQ{by above}
\\&& X(E \cup (E\setminus A)
       |a\seq a
       |R \cup R
       |A \setminus R \cup  A)
       \land (E\setminus A) \cap R = \emptyset
\EQ{simpify}
\\&& X(E|a\seq a|R|A)
       \land (E\setminus A) \cap R = \emptyset
}

A basic action identifies $E$ and $R$:
\RLEQNS{
  && A(E|a|N) \defs X(E|a|E|N)
\\
\\&& A(E_1|a|N_1) \seq A(E_2|a|N_2)
\EQ{by defn}
\\&& A(E_1|a|E_1|N_1) \seq A(E_2|a|E_2|N_2)
\EQ{by calc above, with $R_i = E_i$}
\\&& X(E_1 \cup (E_2\setminus A_1)
       |a\seq b
       |E_1 \cup E_2
       |A_1 \setminus E_2 \cup  A_2)
       \land (E_2\setminus A_1) \cap E_1 = \emptyset
}
We cannot transform this into the form $A(-|a\seq b|-)$
unless it happens to be the case that
\\$E_1 \cup (E_2\setminus A_1)= E_1 \cup E_2$.
For an $A$ composed with itself:
\RLEQNS{
  && A(E|a|N) \seq A(E|a|N)
\EQ{defn $A$}
\\&& X(E|a|E|N) \seq X(E|a|E|N)
\EQ{calc above}
\\&& X(E|a\seq a|E|A)
       \land (E\setminus A) \cap E = \emptyset
}
We see that we get a $\false$ result unless we have $E \subseteq A$,
in which everything we remove is added back in, plus possibly some extra.
We get the following which is similar
\[
  X(E|a\seq a|\emptyset|A)
\]
This is a consequence of the following easy law:
\[
 (ls \setminus R) \cup A
 =
 (ls \setminus (R \setminus A)) \cup A
 =
 (ls \cup A) \setminus (R \setminus A)
\]

Following on, we can pre-compute mixed combinations:
\RLEQNS{
  && X(E_1|a|R_1|A_1) \seq A(E_2|b|N_2)
\EQ{defn $A$}
\\&& X(E_1|a|R_1|A_1) \seq X(E_2|b|E_2|N_2)
\EQ{law for $X$ above}
\\&& X(E_1 \cup (E_2 \setminus A_1)
      | a\seq b
      | R_1 \cup E_2
      | A_1 \setminus E_2 \cup N_2)
     \land
     (E_2 \setminus A_1) \cap R_1 = \emptyset
}
and flipped:
\RLEQNS{
  && A(E_1|a|N_1) \seq X(E_2|b|R_2|A_2)
\EQ{defn $A$}
\\&& X(E_1|a|E_1|N_1) \seq X(E_2|b|R_2|A_2)
\EQ{law for $X$ above}
\\&& X(E_1 \cup (E_2 \setminus N_1)
      | a\seq b
      | E_1 \cup R_2
      | N_1 \setminus R_2 \cup A_2)
     \land
     (E_2 \setminus N_1) \cap E_1 = \emptyset
}

\HDRc{Laws of Actions}
Here we summarise the main results:
\RLEQNS{
  && X(E_1|a|R_1|A_1)\seq X(E_2|b|R_2|A_2)
\EQ{proof above}
\\&& X(E_1 \cup (E_2\setminus A_1)
       \mid a\seq b
       \mid R_1 \cup R_2
       \mid A_1 \setminus R_2 \cup  A_2)
\\&& {} \land (E_2\setminus A_1) \cap R_1 = \emptyset
\\
\\&& \\&& A(E_1|a|N_1) \seq A(E_2|a|N_2)
\EQ{proof above}
\\&& X(E_1 \cup (E_2\setminus N_1)
       \mid a\seq b
       \mid E_1 \cup E_2
       \mid N_1 \setminus E_2 \cup  N_2)
\\&& {} \land (E_2\setminus N_1) \cap E_1 = \emptyset
\\
\\&& A(E|a|N) \seq A(E|a|N)
\EQ{proof above}
\\&& X(E|a\seq a|E|N)   \land   (E\setminus N) \cap E = \emptyset
\EQ{simplify}
\\&& X(E|a\seq a|\emptyset|N)   \land   E \subseteq N
}



\newpage
\HDRb{Working with the Invariant}

We have introduced the following notation:
\[
  [ L_1 | L_2 | \dots | L_n ]
\]
Its first intended meaning is to assert that
all the $L_i$ are mutually disjoint:
\RLEQNS{
   \forall i,j \in 1\dots n @ i \neq j \implies L_i \cap L_j = \emptyset
}
It also states that if any one of its members overlaps with $ls$,
then none of the others do (and similarly for $ls'$):
\RLEQNS{
  &&   \forall i,j \in 1\dots n @
        i \neq j \land L_i \cap ls \neq \emptyset
        \implies L_j \cap ls = \emptyset
\\&&   \forall i,j \in 1\dots n @
        i \neq j \land L_i \cap ls' \neq \emptyset
        \implies L_j \cap ls' = \emptyset
}

By \emph{design}, our labels and generator scheme
establishes the following (weakest) invariant:
\[ [in|out|g]\]
and for any generator expression $G$ we can split it into
four disjoint parts (also by design) to get
\[  [\ell_G|G_{:}|G_1|G_2] . \]
So we can always strengthen our invariant by splitting some $G$
in this way:
\RLEQNS{
  && [in|out|g]
\EQ{split $g$}
\\&& [in|out|\ell_g|\g:|\g1|\g2]
\EQ{split $\g:$}
\\&& [in|out|\ell_g| \ell_{g:}|\g{::}|\g{:1}|\g{:2}|\g1|\g2]
}
However, sometimes we don't want to do this.
The hope is that by requiring $[in|out|g]$ at each level,
that we get the right level of exclusivity vs. sharing of $ls$
by generated labels.

\HDRc{Nested Invariants}


All of these comments above are true if the invariant is just about disjointness.
It is not true in general when we consider exclusivity.
For example in parallel execution it is perfectly reasonable to have both
$\ell_{g1}$ and $\ell_{g2:}$ (or $\ell_{g2}$) together in the label-set,
but not both $\ell_{g1}$ and $\ell_{g1:}$
or both $\ell_{g2}$ and $\ell_{g2:}$.

Notationally, we can represent this as follows:
$$
[in \mid [\ell_{g1}|\ell_{g1:}],[\ell_{g2}|\ell_{g2:}] \mid out].
$$
In effect this declares, at the top, that we have
$$
[in \mid \ell_{g1},\ell_{g1:},\ell_{g2},\ell_{g2:} \mid out]
$$
with finer grained details for the generated labels,
where a comma is a form of set union,
so we can have something from
either or both of $[\ell_{g1}|\ell_{g1:}]$ and $[\ell_{g2}|\ell_{g2:}]$
but within each of these we have mutual exclusion.

In effect we have a tree-like structure with ``$|$''-branches
and ``$,$``-branches, and labels as leaves.
We should be able to flatten cases where a subnode has the same branch type,
so
\RLEQNS{
   [L_1 \mid [L_{21}|L_{22}] \mid L_3]
   &=&
   [L_1|L_{21}|L_{22}|L_3]
\\ L_1 , ( L_{21}, L_{22})  , L_3
   &=&
   L_1,L_{21},L_{22},L_3
}



Invariants are based on the following general construct:
\RLEQNS{
   EXC_{i=1}^n A_i
   &\defs&
   \forall i,j \in 1\dots n @ i\neq j \land A_i \implies \lnot A_j
}
which we can also write as $EXC(A_1,\dots,A_n)$.
A quick calculation shows that
(using hardware logic notation for compactness):
\[
 EXC(A,B,C) = \B A ~ \B B  + \B A ~ \B C + \B B ~ \B C
\]
Do the following laws hold?
\RLEQNS{
   EXC(EXC(A,B),EXC(B,C)) &=& EXC(A,B,C)
\\ EXC(EXC(A,B),EXC(C,D)) &=& EXC(A,B,C,D)
}
No - the first lhs asserts that either $A$ and $B$ are exclusive,
or $B$ and $C$ are exclusive, but not both,
i.e. we have $EXC(A,B) \implies \lnot EXC(B,C)$.

How about:
\RLEQNS{
   EXC(A,B) \land EXC(B,C) &=& EXC(A,B,C)
\\ EXC(A,B) \land EXC(C,D) &=& EXC(A,B,C,D)
}
No, The first doesn't force the exclusivity of $A$ and $C$.

We need full information, so the following is required:
\RLEQNS{
   EXC(A,B) \land EXC(B,C) \land EXC(A,C) &=& EXC(A,B,C)
}
This works

Not sure interpreting $[L|M]$ or $EX(L,M)$ as predicates is helpful.
It might be better being viewed as a tree and defining a satisfaction relation
between it an a label-set.


\HDRc{Using $A$ with invariants}

Noting that $[L_1|L_2|\dots]$ implies that if $ls(L_1)$ is true,
then $ls(L_2)$ is false:
\RLEQNS{
  && X(L_1|a|L_1,L_2|L_3) & [L_1|L_2|\dots]
\EQ{defn $X$}
\\&& ls(L_1) \land [a] \land ls'=(ls\setminus(L_1 \cup L_2)) \cup L_3
\EQ{no need to remove $L_2$ from $ls$ if it is not there}
\\&& ls(L_1) \land [a] \land ls'=(ls\setminus L_1) \cup L_3
\EQ{defn $X$}
\\&& X(L_1|a|L_1|L_3)
\EQ{defn $A$}
\\&& A(L_1|a|L_3)
}


Two actions meant to work together:
\RLEQNS{
  && A(L_1|a|L_2) \seq A(L_2|b|L_3) & [L_1|L_2|L_3]
\EQ{law $A^2$}
\\&& L_2\setminus L_2 \cap L_1 = \emptyset \land {}
\\&& X(    L_1 \cup L_2\setminus L_2
      \mid a\seq b
      \mid L_1 \cup L_2
      \mid L_2\setminus L_2 \cup L_3 )
\EQ{simplify}
\\&& X(    L_1
      \mid a\seq b
      \mid L_1 \cup L_2
      \mid  L_3 )
\EQ{prev law, given invariant}
\\&& X( L_1 |  a\seq b | L_1 | L_3 )
\EQ{defn $A$}
\\&& A( L_1 |  a\seq b | L_3 )
}

Two actions meant to be mutually exclusive (the hint being the invariant):
\RLEQNS{
  && A(L_1|a|L_2) \seq A(L_3|b|L_4) & [L_1|L_2|L_3|L_4]
\EQ{law $A^2$}
\\&& L_3\setminus L_2 \cap L_1 = \emptyset \and {}
\\&& X(    L_1 \cup L_3\setminus L_2
      \mid a\seq b
      \mid L_1 \cup L_3
      \mid L_2\setminus L_3 \cup L_4 )
\EQ{simplify, noting invariant}
\\&& X(    L_1 \cup L_3
      \mid a\seq b
      \mid L_1 \cup L_3
      \mid L_2 \cup L_4 )
\EQ{Falsifies $[L_1|L_3|\ldots]$}
\\&& \false
}
This law also works with invariant fragments
$[L_1|L_3|\dots]$ and $[L_2|L_4|\dots]$

\newpage
\HDRb{Semantic Definitions}

\RLEQNS{
   X(E|a|R|A) &\defs&
   ls(E) \land s' \in \sem a s \land ls' = (ls\setminus R)\cup A
\\ A(E|a|N) &\defs& X(E|a|E|N)
\\ \W(P) &\defs& \true * (\Skip \lor P)
\\       &=& \bigvee_{i \in \Nat} \Skip \seq P^i
\\
\\ atm(a) &=& \W(A(in|a|out)) \land [in|out]
\\
\\ C \cseq D
   &\defs&
   \W( C[\ell_g,\g{:1}/out,g] \lor
       D[\ell_g,\g{:2}/in,g] ) \land [in|\ell_g|out]
\\
\\ C + D
   &\defs&
   \W(~ A(in|ii|\ell_{g1}) \lor
        A(in|ii|\ell_{g2}) \lor {}
\\&& ~~ C[g_{1:},\ell_{g1}/g,in] \lor
       D[g_{2:},\ell_{g2}/g,in]~ )
\\&& {}\land [in|\ell_{g1}|\ell_{g2}|out]
\\
\\ C \parallel D
   &\defs&
   \W(~
      A(in|ii|\ell_{g1},\ell_{g2}) \lor
      A(\ell_{g1:},\ell_{g2:}|ii|out) \lor {}
\\&& ~~
       C[g_{1::},\ell_{g1},\ell_{g1:}/g,in,out] \lor
       D[g_{2::},\ell_{g2},\ell_{g2:}/g,in,out] ~)
\\&& {} \land [in \mid [\ell_{g1}|\ell_{g1:}],[\ell_{g2}|\ell_{g2:}] \mid out]
\\
\\ C^*
   &\defs&
   \W(~ A(in|ii|out) \lor
       A(in|ii|\ell_g) \lor
       C[\g:,\ell_g,in/g,in,out] ~)
\\&& {} \land [in|\ell_g|out]
}

We note that $\W(P) \land I = \W(P \land I)$ for our invariants.

\newpage
\HDRb{Semantic Calculations}

Semantic calculations will be based on the following form:
\[
  \W(P) = \Skip \lor \left(\bigvee_{i=1,\dots} P^i\right)
\]
So all we need to do is to compute $P^i$ for $i>1$
until we get either $\false$ or a prior result as outcome.
The semantics is the the disjunction of all of the results.

Note: where iteration is concerned, $P^i$ may never vanish
or converge, so the treatment there will be a little different.


\HDRc{Atomic Action}

\[ atm(a) \defs \W(A(in|a|out)) \land [in|out] \]

\[ P = A(in|a|out) \qquad I = [in|out] \]

\RLEQNS{
  && P^2
\EQ{expand $P$}
\\&& A(in|a|out) \seq A(in|a|out)
\EQ{$A^2$ law}
\\&& X(in|a\seq a|\emptyset|out) \land in \subseteq out
\EQ{$I$ implies $in$, $out$ are disjoint}
\\&& \false
}

So, we can declare that:
\RLEQNS{
  && atm(a)
\EQ{calculations above}
\\&& (~\Skip \lor A(in|a|out)~) \land [in|out]
}

\newpage
\HDRc{Sequential Composition}

\[
  C \cseq D
   \defs
   \W( C[\ell_g,\g{:1}/out,g] \lor
       D[\ell_g,\g{:2}/in,g] ) \land [in|\ell_g|out]
\]

\[ P =  atm(a)[\ell_g,\g{:1}/out,g] \lor
       atm(b)[\ell_g,\g{:2}/in,g]\qquad I = [in|\ell_g|out] \]

\RLEQNS{
  && P
\EQ{expand $P$}
\\&& atm(a)[\ell_g,\g{:1}/out,g]
     \lor
     atm(b)[\ell_g,\g{:2}/in,g]
\EQ{expand $atm$s}
\\&& ((\Skip \lor A(in|a|out)) \land [in|out])[\ell_g,\g{:1}/out,g]
     \lor {}
\\&& ((\Skip \lor A(in|b|out)) \land [in|out])[\ell_g,\g{:2}/in,g]
\EQ{substitution}
\\&& (\Skip \lor A(in|a|\ell_g)) \land [in|\ell_g]
     \lor {}
\\&& (\Skip \lor A(\ell_g|b|out) \land [\ell_g|out]
\EQ{$I$ subsumes both $atm$ invariants}
\\&& \Skip \lor A(in|a|\ell_g)
     \lor
     \Skip \lor A(\ell_g|b|out)
\EQ{tidy-up}
\\&& \Skip \lor A(in|a|\ell_g) \lor A(\ell_g|b|out)
}

We note that
\[
 \Skip \lor (\Skip \lor Q) \lor (\Skip \lor Q)^2
 \lor (\Skip \lor Q)^3 \lor \dots
\]
reduces to
\[
 \Skip \lor Q \lor Q^2 \lor Q^3 \lor \dots
\]

So we proceed with $Q$
\[
  Q = A(in|a|\ell_g) \lor A(\ell_g|b|out)
  \qquad
  I = [in|\ell_g|out]
\]

\RLEQNS{
  && Q^2
\EQ{expand $Q$}
\\&& (A(in|a|\ell_g) \lor A(\ell_g|b|out))
     \seq
     (A(in|a|\ell_g) \lor A(\ell_g|b|out))
\EQ{distribute}
\\&&     A(in|a|\ell_g)
         \seq
         A(in|a|\ell_g)
\\&\lor& A(in|a|\ell_g)
         \seq
         A(\ell_g|b|out)
\\&\lor& A(\ell_g|b|out)
         \seq
         A(in|a|\ell_g)
\\&\lor& A(\ell_g|b|out)
         \seq
         A(\ell_g|b|out)
\EQ{prop $A^2$}
\\&&     in\setminus \ell_g \cap in = \emptyset \land \dots
\\&\lor& \ell_g\setminus \ell_g \cap in = \emptyset
         \land X( in\cup\ell_g\setminus \ell_g
                \mid a \seq b
                \mid in,\ell_g
                \mid \ell_g\setminus\ell_g\cup out )
\\&\lor& in\setminus out \cap \ell_g = \emptyset
         \land X(\ell_g\cup in \setminus out
                \mid b\seq a
                \mid \ell_g,in
                \mid out\setminus in \cup\ell_g )
\\&\lor& \ell_g\setminus out \cap \ell_g = \emptyset \land \dots
\EQ{simplify}
\\&&     X( in
                \mid a \seq b
                \mid in,\ell_g
                \mid out )
\\&\lor& X( \ell_g,in
                \mid b\seq a
                \mid \ell_g,in
                \mid out, \ell_g )
\EQ{2nd disjunct falsifies $[in|\ell_g|out]$}
\\&&     X( in |a \seq b | in,\ell_g | out )
\EQ{$[in|\ell_g|out]$ implies there will be no $\ell_g$ to remove}
\\&&     X( in |a \seq b | in | out )
\EQ{defn $A$}
\\&&     A( in |a \seq b | out )
}


\RLEQNS{
  && Q^3
\EQ{expand $Q\seq Q^2$}
\\&& (A(in|a|\ell_g) \lor A(\ell_g|b|out))
     \seq  X( in |a \seq b | in,\ell_g | out )
\EQ{distribute, expand $A$}
\\&&     X( in | a | in | \ell_g)
         \seq
         X( in |a \seq b | in,\ell_g | out )
\\&\lor& X( \ell_g | b | \ell_g | out)
         \seq
         X( in |a \seq b | in,\ell_g | out )
\EQ{prop $X^2$}
\\&&     in \setminus \ell_g \cap in = \emptyset \land\dots
\\&\lor& in\setminus out \cap \ell_g = \emptyset \land {}
         X( \ell_g \cup in \setminus out
          | b\seq a\seq b
          | in,\ell_g
          | out\setminus\setof{in,\ell_g} \cup out )
\EQ{simplify}
\\&& X( \ell_g,in
      | b\seq a\seq b
      | in,\ell_g
      | out )
\EQ{Falsifies $[in|\ell_g|out]$}
\\&& \false
}
So we see that $Q^n$ vanishes for $n\geq 3$.

So we have
\RLEQNS{
  && atm(a)\seq atm(b)
\EQ{$Q$ expansion}
\\&& \Skip \lor Q \lor Q^2 \lor Q^3 \lor \dots
\EQ{$Q^n = \false$ for $n \geq 3$}
\\&& \Skip \lor Q \lor Q^2
\EQ{expand $Q^i$}
\\&&     \Skip
\\&\lor& A(in|a|\ell_g) \lor A(\ell_g|b|out)
\\&\lor& A( in |a \seq b | out )
}


\newpage
\HDRc{Choice}


\RLEQNS{
   C + D
   &\defs&
   \W(~ A(in|ii|\ell_{g1}) \lor
        A(in|ii|\ell_{g2}) \lor {}
\\&& \quad~~ C[g_{1:},\ell_{g1}/g,in] \lor
       D[g_{2:},\ell_{g2}/g,in]~ )
\\&& {}\land [in|\ell_{g1}|\ell_{g2}|out]
}

\RLEQNS{
   P &=& A(in|ii|\ell_{g1}) \lor
         A(in|ii|\ell_{g2}) \lor {}
\\   & & atm(a)[g_{1:},\ell_{g1}/g,in] \lor
         atm(b)[g_{2:},\ell_{g2}/g,in]
\\
\\ I &=& [in|\ell_{g1}|\ell_{g2}|out]
}

\RLEQNS{
  && P
\EQ{expand $P$}
\\&& A(in|ii|\ell_{g1}) \lor
     A(in|ii|\ell_{g2}) \lor {}
\\&& atm(a)[g_{1:},\ell_{g1}/g,in] \lor
     atm(b)[g_{2:},\ell_{g2}/g,in]
\EQ{expand $atm$}
\\&& A(in|ii|\ell_{g1}) \lor
     A(in|ii|\ell_{g2}) \lor {}
\\&& ((\Skip \lor A(in|a|out)) \land [in|out])
     [g_{1:},\ell_{g1}/g,in] \lor {}
\\&& ((\Skip \lor A(in|b|out)) \land [in|out])
     [g_{2:},\ell_{g2}/g,in]
\EQ{substitute}
\\&& A(in|ii|\ell_{g1}) \lor
     A(in|ii|\ell_{g2}) \lor {}
\\&& (\Skip \lor A(\ell_{g1}|a|out)) \land [\ell_{g1}|out] \lor
     (\Skip \lor A(\ell_{g2}|b|out)) \land [\ell_{g2}|out]
\EQ{sub-invariants subsumed by $I$}
\\&& A(in|ii|\ell_{g1}) \lor
     A(in|ii|\ell_{g2}) \lor {}
\\&& \Skip \lor A(\ell_{g1}|a|out) \lor
     \Skip \lor A(\ell_{g2}|b|out)
\EQ{re-arrange, simplify}
\\&& \Skip \lor A(in|ii|\ell_{g1}) \lor
     A(in|ii|\ell_{g2}) \lor
     A(\ell_{g1}|a|out) \lor
     A(\ell_{g2}|b|out)
\EQ{$Q$ again}
\\&& \Skip \lor Q
}

\RLEQNS{
  && Q^2
\EQ{defn $Q$}
\\&& (~ A(in|ii|\ell_{g1}) \lor
     A(in|ii|\ell_{g2}) \lor
     A(\ell_{g1}|a|out) \lor
     A(\ell_{g2}|b|out)~) \seq{}
\\&& (~ A(in|ii|\ell_{g1}) \lor
     A(in|ii|\ell_{g2}) \lor
     A(\ell_{g1}|a|out) \lor
     A(\ell_{g2}|b|out)~)
\EQ{distribute,
    noting condition $(E_2\setminus N_1)\cap E_1=\emptyset $}
\\&    & A(in|ii|\ell_{g1}) \seq A(in|ii|\ell_{g1}) \quad \mbox{--- fail}
\\&\lor& A(in|ii|\ell_{g1}) \seq A(in|ii|\ell_{g2}) \quad \mbox{--- fail}
\\&\lor& A(in|ii|\ell_{g1}) \seq A(\ell_{g1}|a|out) \quad \mbox{--- ok}
\\&\lor& A(in|ii|\ell_{g1}) \seq A(\ell_{g2}|b|out) \quad \mbox{--- ok}
\\&\lor& A(in|ii|\ell_{g2}) \seq A(in|ii|\ell_{g1}) \quad \mbox{--- fail}
\\&\lor& A(in|ii|\ell_{g2}) \seq A(in|ii|\ell_{g2}) \quad \mbox{--- fail}
\\&\lor& A(in|ii|\ell_{g2}) \seq A(\ell_{g1}|a|out) \quad \mbox{--- ok}
\\&\lor& A(in|ii|\ell_{g2}) \seq A(\ell_{g2}|b|out) \quad \mbox{--- ok}
\\&\lor& A(\ell_{g1}|a|out) \seq A(in|ii|\ell_{g1}) \quad \mbox{--- ok}
\\&\lor& A(\ell_{g1}|a|out) \seq A(in|ii|\ell_{g2}) \quad \mbox{--- ok}
\\&\lor& A(\ell_{g1}|a|out) \seq A(\ell_{g1}|a|out) \quad \mbox{--- fail}
\\&\lor& A(\ell_{g1}|a|out) \seq A(\ell_{g2}|b|out) \quad \mbox{--- ok}
\\&\lor& A(\ell_{g2}|b|out) \seq A(in|ii|\ell_{g1}) \quad \mbox{--- ok}
\\&\lor& A(\ell_{g2}|b|out) \seq A(in|ii|\ell_{g2}) \quad \mbox{--- ok}
\\&\lor& A(\ell_{g2}|b|out) \seq A(\ell_{g1}|a|out) \quad \mbox{--- ok}
\\&\lor& A(\ell_{g2}|b|out) \seq A(\ell_{g2}|b|out) \quad \mbox{--- fail}
\EQ{drop fails, apply $A^2$ work together law w.r.t $I$}
\\&    & A(in|ii\seq a|out)
\\&\lor& A(in|ii|\ell_{g1}) \seq A(\ell_{g2}|b|out)
\\&\lor& A(in|ii|\ell_{g2}) \seq A(\ell_{g1}|a|out)
\\&\lor& A(in|ii \seq b|out)
\\&\lor& A(\ell_{g1}|a|out) \seq A(in|ii|\ell_{g1})
\\&\lor& A(\ell_{g1}|a|out) \seq A(in|ii|\ell_{g2})
\\&\lor& A(\ell_{g1}|a|out) \seq A(\ell_{g2}|b|out)
\\&\lor& A(\ell_{g2}|b|out) \seq A(in|ii|\ell_{g1})
\\&\lor& A(\ell_{g2}|b|out) \seq A(in|ii|\ell_{g2})
\\&\lor& A(\ell_{g2}|b|out) \seq A(\ell_{g1}|a|out)
\EQ{apply $A^2$ mutually-exclusive law w.r.t. $I$}
\\&    & A(in|ii\seq a|out)
\\&\lor& A(in|ii\seq b|out)
\EQ{$ii$ is unit for $\seq$ over $s$, $s'$}
\\&& A(in|a|out) \lor A(in|b|out)
}

\RLEQNS{
  && Q^3
\EQ{split as $Q^2 \seq Q$}
\\&& (A(in|a|out) \lor A(in|b|out)) \seq {}
\\&& (~ A(in|ii|\ell_{g1}) \lor
     A(in|ii|\ell_{g2}) \lor
     A(\ell_{g1}|a|out) \lor
     A(\ell_{g2}|b|out)~)
\EQ{distribute}
\\&    & A(in|a|out) \seq A(in|ii|\ell_{g1}) \quad \mbox{--- $A^2$ cond fail}
\\&\lor& A(in|a|out) \seq A(in|ii|\ell_{g2}) \quad \mbox{--- $A^2$ cond fail}
\\&\lor& A(in|a|out) \seq A(\ell_{g1}|a|out) \quad \mbox{--- mut-exc fail}
\\&\lor& A(in|a|out) \seq A(\ell_{g2}|b|out) \quad \mbox{--- mut-exc fail}
\\&\lor& A(in|b|out) \seq A(in|ii|\ell_{g1}) \quad \mbox{--- $A^2$ cond fail}
\\&\lor& A(in|b|out) \seq A(in|ii|\ell_{g2}) \quad \mbox{--- $A^2$ cond fail}
\\&\lor& A(in|b|out) \seq A(\ell_{g1}|a|out) \quad \mbox{--- mut-exc fail}
\\&\lor& A(in|b|out) \seq A(\ell_{g2}|b|out) \quad \mbox{--- mut-exc fail}
\EQ{all gone!}
\\&& \false
}
So, we go as far as $Q^2$:

\RLEQNS{
  && atm(a)+atm(b)
\EQ{$Q$ expansion}
\\&& \Skip \lor Q \lor Q^2
\EQ{expand $Q^i$}
\\&& \Skip \lor A(in|ii|\ell_{g1}) \lor A(in|ii|\ell_{g2}) \lor {}
\\&& A(\ell_{g1}|a|out) \lor A(\ell_{g2}|b|out) \lor{}
\\&& A(in|a|out) \lor A(in|b|out)
}

\newpage
\HDRc{Parallel Composition}


\RLEQNS{
   C \parallel D
   &\defs&
   \W(~
      A(in|ii|\ell_{g1},\ell_{g2}) \lor
      A(\ell_{g1:},\ell_{g2:}|ii|out) \lor {}
\\&& ~~
       C[g_{1::},\ell_{g1},\ell_{g1:}/g,in,out] \lor
       D[g_{2::},\ell_{g2},\ell_{g2:}/g,in,out] ~)
\\&& {} \land [in|\ell_{g1},\ell_{g2}|\ell_{g1:},\ell_{g2:}|out]
}

\RLEQNS{
   P &=& A(in|ii|\ell_{g1},\ell_{g2}) \lor
         A(\ell_{g1:},\ell_{g2:}|ii|out) \lor {}
\\   & & atm(a)[g_{1::},\ell_{g1},\ell_{g1:}/g,in,out] \lor
         atm(b)[g_{2::},\ell_{g2},\ell_{g2:}/g,in,out]
\\
\\ I &=& [in\mid[\ell_{g1}|\ell_{g1:}],[\ell_{g2}|\ell_{g2:}]\mid out]
\\ && \mbox{Note the new complicated invariant!}
}

\RLEQNS{
  && P
\EQ{expand $P$}
\\&& A(in|ii|\ell_{g1},\ell_{g2}) \lor
     A(\ell_{g1:},\ell_{g2:}|ii|out) \lor {}
\\&& atm(a)[g_{1::},\ell_{g1},\ell_{g1:}/g,in,out] \lor
     atm(b)[g_{2::},\ell_{g2},\ell_{g2:}/g,in,out]
\EQ{expand $atm$}
\\&& A(in|ii|\ell_{g1},\ell_{g2}) \lor
     A(\ell_{g1:},\ell_{g2:}|ii|out) \lor {}
\\&& ((\Skip \lor A(in|a|out)) \land [in|out])
     [g_{1::},\ell_{g1},\ell_{g1:}/g,in,out] \lor {}
\\&& ((\Skip \lor A(in|b|out)) \land [in|out])
     [g_{2::},\ell_{g2},\ell_{g2:}/g,in,out]
\EQ{substitution}
\\&& A(in|ii|\ell_{g1},\ell_{g2}) \lor
     A(\ell_{g1:},\ell_{g2:}|ii|out) \lor {}
\\&& (\Skip \lor A(\ell_{g1}|a|\ell_{g1:})) \land [\ell_{g1}|\ell_{g1:}] \lor {}
\\&& (\Skip \lor A(\ell_{g2}|b|\ell_{g2:})) \land [\ell_{g2}|\ell_{g2:}]
\EQ{atomic invariants subsumed by $I$}
\\&& A(in|ii|\ell_{g1},\ell_{g2}) \lor
     A(\ell_{g1:},\ell_{g2:}|ii|out) \lor {}
\\&& \Skip \lor A(\ell_{g1}|a|\ell_{g1:}) \lor \Skip \lor A(\ell_{g2}|b|\ell_{g2:})
\EQ{re-arrange, simplify}
\\&& \Skip \lor
     A(in|ii|\ell_{g1},\ell_{g2}) \lor
     A(\ell_{g1:},\ell_{g2:}|ii|out) \lor
     A(\ell_{g1}|a|\ell_{g1:}) \lor
     A(\ell_{g2}|b|\ell_{g2:})
\EQ{$Q$ again}
\\&& \Skip \lor Q
}

\RLEQNS{
  && Q^2
\EQ{expand $Q$}
\\&& (~ A(in|ii|\ell_{g1},\ell_{g2}) \lor
        A(\ell_{g1:},\ell_{g2:}|ii|out) \lor {}
\\&& ~~ A(\ell_{g1}|a|\ell_{g1:}) \lor
        A(\ell_{g2}|b|\ell_{g2:}) ~) \seq {}
\\&& (~ A(in|ii|\ell_{g1},\ell_{g2}) \lor
        A(\ell_{g1:},\ell_{g2:}|ii|out) \lor {}
\\&& ~~ A(\ell_{g1}|a|\ell_{g1:}) \lor
        A(\ell_{g2}|b|\ell_{g2:}) ~)
\EQ{distr., assess w.r.t $A^2$, $[in|\ell_{g1},\ell_{g2}|\ell_{g1:},\ell_{g2:}|out]$}
\\&\lor& A(in|ii|\ell_{g1},\ell_{g2})
    \seq A(in|ii|\ell_{g1},\ell_{g2})    & fail
\\&\lor& A(in|ii|\ell_{g1},\ell_{g2})
    \seq A(\ell_{g1:},\ell_{g2:}|ii|out) & fail
\\&\lor& A(in|ii|\ell_{g1},\ell_{g2})
    \seq A(\ell_{g1}|a|\ell_{g1:})       & ok
\\&\lor& A(in|ii|\ell_{g1},\ell_{g2})
    \seq A(\ell_{g2}|b|\ell_{g2:})       & ok
\\&\lor& A(\ell_{g1:},\ell_{g2:}|ii|out)
    \seq A(in|ii|\ell_{g1},\ell_{g2})    & fail
\\&\lor& A(\ell_{g1:},\ell_{g2:}|ii|out)
    \seq A(\ell_{g1:},\ell_{g2:}|ii|out) & fail
\\&\lor& A(\ell_{g1:},\ell_{g2:}|ii|out)
    \seq A(\ell_{g1}|a|\ell_{g1:})       & fail
\\&\lor& A(\ell_{g1:},\ell_{g2:}|ii|out)
    \seq A(\ell_{g2}|b|\ell_{g2:})       & fail
\\&\lor& A(\ell_{g1}|a|\ell_{g1:})
    \seq A(in|ii|\ell_{g1},\ell_{g2})    & fail
\\&\lor& A(\ell_{g1}|a|\ell_{g1:})
    \seq A(\ell_{g1:},\ell_{g2:}|ii|out) & ok
\\&\lor& A(\ell_{g1}|a|\ell_{g1:})
    \seq A(\ell_{g1}|a|\ell_{g1:})       & fail
\\&\lor& A(\ell_{g1}|a|\ell_{g1:})
    \seq A(\ell_{g2}|b|\ell_{g2:})       & ok
\\&\lor& A(\ell_{g2}|b|\ell_{g2:})
    \seq A(in|ii|\ell_{g1},\ell_{g2})    & fail
\\&\lor& A(\ell_{g2}|b|\ell_{g2:})
    \seq A(\ell_{g1:},\ell_{g2:}|ii|out) & ok
\\&\lor& A(\ell_{g2}|b|\ell_{g2:})
    \seq A(\ell_{g1}|a|\ell_{g1:})       & ok
\\&\lor& A(\ell_{g2}|b|\ell_{g2:})
    \seq A(\ell_{g2}|b|\ell_{g2:})       & fail
\EQ{drop fails}
\\&\lor& A(in|ii|\ell_{g1},\ell_{g2})
    \seq A(\ell_{g1}|a|\ell_{g1:})       & ok
\\&\lor& A(in|ii|\ell_{g1},\ell_{g2})
    \seq A(\ell_{g2}|b|\ell_{g2:})       & ok
\\&\lor& A(\ell_{g1}|a|\ell_{g1:})
    \seq A(\ell_{g1:},\ell_{g2:}|ii|out) & ok
\\&\lor& A(\ell_{g1}|a|\ell_{g1:})
    \seq A(\ell_{g2}|b|\ell_{g2:})       & ok
\\&\lor& A(\ell_{g2}|b|\ell_{g2:})
    \seq A(\ell_{g1:},\ell_{g2:}|ii|out) & ok
\\&\lor& A(\ell_{g2}|b|\ell_{g2:})
    \seq A(\ell_{g1}|a|\ell_{g1:})       & ok
\EQ{law $A^2$}
\\&& \mbox{the calculator might be a good idea at this point!}
}
We will need to go to $Q^4$, and show that $Q^5$ is $\false$.

\newpage
\HDRc{Iteration}

A quick pen'n'paper calculation for $atm(a)^*$ yields:
\[\begin{array}{rcccc}
   Q   &=& in\arr{} out & in\arr{}\ell_g & in\arr{a} in
\\ Q^2 &=& in \arr{a} in & \ell_g \arr{a} out & \ell_g \arr{a} \ell_g
\\ Q^3   &=& in\arr{a} out & in\arr{a}\ell_g & in\arr{aa} in
\\ Q^4 &=& in \arr{aa} in & \ell_g \arr{aa} out & \ell_g \arr{aa} \ell_g
\\ Q^5   &=& in\arr{aa} out & in\arr{aa}\ell_g & in\arr{aaa} in
\\ Q^6 &=& in \arr{aaa} in & \ell_g \arr{aaa} out & \ell_g \arr{aaa} \ell_g
\end{array}\]


\newpage
\HDRb{Semantic Results}

\RLEQNS{
   atm(a)
   &=&
   (~\Skip \lor A(in|a|out)~) \land [in|out]
\\ atm(a)\seq atm(b)
   &=& (~\Skip \lor
         A(in|a|\ell_g) \lor
         A(\ell_g|b|out) \lor
         A(in|a \seq b| out) ~)
\\&& {} \land [in|\ell_g|out]
\\ atm(a)+atm(b)
   &=& (~\Skip \lor A(in|ii|\ell_{g1}) \lor A(in|ii|\ell_{g2}) \lor {}
\\&& A(\ell_{g1}|a|out) \lor A(\ell_{g2}|b|out) \lor{}
\\&& A(in|a|out) \lor A(in|b|out)~)
\\&& {} \land [in|\ell_{g1}|\ell_{g2}|out]
}


Better formatting(?): 1st line is invariant,
other lines are the disjuncts (poss. several on one line), with $\Skip$ ommitted

\RLEQNS{
   atm(a)
   &=& [in|out]
\\ & & A(in|a|out)
\\
\\ atm(a)\seq atm(b)
   &=& [in|\ell_g|out]
\\ & & A(in|a|\ell_g) \quad A(\ell_g|b|out)
\\ & & A(in|a \seq b|out)
\\
\\ atm(a)+atm(b)
   &=& [in|\ell_{g1}|\ell_{g2}|out]
\\ & & A(in|ii|\ell_{g1}) \quad A(in|ii|\ell_{g2})
\\ & & A(\ell_{g1}|a|out) \quad A(\ell_{g2}|b|out)
\\ & & A(in|a|out) \quad A(in|b|out)
}

