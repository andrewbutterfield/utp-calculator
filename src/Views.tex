\HDRa{Math support for Views}\label{ha:mViews}

\HDRb{The Big Plan}

We go for a new formulation
\RLEQNS{
   X(E|a|R|A) &\defs&
   ls(E) \land s' \in \sem a s \land ls' = (ls\setminus R)\cup A
\\ A(E|a|N) &\defs& X(E|a|E|N)
\\ \W(P) &\defs& \true * (\Skip \lor P)
\\       &=& \bigvee_{i \in \Nat} \Skip \seq P^i
\\ atm(a) &=& \W(A(in|a|out) \land [in|out])
\\ C &=& \W(actions(C) \land [in|out|g] \land I_C)
}
where $I_C$ is some more $C$-specific invariants.

\HDRb{Set Inclusion/Membership}

An atomic action $A(E|a|N)$ is enabled if $E$ is contained
in the global label-set ($ls(E)$)
and results in $E$ being removed from that set, and new labels
$N$ being added ($ls'=(ls\setminus E)\cup N$).
We need a way to reason about containment is such an $ls'$
in terns of $E$ and $N$, and to compute sequential compositions
of such forms, which will take the more general form $X(E|a|R|A)$.

We find we get assertions of the form $(F(ls))(E)$,
asserting that $E$ is a subset of $F(ls)$ where $F$ is a set-function
composed of named/enumerated sets and standard set-operations.
We want to transform it into $ls(G) \land P$ where $G$ and $P$
do not involve $ls$.

We present the laws,
then the proofs
\RLEQNS{
   (ls \cup A)(S)      &=& ls(S\setminus A)
\\ (ls \setminus R)(S) &=& ls(S) \land S \cap R = \emptyset
\\ (ls \cap M)(S)      &=& ls(S) \land S \subseteq M
\\ ((ls \setminus R) \cup A)(S)
   &=& ls(S \setminus A) \land (S \setminus A) \cap R = \emptyset
\\ (((ls\setminus R_1) \cup A_1)\setminus R_2) \cup A_2
  &=& (ls \setminus (R_1 \cup R_2)) \cup ((A_1\setminus R_2) \cup A_2)
}

We do the proofs in ``classical'' set notation
\RLEQNS{
  && S \subseteq (ls \cup A)
\EQ{set definitions}
\\&& x \in S \implies x \in (ls \cup A)
\EQ{defn $\cup$}
\\&& x \in S \implies x \in ls \lor x \in A
\EQ{defn $\implies$}
\\&& x \notin S \lor x \in ls \lor x \in A
\EQ{rearrange}
\\&& x \notin S \lor x \in A \lor x \in ls
\EQ{De-Morgan, defn $\implies$}
\\&& (x \in S \land x \notin A) \implies x \in ls
\EQ{defn subset}
\\&& (S \setminus A) \subseteq ls
}

\RLEQNS{
  && S \subseteq (ls \setminus R)
\EQ{set definitions}
\\&& x \in S \implies x \in (ls \setminus R)
\EQ{set definition}
\\&& x \in S \implies x \in ls \land x \notin R
\EQ{defn $\implies$}
\\&& x \notin S \lor x \in ls \land x \notin R
\EQ{distribution}
\\&& (x \notin S \lor x \in ls)
     \land
     ( x \notin S \lor x \notin R)
\EQ{defn implies, de-morgan}
\\&& (x \in S \implies x \in ls)
     \land
     \lnot( x \in S \land x \in R)
\EQ{defn subset}
\\&&S \subseteq ls \land S \cap R = \emptyset
}

\RLEQNS{
  && S \subseteq (ls \cap M)
\EQ{set definitions}
\\&& x \in S \implies x \in ls \land x \in M
\EQ{defn $\implies$}
\\&& x \notin S \lor x \in ls \land x \in M
\EQ{distribution}
\\&& (x \notin S \lor x \in ls)
     \land
     (x \notin S \lor x \in M)
\EQ{defn $implies$}
\\&& (x \in S \implies x \in ls)
     \land
     (x \in S \implies x \in M)
\EQ{def subset}
\\&& S \subseteq ls \land S \subseteq M
}

\RLEQNS{
  && ((ls \setminus R) \cup A)(S)
\EQ{laws above}
\\&& (ls \setminus R)(S \setminus A)
\EQ{laws above}
\\&& ls(S \setminus A) \land (S \setminus A) \cap R = \emptyset
}

\RLEQNS{
  && x \in (ls\setminus R_1) \cup A_1
\EQ{defn $\cup$}
\\&& x \in (ls\setminus R_1) \lor x \in  A_1
\EQ{defn $\setminus$}
\\&& x \in ls \land x \notin R_1 \lor x \in  A_1
}

\RLEQNS{
  && x \in (((ls\setminus R_1) \cup A_1)\setminus R_2) \cup A_2
\EQ{above law}
\\&& x \in (ls\setminus R_1) \cup A_1) \land x \notin R_2 \lor x \in  A_2
\EQ{above law}
\\&& (x \in ls \land x \notin R_1 \lor x \in  A_1) \land x \notin R_2 \lor x \in  A_2
\EQ{distributivity}
\\&& (x \in ls \land x \notin R_1 \land x \notin R_2)
     \lor
     (x \in  A_1 \land x \notin R_2)
     \lor x \in  A_2
\EQ{de-Morgan, defn $\setminus$}
\\&& (x \in ls \land \lnot(x \in R_1 \lor x \in R_2))
     \lor
     x \in  A_1 \setminus R_2
     \lor x \in  A_2
\EQ{defn $\cup$, twice}
\\&& (x \in ls \land \lnot(x \in R_1 \cup R_2))
     \lor
     x \in  A_1 \setminus R_2 \cup  A_2
\EQ{tweak}
\\&& (x \in ls \land x \notin R_1 \cup R_2)
     \lor
     x \in  A_1 \setminus R_2 \cup  A_2
\EQ{definition of $\setminus$}
\\&& x \in (ls \setminus R_1 \cup R_2)
     \lor
     x \in  A_1 \setminus R_2 \cup  A_2
\EQ{definition of $cup$}
\\&& x \in (ls \setminus R_1 \cup R_2)
     \cup
     (A_1 \setminus R_2 \cup  A_2)
}

\HDRb{Basic Actions}

We first by defining a basic operation form that is closed
under sequential composition, modulo some `ground' side-conditions.
\RLEQNS{
   X(E|a|R|A)
   &\defs&
   ls(E) \land [a] \land ls' = (ls\setminus R) \cup A
}
Composing these requires us decouple the enabling labels
from those removed---they are the same for a basic action,
but differ as they are sequentially composed.

The general composition:
\RLEQNS{
  && X(E_1|a|R_1|A_1)\seq X(E_2|b|R_2|A_2)
\EQ{Defn $X$}
\\&& ls(E_1) \land [a] \land ls' = (ls\setminus R_1) \cup A_1
     \quad\seq\quad
     ls(E_2) \land [b] \land ls' = (ls\setminus R_2) \cup A_2
\EQ{Defn $\seq$}
\\&& \exists s_m,ls_m @
\\&& \quad ls(E_1) \land
           [a][s_m/s'] \land
           ls_m = (ls\setminus R_1) \cup A_1 \land {}
\\&& \quad ls_m(E_2) \land
           [b][s_m/s] \land
           ls' = (ls_m\setminus R_2) \cup A_2
\EQ{1-pt rule ($ls_m$), re-arrange}
\\&& \exists s_m @
\\&& \quad ls(E_1) \land
           ((ls\setminus R_1) \cup A_1)(E_2) \land {}
\\&& \quad [a][s_m/s'] \land
           [b][s_m/s] \land {}
\\&& \quad ls' = (((ls\setminus R_1) \cup A_1)\setminus R_2) \cup A_2
\EQ{shrink $s_m$ scope, use $ls(-)$ laws above}
\\&& ls(E_1) \land
     ls(E_2\setminus A_1) \land
     (E_2\setminus A_1) \cap R_1 = \emptyset \land {}
\\&& (\exists s_m @ [a][s_m/s'] \land [b][s_m/s]) \land {}
\\&& ls' = (((ls\setminus R_1) \cup A_1)\setminus R_2) \cup A_2
\EQ{defn $\seq$, prop of $ls(-)$, simplification from above}
\\&& ls(E_1 \cup (E_2\setminus A_1)) \land
     (E_2\setminus A_1) \cap R_1 = \emptyset \land {}
\\&& [a;b] \land {}
\\&& ls' = (ls \setminus R_1 \cup R_2)
           \cup
           (A_1 \setminus R_2 \cup  A_2)
\EQ{defn of $X$}
\\&& X(E_1 \cup (E_2\setminus A_1)
       |a\seq b
       |R_1 \cup R_2
       |A_1 \setminus R_2 \cup  A_2)
       \land (E_2\setminus A_1) \cap R_1 = \emptyset
}
Here the `ground' side-condition is
 $(E_2\setminus A_1) \cap R_1 = \emptyset$.

Composing a basic operation with itself:
\RLEQNS{
  && X(E|a|R|A) \seq X(E|a|R|A)
\EQ{by above}
\\&& X(E \cup (E\setminus A)
       |a\seq a
       |R \cup R
       |A \setminus R \cup  A)
       \land (E\setminus A) \cap R = \emptyset
\EQ{simpify}
\\&& X(E|a\seq a|R|A)
       \land (E\setminus A) \cap R = \emptyset
}

A basic action identifies $E$ and $R$:
\RLEQNS{
  && A(E|a|N) \defs X(E|a|E|N)
\\
\\&& A(E_1|a|N_1) \seq A(E_2|a|N_2)
\EQ{by defn}
\\&& A(E_1|a|E_1|N_1) \seq A(E_2|a|E_2|N_2)
\EQ{by calc above, with $R_i = E_i$}
\\&& X(E_1 \cup (E_2\setminus A_1)
       |a\seq b
       |E_1 \cup E_2
       |A_1 \setminus E_2 \cup  A_2)
       \land (E_2\setminus A_1) \cap E_1 = \emptyset
}
We cannot transform this into the form $A(-|a\seq b|-)$
unless it happens to be the case that
\\$E_1 \cup (E_2\setminus A_1)= E_1 \cup E_2$.
For an $A$ composed with itself:
\RLEQNS{
  && A(E|a|N) \seq A(E|a|N)
\EQ{defn $A$}
\\&& X(E|a|E|N) \seq X(E|a|E|N)
\EQ{calc above}
\\&& X(E|a\seq a|E|A)
       \land (E\setminus A) \cap E = \emptyset
}
We see that we get a $\false$ result unless we have $E \subseteq A$,
in which everything we remove is added back in, plus possibly some extra.
We get the following which is similar
\[
  X(E|a\seq a|\emptyset|A)
\]
This is a consequence of the following easy law:
\[
 (ls \setminus R) \cup A
 =
 (ls \setminus (R \setminus A)) \cup A
 =
 (ls \cup A) \setminus (R \setminus A)
\]

\HDRc{Laws of Actions}
Here we summarise the main results:
\RLEQNS{
  && X(E_1|a|R_1|A_1)\seq X(E_2|b|R_2|A_2)
\EQ{proof above}
\\&& X(E_1 \cup (E_2\setminus A_1)
       \mid a\seq b
       \mid R_1 \cup R_2
       \mid A_1 \setminus R_2 \cup  A_2)
\\&& {} \land (E_2\setminus A_1) \cap R_1 = \emptyset
\\
\\&& \\&& A(E_1|a|N_1) \seq A(E_2|a|N_2)
\EQ{proof above}
\\&& X(E_1 \cup (E_2\setminus A_1)
       \mid a\seq b
       \mid E_1 \cup E_2
       \mid A_1 \setminus E_2 \cup  A_2)
\\&& {} \land (E_2\setminus A_1) \cap E_1 = \emptyset
\\
\\&& A(E|a|N) \seq A(E|a|N)
\EQ{proof above}
\\&& X(E|a\seq a|E|A)
\\&& {} \land (E\setminus A) \cap E = \emptyset
}



\newpage
\HDRb{Working with the Invariant}

We have introduced the following notation:
\[
  [ L_1 | L_2 | \dots | L_n ]
\]
Its first intended meaning is to assert that
all the $L_i$ are mutually disjoint:
\RLEQNS{
   \forall i,j \in 1\dots n @ i \neq j \implies L_i \cap L_j = \emptyset
}
It also states that if any one of its members overlaps with $ls$,
then none of the others do (and similarly for $ls'$):
\RLEQNS{
  &&   \forall i,j \in 1\dots n @
        i \neq j \land L_i \cap ls \neq \emptyset
        \implies L_j \cap ls = \emptyset
\\&&   \forall i,j \in 1\dots n @
        i \neq j \land L_i \cap ls' \neq \emptyset
        \implies L_j \cap ls' = \emptyset
}

By \emph{design}, our labels and generator scheme
establishes the following (weakest) invariant:
\[ [in|out|g]\]
and for any generator expression $G$ we can split it into
four disjoint parts (also by design) to get
\[  [\ell_G|G_{:}|G_1|G_2] . \]
So we can always strengthen out invariant by splitting some $G$
in this way:
\RLEQNS{
  && [in|out|g]
\EQ{split $g$}
\\&& [in|out|\ell_g|\g:|\g1|\g2]
\EQ{split $\g:$}
\\&& [in|out|\ell_g| \ell_{g:}|\g{::}|\g{:1}|\g{:2}|\g1|\g2]
}
However, sometimes we don't wait to do this.
The hope is that by requiring $[in|out|g]$ at each level,
that we get the right level of exclusivity vs. sharing of $ls$
by generated labels.

\newpage
\HDRb{Semantic Definitions}

For ``wheels-within-wheels'' to work,
we need to ensure we spin up as long as any constituent
atomic action does not have its $out$ label in $ls$.
\RLEQNS{
  C &=& ls(\B{OUT}) * \bigvee_{i=1}^n in_i \arr{a_i} out_i
\\ && \textbf{where }OUT = \setof{out_1,\dots,out_n}
}
Now this looks a little non-compositional,
but perhaps we can get this outcome by only looking one level down\dots

We assume the generator invariant $[in|out|g]$
and all its strengthening hold throughout.



\RLEQNS{
   E \arr a N &\defs& ls(E) \land s'\in\sem a \land ls'=(ls\setminus E)\cup N
\\ atm(a) &\defs& ls(\B{out}) * in \arr a out
\\ C \cseq D
   &\defs&
   ls(\B{\ell_g,out}) *
     ( C[\ell_g,\g{:1}/out,g] \lor
       D[\ell_g,\g{:2}/in,g] )
\\ C + D
   &\defs&
   ls(\B{out}) *
     ( C[g_{1:},\ell_{g1}/g,in] \lor
       D[g_{2:},\ell_{g2}/g,in] )
\\ C \parallel D
   &\defs&
   ls(\B{\ell_{g1},\ell_{g2},\ell_{g1:},\ell_{g2:},out}) * {}
\\&& \quad (~
      (in \arr{ii} \ell_{g1},\ell{g2}) \lor
      (\ell_{g1:},\ell{g2:} \arr{ii} out) \lor {}
\\&& \quad ~~
       C[g_{1::},\ell_{g1},\ell_{g1:}/g,in,out] \lor
       D[g_{2::},\ell_{g2},\ell_{g2:}/g,in,out] ~)
\\ C^*
   &\defs&
   ls(\B{out}) *
     (~ (in \arr{ii} out) \lor
       (in \arr{ii} \ell_g) \lor
       C[\g:,\ell_g,in/g,in,out] ~)
}


\newpage
\HDRb{Semantic Calculations}

We need to only investigate the combinators
applied to atoms, but also some nesting to see if everything
does keep going.

The basic principle relies on the following properties of
standard UTP iteration:
\RLEQNS{
  && c*P
\EQ{loop unroll}
\\&& P \seq c*P \cond c \Skip
\EQ{defn. of $\cond{}$}
\\&& \lnot c \land \Skip
     \lor
     c \land (P \seq c*P)
\EQ{push $c$ into $\seq$}
\\&& \lnot c \land \Skip
     \lor
     c \land P \seq c*P
\EQ{repeat above steps several times with $\lor$-$\seq$-distr.}
\\&& \lnot c \land \Skip
     \lor
     c \land P \seq \lnot c \land \Skip
     \lor
     (c \land P)^2 \seq \lnot c \land \Skip
     \lor \dots \lor
     (c \land P)^n \seq c*P
\EQ{pushed to the limit}
\\&& \bigvee_{i=0}^\infty (c \land P)^i \seq \lnot c \land \Skip
}

We introduce further shorthand:
\RLEQNS{
   D &\defs& \lnot c \land \Skip
\\ S &\defs& c \land P
\\ c*P &=& \bigvee_{i=0}^\infty S^i \seq D
}
So the nature of $S^i\seq D$ is the critical thing to compute.


SOME OLD STUFF THAT MAY NEED REWRITING:

Now, for our basic primitives some easy calculations show:
\RLEQNS{
   D(T) &\implies& ls(T) \land ls'(T)
\\ E \arr a N &\implies& ls(E) \land ls'(N)
}
If we keep in mind that,
\RLEQNS{
  P \land c' \seq Q &=& P \seq c \land Q
}
then we can very easily read-off whether or not sequential compositions
of these two violate the invariant, e.g, given $[in|out|\ell_g]$
\RLEQNS{
   D(in) \seq D(\ell_g) &=& \false
\\ (in \arr a \ell_g) \seq (\ell_g \arr b out)
    &=& (in \arr{a \seq b} out)
\\ (in \arr a \ell_g) \seq (in \arr a \ell_g) &=& \false
\\ (in \arr a \ell_g) \seq D(out) &=& \false
\\ (in \arr a \ell_g) \seq D(\ell_g) &=& (in \arr a \ell_g)
}
