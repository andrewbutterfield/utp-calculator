\HDRa{Math support for Views}\label{ha:mViews}

We shall start with a quick look at the semantics:

\HDRb{Definitions}

\RLEQNS{
   \I o i g &\defs& [i|o|G]
\\ \DLP i o G (P) &\defs& P \land\I i o G
\\ \KSP i o G(P) &\defs& ls(\B{o}) * P
\\ \W(P) &\defs& \DL \circ \KS
\\ &=& \I i o G \land (~ls(\B o) * P~)
\\ E \arr a N
   &\defs&
   ls(E) \land s'\in\sem a s \land ls'=(ls\setminus E)\cup N
\\ atm(a) &\defs& \W(in \arr a out)
\\ P \cseq Q
   &\defs&
   \W( P[\ell_g,\g{:1}/out,g]
       \lor Q[\ell_g,\g{:2}/in,g])
\\ P + Q
   &\defs&
   \W(P[g_{1:},\ell_{g1}/g,in] \lor Q[g_{2:},\ell_{g2}/g,in])
}

\RLEQNS{
   D(T) &\defs& ls(T) \land \Skip
\\ X(E|a|N) &\defs& ls(E) \land s'\in\sem a s \land ls'=(ls\setminus E)\cup N
}

\newpage
\HDRb{Working with the Invariant}

We have introduced the following notation:
\[
  [ L_1 | L_2 | \dots | L_n ]
\]
Its first intended meaning is to assert that
all the $L_i$ are mutually disjoint:
\RLEQNS{
   \forall i,j \in 1\dots n @ i \neq j \implies L_i \cap L_j = \emptyset
}
It also states that if any one of its members overlaps with $ls$,
then none of the others do (and similarly for $ls'$):
\RLEQNS{
  &&   \forall i,j \in 1\dots n @
        i \neq j \land L_i \cap ls \neq \emptyset
        \implies L_j \cap ls = \emptyset
\\&&   \forall i,j \in 1\dots n @
        i \neq j \land L_i \cap ls' \neq \emptyset
        \implies L_j \cap ls' = \emptyset
}
Now, for our basic primitives some easy calculations show:
\RLEQNS{
   D(T) &\implies& ls(T) \land ls'(T)
\\ E \arr a N &\implies& ls(E) \land ls'(N)
}
If we keep in mind that,
\RLEQNS{
  P \land c' \seq Q &=& P \seq c \land Q
}
then we can very easily read-off whether or not sequential compositions
of these two violate the invariant, e.g, given $[in|out|\ell_g]$
\RLEQNS{
   D(in) \seq D(\ell_g) &=& \false
\\ (in \arr a \ell_g) \seq (\ell_g \arr b out)
    &=& (in \arr{a \seq b} out)
\\ (in \arr a \ell_g) \seq (in \arr a \ell_g) &=& \false
\\ (in \arr a \ell_g) \seq D(out) &=& \false
\\ (in \arr a \ell_g) \seq D(\ell_g) &=& (in \arr a \ell_g)
}

By \emph{design}, our labels and generator scheme
establishes the following (weakest) invariant:
\[ [in|out|g]\]
and for any generator expression $G$ we can split it into
four disjoint parts (also by design) to get
\[  [\ell_G|G_{:}|G_1|G_2] . \]
So we can always strengthen out invariant by splitting some $G$
in this way:
\RLEQNS{
  && [in|out|g]
\EQ{split $g$}
\\&& [in|out|\ell_g|\g:|\g1|\g2]
\EQ{split $\g:$}
\\&& [in|out|\ell_g| \ell_{g:}|\g{::}|\g{:1}|\g{:2}|\g1|\g2]
}
\newpage
\HDRb{Semantic Calculations}

The key principle being used is the following:
\RLEQNS{
  && \W(S) & S \mbox{ is iteration `step'}
\EQ{expanding $\W$, ignoring invariant part}
\\&& ls(\B{out}) * S
\EQ{loop unroll $n$ times}
\\&& D(out) \lor S\seq D(out) \lor S^2\seq D(out) \lor
     \dots \lor S^n \seq ls(\B{out})*S
}
At this point we note that if we calculate  $S^i\seq D(out)$
for small $i$ we should notice terms disappearing or getting subsumed,
leading to a finite disjunction of the form:
\[
  \bigvee_{i \in 0\dots k} S^i \seq D(out)
\]

We summarise the results from below here,
with the invariant instance used in each at the end
of that result on a line by itself
\RLEQNS{
   atm(a) &=& D(out) \lor (in \arr a out)
\\ && [in|out|g]
\\ atm(a) \cseq atm(b)
   &=& (in \arr a \ell_g)  \lor
     D(\ell_g)  \lor
     (\ell_g \arr b out) \lor
     (in \arr{a \seq b}  out) \lor
     D(out)
\\ && [in|out|\ell_g|\ell_{g:}|\g{::}|\g{:1}|\g{:2}|\g1|\g2]
}


\newpage
\HDRc{Atomic Semantics}

Here $S = in \arr a out$.

A complete restart:
\RLEQNS{
  && S \seq D(out)
\EQ{expand $S$}
\\&& (in \arr a out) \seq D(out)
\EQ{$D(out)$ is true, and gets absorbed}
\\&& in \arr a out
\EQ{shorthand}
\\&& S
\\
\\&& S \seq S
\EQ{expand $S$}
\\&& (in \arr a out) \seq (in \arr a out)
\EQ{fails invariant check}
\\&& \false
\\
\\&& atm(a)
\EQ{defn., $atm$, shorthand}
\\&& \W(S)
\EQ{unroll loop}
\\&& D(out) \lor S\seq D(out) \lor S^2\seq D(out) \lor S^3\seq D(out) \lor \dots
\EQ{above calcs}
\\&& D(out) \lor S \lor \false \lor \false \lor \dots
\EQ{shorthand}
\\&& D(out) \lor (in \arr a out)
}

\RLEQNS{
  && atm(a)
\EQ{defn $atm$}
\\&&  \W(in \arr a out)
\EQ{expand $\W$}
\\&& [in|out|g] \land (ls(\B{out}) * (in \arr a out))
\EQ{loop unroll, twice}
\\&    & ls(out) \land \Skip & [in|out|g]
\\&\lor& ls(\B{out}) \land  (in \arr a out)
         \seq ls(out) \land \Skip
\\&\lor& (ls(\B{out}) \land  (in \arr a out)
         \seq ls(\B{out}) \land  (in \arr a out)
         \seq ls(\B{out}) * (in \arr a out)
\EQ{pull before-expr back through $\seq$, $\Skip$-r-unit}
\\&    & ls(out) \land \Skip & [in|out|g]
\\&\lor& ls(\B{out}) \land  (in \arr a out)
         \land ls'(out)
\\&\lor& (ls(\B{out}) \land  (in \arr a out)
         \seq ls(\B{out}) \land  (in \arr a out)
         \seq ls(\B{out}) * (in \arr a out)
\EQ{$(\_\arr a out) \seq ls(\B{out})$ is False}
\\&    & ls(out) \land \Skip & [in|out|g]
\\&\lor& ls(\B{out}) \land  (in \arr a out)
         \land ls'(out)
\EQ{$(\_\arr a out)$ implies $ls'(out)$}
\\&    & ls(out) \land \Skip & [in|out|g]
\\&\lor& ls(\B{out}) \land  (in \arr a out)
\EQ{$[in|out|g] \land ls(in)$ implies $ls(\B{out})$}
\\&    & ls(out) \land \Skip & [in|out|g]
\\&\lor& (in \arr a out)
\EQ{shorthand}
\\&& [in|out|g] \land ( D(out) \lor (in \arr a out) )
}

\newpage
\HDRc{Sequence Semantics}

Here
\[ S
   = D(\ell_g) \lor (in \arr a \ell_g)
     \lor
     D(out) \lor (\ell_g \arr b out)
\]

\RLEQNS{
  && S \seq D(out)
\EQ{expand $S$}
\\&& D(\ell_g) \lor (in \arr a \ell_g)
     \lor
     D(out) \lor (\ell_g \arr b out)
     \seq D(out)
\EQ{distribute}
\\&& D(\ell_g)           \seq D(out) & fail
\\&& (in \arr a \ell_g)  \seq D(out) & fail
\\&& D(out)              \seq D(out) & ok
\\&& (\ell_g \arr b out) \seq D(out) & ok
\EQ{simplify survivors}
\\&& D(out) \lor (\ell_g \arr b out)
}

\RLEQNS{
  && atm(a)[\ell_g,\g{:1}/out,g]
\EQ{atomic calculation}
\\&& ( [in|out|g] \land ( D(out) \lor (in \arr a out) ))
    [\ell_g,\g{:1}/out,g]
\EQ{substitution}
\\&& [in|\ell_g|\g{:1}] \land ( D(\ell_g) \lor (in \arr a \ell_g) )
\\&& \coz{and for the $atm(b)$ case below \dots}
\\&& [\ell_g|out|\g{:2}] \land ( D(out) \lor (\ell_g \arr b out) )
}

\RLEQNS{
   g - \ell_g &=& [\g:|\g1|\g2]
\\ g - \g{:1} &=& [\ell_g|\g1|\g2|\ell_{g:}|\g{::}|\g{:2}]
\\ g - \g{:2} &=& [\ell_g|\g1|\g2|\ell_{g:}|\g{::}|\g{:1}]
\\ ~[in|\ell_g|\g{:1}] \land [in|\ell_g|\g{:2}]
   &=& [in|\ell_g|\g{:1}|\g{:2}]
\\ ~[in|out|g] \land [in|\ell_g|\g{:1}]
   &=& [in|out|\ell_g|\g{:1}|\g1|\g2|\ell_{g:}|\g{::}|\g{:2}]
\\ ~[in|out|g] \land [in|\ell_g|\g{:2}]
   &=& [in|out|\ell_g|\g{:2}|\g1|\g2|\ell_{g:}|\g{::}|\g{:1}]
\\ ~[in|out|g] \land [in|\ell_g|\g{:1}] \land [in|\ell_g|\g{:2}]
   &=& [in|out|\ell_g|\ell_{g:}|\g{::}|\g{:1}|\g{:2}|\g1|\g2]
}
\RLEQNS{
  && atm(a) \cseq atm(b)
\EQ{Defn. $\cseq$}
\\&& \W( atm(a)[\ell_g,\g{:1}/out,g]
       \lor atm(b)[\ell_g,\g{:2}/in,g])
\EQ{Expand $\W$}
\\&& [in|out|g]
     \land (~ ls(\B{out})
       * atm(a)[\ell_g,\g{:1}/out,g]
           \lor
           atm(b)[\ell_g,\g{:2}/in,g] ~)
\EQ{Sequence $atm$ calc. above}
\\&& ls(\B{out}) * & [in|out|g]
\\&& [in|\ell_g|\g{:1}] \land ( D(\ell_g) \lor (in \arr a \ell_g) ) \lor {}
\\&& [\ell_g|out|\g{:2}] \land ( D(out) \lor (\ell_g \arr b out) )
\EQ{label-sets calc above}
\\&& ls(\B{out}) * & [in|out|g]
\\&& ( D(\ell_g) \lor (in \arr a \ell_g) )
     \land [in|out|\ell_g|\ell_{g:}|\g{::}|\g{:1}|\g{:2}|\g1|\g2]
\\&& \lor
    ( D(out) \lor (\ell_g \arr b out) )
     \land  [in|out|\ell_g|\ell_{g:}|\g{::}|\g{:1}|\g{:2}|\g1|\g2]
\EQ{shorthand}
\\&& ls(\B{out}) * S & [in|out|g]
\\&& \textbf{where}
\\&& S =
   ( D(\ell_g) \lor (in \arr a \ell_g)\lor  D(out) \lor (\ell_g \arr b out) )
\\&& \qquad {} \land  [in|out|\ell_g|\ell_{g:}|\g{::}|\g{:1}|\g{:2}|\g1|\g2]
\EQ{loop unroll twice}
\\&& D(out)
     \lor S \seq D(out)
     \lor S \seq S \seq ls(\B{out}) * S
\\&& {} \land  [in|out|\ell_g|\ell_{g:}|\g{::}|\g{:1}|\g{:2}|\g1|\g2]
}

We now assume the invariant
$[in|out|\ell_g|\ell_{g:}|\g{::}|\g{:1}|\g{:2}|\g1|\g2]$
and proceed to calculate out
$D(out) \seq S$
and $S\seq S$.

We note a useful property of $D$:
\RLEQNS{
  && D(out)
\EQ{expand $D$ (and $\Skip$)}
\\&& ls(out) \land s'=s \land ls'=ls
\EQ{use equality of $ls$ and $ls'$ to add implied conjunct}
\\&& ls(out) \land s'=s \land ls'=ls \land ls'(out)
\EQ{use equality of $ls$ and $ls'$ to remove implied conjunct}
\\&& s'=s \land ls'=ls \land ls'(out)
\EQ{Defn of $\Skip$}
\\&& \Skip \land ls'(out)
}

We then note another:
\RLEQNS{
  && D(N) \seq P
\EQ{expand $D$, using property above}
\\&& \Skip \land ls'(N) \seq P
\EQ{push after-expr through $\seq$}
\\&& \Skip \seq ls(N) \land P
\EQ{$\seq$-l-unit}
\\&& ls(N) \land P
}
Note that an easy consequence of this is that $D(L)\seq D(L) = D(L)$,
and it is easy to show that $P \seq D(N) = P \land ls'(N)$.

\RLEQNS{
  && D(out) \seq S
\EQ{using property above}
\\&& ls(out) \land S
\EQ{expand $S$ and distribute}
\\&& ls(out) \land D(\ell_g) \lor
     ls(out) \land (in \arr a \ell_g) \lor {}
\\&& ls(out) \land D(out) \lor
     ls(out) \land (\ell_g \arr b out)
\EQ{apply invariant check}
\\&& \false \lor
     \false \lor {}
\\&& ls(out) \land D(out) \lor
     \false
\EQ{simplify, noting $D(out) = ls(out)\land\Skip$}
\\&& D(out)}

Composing very composable actions, ignoring invariants
\RLEQNS{
  && E \arr a A \seq A \arr b B
\EQ{Defn. action}
\\&& ls(E) \land [a] \land ls'=(ls\setminus E)\cup A \seq {}
\\&& ls(A) \land [b] \land ls'=(ls\setminus A)\cup B
\EQ{Isn't it obvious?}
\\&& ls(E) \land [a\seq b] \land ls'=(ls\setminus E)\cup B
\EQ{Defn. action}
\\&& E \arr{a\seq b} B
}

\RLEQNS{
  && S \seq S
\EQ{Expand both $S$}
\\&& \quad D(\ell_g) \lor (in \arr a \ell_g)\lor  D(out) \lor (\ell_g \arr b out)
\\&& \,{} \seq
     D(\ell_g) \lor (in \arr a \ell_g)\lor  D(out) \lor (\ell_g \arr b out)
\EQ{distribute (step 1)}
\\&& D(\ell_g) \seq
     (~D(\ell_g) \lor (in \arr a \ell_g)\lor  D(out) \lor (\ell_g \arr b out)~)
\\&& (in \arr a \ell_g) \seq
     (~D(\ell_g) \lor (in \arr a \ell_g)\lor  D(out) \lor (\ell_g \arr b out)~)
\\&& D(out) \seq
     (~D(\ell_g) \lor (in \arr a \ell_g)\lor  D(out) \lor (\ell_g \arr b out)~)
\\&& (\ell_g \arr b out) \seq
     (~D(\ell_g) \lor (in \arr a \ell_g)\lor  D(out) \lor (\ell_g \arr b out)~)
\EQ{delete those ruled out by invariant.}
\\&& D(\ell_g) \seq
     (~D(\ell_g) \lor (\ell_g \arr b out)~)
\\&& (in \arr a \ell_g) \seq
     (~D(\ell_g) \lor (\ell_g \arr b out)~)
\\&& D(out) \seq
     D(out)
\\&& (\ell_g \arr b out) \seq
     D(out)
\EQ{distribute (step 2), and simplify $D(L)^2$) and $D\seq P$ or $P\seq D$}
\\&& D(\ell_g)  \lor
     (ls(\ell_g) \land (\ell_g \arr b out)) \lor {}
\\&& ((in \arr a \ell_g) \land ls'(\ell_g)) \lor
     ((in \arr a \ell_g) \seq (\ell_g \arr b out)) \lor {}
\\&& D(out)  \lor
     ((\ell_g \arr b out)\land ls'(out))
\EQ{obvious(?) subsumptions.}
\\&& D(\ell_g)  \lor
     (\ell_g \arr b out) \lor
     (in \arr a \ell_g) \lor
     D(out)  \lor {}
\\&& ((in \arr a \ell_g) \seq (\ell_g \arr b out))
\EQ{do very composable composition, re-arrange}
\\&& (in \arr a \ell_g)  \lor
     D(\ell_g)  \lor
     (\ell_g \arr b out) \lor
     (in \arr{a \seq b}  out) \lor
     D(out)
}

Hypothesis: $S^n = S^2$, for $n \geq 2$.

If we look at both $S\seq S^2$ and $S^2\seq S$ we see that the
only potential new terms result from combining
the `extra' term $(in \arr{a \seq b}  out)$
in $S^2$ with the four terms of $S$

\RLEQNS{
   D(\ell_g)  \seq (in \arr{a \seq b}  out) && \false
\\ (in \arr{a \seq b}  out) \seq D(\ell_g) && \false
\\ (in \arr a \ell_g) \seq (in \arr{a \seq b}  out) && \false
\\ (in \arr{a \seq b}  out) \seq (in \arr a \ell_g) && \false
\\ D(out) \seq (in \arr{a \seq b}  out) && \false
\\ (in \arr{a \seq b}  out) \seq D(out) && (in \arr{a \seq b}  out)
\\ (\ell_g \arr b out) \seq (in \arr{a \seq b}  out) && \false
\\ (in \arr{a \seq b}  out) \seq (\ell_g \arr b out) && \false
}
So every term except the `extra` one disappears.
So we have shown that $S\seq S^2 = S^2 = S^2\seq S$.

We need to do $S^2 \seq D(out)$:
\RLEQNS{
  && S^2 \seq D(out)
\EQ{expand}
\\&& (in \arr a \ell_g)  \lor
     D(\ell_g)  \lor
     (\ell_g \arr b out) \lor
     (in \arr{a \seq b}  out) \lor
     D(out)
     \seq D(out)
\EQ{distribute}
\\&& (in \arr a \ell_g)      \seq D(out) & fail
\\&& D(\ell_g)               \seq D(out) & fail
\\&& (\ell_g \arr b out)     \seq D(out) & ok
\\&& (in \arr{a \seq b} out) \seq D(out) & ok
\\&& D(out)                  \seq D(out) & ok
\EQ{simplify survivors}
\\&& (\ell_g \arr b out) \lor
     (in \arr{a \seq b} out) \lor
     D(out)
}

So the expansion of $atm(a) \cseq atm(b)$ becomes
\RLEQNS{
  && atm(a) \cseq atm(b)
\EQ{loop unroll, given $S$}
\\&& D(out) \lor S\seq D(out) \lor S^2\seq D(out) \lor S^2\seq D(out) \dots
\EQ{simplify, expand $S^i$}
\\&& D(out) \lor D(out) \lor (\ell_g \arr b out)
       \lor S^2\seq D(out) \lor {}
\\&& (\ell_g \arr b out) \lor
     (in \arr{a \seq b} out) \lor
      D(out)
}

\newpage
\HDRc{Choice Semantics}

\RLEQNS{
  && atm(a)[g_{1:},\ell_{g1}/g,in]
\EQ{defn $atm$}
\\&& (D(out) \lor (in \arr a out))[g_{1:},\ell_{g1}/g,in]
\EQ{substitution}
\\&& (D(out) \lor (\ell_{g1} \arr a out))
\\
\\&& atm(b)[g_{2:},\ell_{g2}/g,in]
\EQ{defn $atm$}
\\&& (D(out) \lor (in \arr b out))[g_{2:},\ell_{g2}/g,in]
\EQ{substitution}
\\&& (D(out) \lor (\ell_{g2} \arr b out))
\\
\\&& atm(a)[g_{1:},\ell_{g1}/g,in] + atm(b)[g_{2:},\ell_{g2}/g,in]
\EQ{merge above two}
\\&& D(out) \lor (\ell_{g1} \arr a out) \lor (\ell_{g2} \arr b out)
\EQ{introduce shorthand}
\\&& S
}

\RLEQNS{
  && atm(a) + atm(b)
\EQ{Defn. $+$}
\\&& \W(atm(a)[g_{1:},\ell_{g1}/g,in] + atm(b)[g_{2:},\ell_{g2}/g,in])
}
