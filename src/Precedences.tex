\HDRb{Precedences Introduction}\label{hb:pred}

We see that most UTP theories can be envisaged as having three
layers:
\begin{enumerate}
  \item a base layer of standard logic (see \S\ref{hb:standard-preds});
  \item a middle layer of common programming/specification constructs
     (see \S\ref{hb:standard-UTP-preds});
  \item the top layer being the theory-specific language constructs.
\end{enumerate}

These are the current precedence levels for standard and UTP predicates,
determined by the following \emph{choices}%
\footnote{These are not laws,
just conventions we feel are most useful
for the kinds of things we usually write.}%
\RLEQNS{
   P \refinedby Q \implies R &=& P \refinedby (Q \implies R) & \refinedby_1
\\ P \equiv Q \implies R  &=& P \equiv (Q \implies R) & \equiv_1
\\ P \implies Q \cond R S &=& P \implies (Q \cond R S) & \implies_2
\\ P \cond Q R \seq S     &=& P \cond Q (R \seq D)     & \cond\!_3
\\ P \seq Q \lor R        &=& P \seq (Q \lor R)        & \seq_4
\\ P \lor Q \land R       &=& P \lor (Q \land R)       & \lor_5, \ndc_5
\\ P \land c * Q          &=& P \land (c * Q)          & \land_6
\\ \lnot c * P            &=& (\lnot c) * P            & *_7
\\ \lnot e = f            &=& \lnot(e=f)               & \lnot_8
\\ e = f[e/x]             &=& e = (f[e/x])             & =_9, [/]_{10}
}
