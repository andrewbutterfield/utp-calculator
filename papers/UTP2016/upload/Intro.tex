\section{Introduction}\label{sec:Intro}

\subsection{Motivation}

The development of a Unifying Theory of Programming (UTP)
can involve a number of false starts,
as alphabet variables are chosen
and semantics and healthiness conditions are defined.
Typically, some calculations done to check that everything
works end up revealing problems with the theory.
So an iteration occurs by revising the basic definitions,
and attempting the calculations again.

We have recently started to explore using UTP
to describe shared-variable concurrency,
by adapting the work of the  UTP semantics for Unifying Theories
of Parallel Programming (UTPP) \cite{DBLP:conf/icfem/WoodcockH02}.
We have reworked it to use a system for generating unique labels,
in order to give a slight improvement to the compositionality
of the semantics.
This we call a Unifying Theory of Concurrent Programming (UTCP)\cite{conf/tase/BMN16}.

We illustrate the calculator here
with, as a running example,
the definition of the UTP semantics of an atomic action.
We assume basic atomic actions $A,B$ which modify the
shared global state (program variables),
represented by observation variables $s,s'$.
The concurrent flow of control is managed by using labels
associated with language constructs, which are added to and removed
from a global label-set as execution proceeds.
We represent this label-set using observations $ls,ls'$.
Our main change to the original UTPP theory
is to provide a mechanism to create unique labels,
to be associated with both the beginning and end of each language
construct. This results in three static observables:
a generator $g$;
and two labels $in$ and $out$.
So our UTCP theory is based on a non-homogeneous relation
with alphabet $s$, $s'$, $ls$, $ls'$, $g$, $in$, and $out$.
See \cite{conf/tase/BMN16} for details.

Our running example is the need to calculate the outcome
of sequentially composing ($\pseq$)
two basic atomic actions ($A,B$),
that are lifted $(\A(\_))$ to the full alphabet by adding control-flow,
and are then $run$ in order to see their dynamic behaviour:
\[
  run(\A(A) \pseq \A(B))
\]
We hope that the final result would be
\[
  (A \seq B) \land ls' = \setof{\ell_{g:}}
\]
We have the standard UTP sequential composition of $A$ and $B$ defined on $s,s'$,
and an assertion that the termination label $\ell_{g:}$
is the sole member of the final label-set.

The theory with its somewhat unusual arrangement of observation variables
did \emph{not} emerge as an immediate and obvious solution,
but as a result of many trial calculations.
These trial calculations exposed two problems:
one was the difficulty in reading very long complex
set-based expressions in order to assess their correctness.
The second was the sheer length and drudgery of these calculations,
often involving many repetitions of very similar steps.

To be specific, the calculator described in this paper
is intended to be used for \emph{calculation}, and not \emph{theorem proving}.
In particular, it is designed to help solve the problem
just described above.
Both the starting predicate and the final result have free variables
and are not theorems.
That means counter-example generators
like Nitpick or Alloy
are not helpful.


If we consider the reasoning processes used in the development
and deployment of a theory, we can see a spectrum ranging from informal,
through to fully mechanised: hand calculation; simulation; proof assistant;
and automated theorem provers.
The level of detail, complexity,
and rigour rises as we proceed along the spectrum.
The calculator described here is designed
to assist with the exploratory hand-calculation phase early on,
by making it easier to calculate, and to manually check the outcomes.
It is not intended to provide the soundness guarantees that are quite rightly
expected from the tools further along the spectrum.


\subsection{Structure of this paper}

In Sect. \ref{sec:Related} we discuss related work
to justify our decision to develop the calculator.
The key design decisions and tool architecture is then described
in Sect. \ref{sec:Design}.
Three key components of the system are then discussed:
Dictionaries (Sect. \ref{sec:Dictionaries});
Expressions (Sect. \ref{sec:Expressions}); and
Predicates (Sect. \ref{sec:Predicates}).
In Sect. \ref{sec:Laws} we describe how to encode laws
and then conclude (Sect. \ref{sec:Conc}).
