\section{Expressions}\label{sec:Expressions}

In Fig. \ref{fig:expr-types} we show the Haskell declarations
of the datatypes used to represent expressions and substitution.
\begin{figure}[tb]
\begin{verbatim}
data Expr s
  = St s | B Bool | Z Int | Var String
  | App String [Expr s] | Sub (Expr s) (Substn s) | Undef
  deriving (Eq, Ord, Show)
type Substn s = [(String,Expr s)]
\end{verbatim}
  \caption{Expression and Substitution Datatypes (\texttt{CalcTypes.lhs})}
  \label{fig:expr-types}
\end{figure}
Both types are parameterised on a generic state type \texttt{s},
which allows us to be able to reason independently
of any particular notion of state.
We provide booleans (\texttt{B}),
integers (\texttt{Z}),
values of the generic state type (\texttt{St}),
and named function application (\texttt{App}).
We also have substitution (\texttt{Sub}), which pairs an expression
with a substitution (\texttt{Substn}),
which is a list of variable/expression pairs.
The \texttt{deriving} clause for \texttt{Expr}
enables the Haskell default notions
of equality, ordering and display for the type.

\subsection{Set Expressions}

We shall explore the use of the \texttt{Expr}  datatype
by indicating how the notions of sets and some basic operators
could be defined with the calculator.
We shall represent sets as instances of \texttt{App} with the name ``set'',
and the subset relation as an \texttt{App} with name ``subset'',
so the set $\setof{1,2}$  and predicate $S \subseteq T$
would be represented by
\verb$App "set" [Z 1,Z 2]$,
and
\verb$App "subset" [Var "S",Var "T"]$ respectively.
In practice, we would define constructor functions to build
these:
\begin{code}
set es = App "set" es
subset s1 s2 = App "subset" [s1,s2]
\end{code}
There is a standard interface for defining expression simplifiers:
define a function with the following type:
\begin{verbatim}
Dict s -> [Expr s] -> (String, Expr s)
\end{verbatim}
The first argument, of type \texttt{Dict},
is the dictionary currently in use.
The second argument is the list of sub-expressions of
the \texttt{App} construct for which the simplifier is intended.
The result is a pair consisting of a string and an expression.
If the simplification succeeds, then the string is non-empty
and gives some indication for the user
of what was simplified.
In this case the expression component is the simplified result.
If the simplification has no effect, then the string is empty,
and the expression returned is not defined.

The following code defines a simplifier for ``subset'',
which expects it to have precisely two set components:
\begin{code}
evalSubset d [App "set" s1,App "set" s2] = dosubset d s1 s2
evalSubset _ _ = none -- predefined shorthand for ("",Undef)
\end{code}
The two underscores in the second line are pattern matching
wildcards, so this catches all other possibilities.
It makes use of the following helper,
which gets the two lists of expressions associated with each set:
\begin{code}
dosubset d es1 es2 -- is es1 a subset of es2 ?
  | null (es1 \\ es2)  =  ( "subset", B True )
  | all (isGround d) ((es1 \\ es2) ++ es2)
                       =  ( "subset", B False )
  | otherwise          =  none
\end{code}
If the result of removing \texttt{es2} from \texttt{es1} is null
it then returns true.
If not, then if all elements remaining are ``ground'',
i.e., contain no variables, it returns false.
Otherwise, we cannot infer anything, so return \texttt{none}.

\subsection{Rendering Expressions}

The UTCP theory definitions and calculations
involve a lot of reasoning about sets,
leading to quite complicated expressions.
To avoid complex set expressions that are hard to parse visually,
a number of simplifying notations are desirable,
so that a singleton set $\setof x$ is rendered as $x$
and the very common idiom $S \subseteq ls$
is rendered as $ls(S)$,
so that for example, $ls(in)$ is short for $\setof{in} \subseteq ls$.
This shrinks the expressions to a much more readable form,
mainly by reducing the number of infix operators and set brackets.


When rendering expressions,
if an \texttt{App} construct is found, then its name
is looked up in the dictionary.
If an \texttt{ExprEntry} is not found, then the default rendering is used,
in which \verb$App "f" [e1,e2,..,en]$
is converted into \verb$f(e1,e2,..,en)$.
Otherwise, a function of type \verb$Dict s -> [Expr s] -> String$,
in that entry, is used to render the construct.

As far as expressions are concerned,
they become strings, and so are viewed as atomic
by the predicate pretty-printer (see Sect. \ref{sec:Predicates}).
So, we could show singleton sets without enclosing braces
by defining:
\begin{code}
showSet d [elm] = edshow d elm   -- drop {,} from singleton
showSet d elms = "{" ++ dlshow d "," elms ++ "}"
\end{code}
Here \texttt{edshow} (expression-dict-show)
displays its \texttt{elm} argument,
while \texttt{dlshow} (dictionary-list-show) displays the expressions
in \texttt{elms} separated by the \verb@","@ string.
Similar tricks are used to code a very compact rendering
of a mechanism that involves unique label generator expressions
that involve very deep nesting, such as:
\[
 \pi_2(new(\pi_1(new(\pi_2(split(\pi_1(new(g))))))))
\]
This can be displayed as \texttt{lg:2:},
using a very compact shorthand described in \cite{conf/tase/BMN16}
which we do not explain here.


\subsection{Expression Equality}

In contrast to the way that the subset predicate
is captured as an expression above,
the notion of expression equality is hardwired in,
as part of the predicate abstract syntax (see Sect. \ref{sec:Predicates}).
The simplifier will look at the two expression
arguments of that construct,
and if they are both instances of \texttt{App} with the same name,
will do a dictionary lookup, to see if there
is an entry, from
which an equality checking function can be obtained (\texttt{isEqual} component).
This has the following signature:
\begin{verbatim}
Dict s -> [Expr s] -> [Expr s] -> Maybe Bool
\end{verbatim}
The \texttt{Maybe} type constructor is standard Haskell, defined as
\begin{verbatim}
data Maybe t = Nothing | Just t
\end{verbatim}
It converts a type \texttt{t} into one which is now ``optional'',
or equivalently has a undefined value added.

The equality testing function takes a dictionary and the two expression
lists from the two \texttt{App} instances
and either returns \texttt{Nothing},
if it cannot establish the truth or falsity of the equality,
or \texttt{Just} the appropriate result.
Suitable code for \verb$"set"$ is the following
\begin{code}
eqSet d es1 es2
 = let ns1 = nub (sort es1)
       ns2 = nub (sort es2)
   in if all (isGround d) (ns1++ns2)
      then Just (ns1==ns2) else Nothing
\end{code}
The standard function \texttt{nub} removes duplicates,
which we do after we \texttt{sort}.
If both lists are ground we just do an equality comparison
and return \texttt{Just} it. Otherwise, we return \texttt{Nothing}.




\subsection{The Expression Entry}~

The dictionary entry for expressions has the following form:
\begin{verbatim}
ExprEntry
    { ecansub :: [String]
    , eprint :: Dict s -> [Expr s] -> String
    , eval :: Dict s -> [Expr s] -> (String, Expr s)
    , isEqual :: Dict s -> [Expr s] -> [Expr s] -> Maybe Bool}
\end{verbatim}
One big win in using a functional language like Haskell,
in which functions are first class data values,
is that we can easily define datatypes that
contain function-valued components.
We make full use of this in three of the entry kinds,
for expressions, predicates and laws.

The \texttt{eprint}, \texttt{eval} and \texttt{isEqual} components correspond
to the various examples we have seen above.
The \texttt{ecansub} component
indicates those variables occurring in the \texttt{App} expression
list for which it is safe to replace in substitutions.

To understand the need for \texttt{ecansub},
consider the following shorthand definition for an expression:
\[
  D(L) \defs L \subseteq ls
\]
in a context where we know that $L$ is a set expression defined
only over variables $g$, $in$ and $out$.
The variable $ls$ is not free in the lhs,
but does occur in the rhs.
A substitution of the form $[E/ls]$ say,
would leave the lhs unchanged, but alter the rhs to $L \subseteq E$.
For this reason the entry for $D$ would need to disallow substitution
for $ls$.
The \texttt{ecansub} entry lists the variables for which substitution
is safe with the expression as-is.
With the definition above, the value of this entry
 should be \texttt{["g","in","out"]}.
If we want to state that any substitution is safe,
then we use the ``wildcard'' form: \texttt{["*"]}.
We choose to list the substitutable variables
rather than those that are non-substitutable,
because the former is always easy to determine,
whereas the latter can be very open ended.

Given all of the above,
we can define dictionary entries for set and subset as
\begin{code}
setUTCPDict = makeDict
 [ ("set",(ExprEntry ["*"] showSet evalSet eqSet))
 , ("subset",(ExprEntry ["*"] showSubSet evalSubset noEq)) ]
\end{code}
Here \texttt{noEq} is an equality test function that always returns \texttt{Nothing}.
