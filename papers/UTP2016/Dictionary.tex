\section{Dictionaries}\label{sec:Dictionaries}

\begin{figure}
  \centering
%\begin{verbatim}
\texttt{Alf Obs Obs' Mdl Mdl' Scr Scr' Dyn Dyn' Stc laws}
%\end{verbatim}
  \caption{Reserved Dictionary Names}\label{fig:rsvd-names}
\end{figure}

The dictionary datatype (\texttt{Dict s}),
parameterised with a generic type \texttt{s},
is critical to the functioning of the calculator.
\begin{verbatim}
type Dict s = M.Map String (Entry s)
-- M is the renamed import of Data.Map
\end{verbatim}
It is the basic way in which the user of calculator
describes the alphabet, definitions and laws associated with their theory.

The dictionary uses the Haskell \texttt{String} datatype for keys,
and contains four different kinds of entries: alphabets,
expressions, predicates and laws.
{\small
\begin{verbatim}
data Entry s = AlfEntry .. | ExprEntry .. | PredEntry .. | LawEntry ..
\end{verbatim}
}
For simplicity, there is only one namespace involved,
and some names are reserved for special purposes.
These are listed in Fig. \ref{fig:rsvd-names}.
There are ten names that describe different
(overlapping)
parts of the theory alphabet (Fig. \ref{fig:rsvd-names}).
While is is possible to define these individually,
this can be quite error-prone,
so a function is provided to construct all these entries
from three basic pieces of information:
program variable names (`script', \texttt{Scr});
auxiliary variable names (`model',\texttt{Mdl}),
e.g. variables like $ok$ or $tr$ that don't represent variable values,
but instead some other observable program property of interest);
and static parameter variable names (\texttt{Stc}).
\begin{verbatim}
stdAlfDictGen :: [String] -> [String] -> [String] -> Dict s
\end{verbatim}
All lists contain undashed names, with dashes added when required
by the function.
So, the alphabet entries for the UTCP theory are defined as follows:
\begin{code}
dictAlpha = stdAlfDictGen ["s"] ["ls"] ["g","in","out"]
\end{code}
All of these entries will be of kind \texttt{AlfEntry},
i.e, just lists of the relevant variables.
\begin{verbatim}
   AlfEntry { avars  :: [String]}
\end{verbatim}

There are two important utility functions,
one that builds dictionaries from lists of string/entry pairs,
and another that merges two dictionaries, resolving conflicts
by merging entries if possible, other wise favouring the
second dictionary:
\begin{verbatim}
makeDict :: [(String, Entry s)] -> Dict s
dictMrg :: Dict s -> Dict s -> Dict s
\end{verbatim}
