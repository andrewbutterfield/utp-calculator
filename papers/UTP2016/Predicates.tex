\section{Predicates}\label{sec:Predicates}

In Fig. \ref{fig:pred-types} we show the Haskell declarations
of the datatypes to represent predicates.
\begin{figure}[tb]
\begin{verbatim}
data Pred s
  = T | F | PVar String | Equal (Expr s) (Expr s) | Atm (Expr s)
  | Comp String [MPred s] | PSub (MPred s) (Substn s) | PUndef

type MPred s = (Marks, Pred s)
\end{verbatim}
  \caption{Predicate Datatypes (\texttt{CalcTypes.lhs})}
  \label{fig:pred-types}
\end{figure}
Both datatypes are parameterised on a generic state type \texttt{s},
which allows us to be able to handle different concrete types
of shared state with one piece of code.
Both types have basic values (\texttt{St},\texttt{B},\texttt{Z},\texttt{T},\texttt{F}),
variables (\texttt{Var},\texttt{PVar}),
generic composites (\texttt{App},\texttt{Comp}),
substitution (\texttt{Sub},\texttt{PSub})
and undefined values (\texttt{Undef},\texttt{PUndef}).
The predicate datatype has a way to embed a (boolean-valued)
expression to form an atomic predicate (\texttt{Atm}),
as well as an explicit form for atomic predicates that take
the form of an equality (\texttt{Equal}).
We give a specialised form for equality simply because
its use in laws is widespread and worth optimising.

Also important to point out is the fact that \texttt{Pred} is
defined in terms of \texttt{MPred}, which in turn is defined
in terms of \texttt{Pred}.
This is done to facilitate the association of markings (lists of integers)
with predicate constructs.
These markings are used to indicate which parts of a predicate
were changed at each calculation step.
We will not discuss marking further in this document
as it runs completely ``under the hood'',
and its only impact on the users of this calculator
is their need to be aware of the interplay involving \texttt{Pred} and \texttt{MPred}.

The abstract syntax is very simple,
and uses names as Haskell Strings to distinguish between
different composites.
So for example logical operators $\land$ and $\lor$
might be represented as \texttt{Comp "And"} and \texttt{Comp "Or"}
respectively.
We have a default way to pretty-print composites,
so that the predicate $\true \land \false$,
represented in Haskell by \texttt{Comp "And" [bT,bF]},
would render as \texttt{And(true,false)}.
However it is nice to be able to render it as the usual
infix operator, giving \texttt{true /\BS\ false}.
In addition, we need some way to associate laws and definitions,
as appropriate, with composites.

\subsection{Coding Atomic Semantics}

Formally, using our shorthand notations, we define atomic behaviour as:
\[
    \A(A) \defs ls(in) \land A \land ls'=ls\ominus(in,out)
\]
where $A$ is a predicate whose alphabet is restricted to $s$ and $s'$.
We code this up as follows:
\begin{code}
patm mpr = comp "A" [mpr] -- we assume mpr has only s, s' free

defnAtomic d [a] = ldefn shPAtm $ mkAnd [lsin,a,ls'eqlsinout]

inp = Var "in" -- 'in' is a Haskell keyword
out = Var "out"
lsin = atm $ App "subset" [inp,ls]
lsinout = App "sswap" [ls,inp,out]
ls'eqlsinout = equal ls' lsinout

patmEntry
 = ( nPAtm
   , PredEntry [] ppPAtm [] defnAtomic (pNoChg nPAtm) )
\end{code}
Here \texttt{atm} lifts an expression to a marked predicate,
while \texttt{"sswap"} names the ternary operation $\_\ominus(\_,\_)$,
and \texttt{equal} is the marked form of \texttt{Equal}.


We won't show the encoding of the composite constructs,
or a predicate transformer called $run$ that actually
enables us to symbolically `execute' our semantics.
We will show how the \texttt{pprint} entry for
sequential composition in UTCP is defined,
just to show how easy support for infix notation is.
\begin{code}
ppPSeq d ms p [mpr1,mpr2]
 = paren p precPSeq -- parenthesise if precedence requires it
     $ ppopen  (pad ";;") [ mshowp d ms precPSeq mpr1
                          , mshowp d ms precPSeq mpr2 ]
\end{code}
Here \texttt{pad} puts spaces around its argument,
while \texttt{ppopen} uses its first argument as a
separator between all the elements of its second list argument.
The function \texttt{mshowp} is the top-level predicate printer.

\subsubsection{Predicate Entries}~

\begin{verbatim}
Entry s
  = ...
  | PredEntry
    { pcansub :: [String]
    , pprint :: Dict s -> MarkStyle -> Int -> [MPred s] -> PP
    , alfa :: [String]
    , pdefn :: Rewrite s
    , prsimp :: Rewrite s }
type Rewrite s = Dict s -> [MPred s] -> (String, Pred s)
\end{verbatim}
The predicate entry associated with Comp name pargs
has five fields:
\begin{description}
  \item[\texttt{pcansub}]
    does for predicates what \texttt{ecansub} does for expressions.
  \item[\texttt{pprint}]
    is similar to \texttt{eprint} except it has two extra arguments,
    used to help with special renderings for predicates marked as changed,
    and precedence levels for managing bracketing.
    We do pretty-printing tricks such as indenting and adding line-breaks
    at the predicate level.
  \item[\texttt{alfa}]
    is used to identify the alphabet of the predicate.
  \item[\texttt{pdefn}]
    is a function invoked when \texttt{name} has a definition,
    and we want to expand it.
  \item[\texttt{prsimp}]
    is called by the simplifier when processing \texttt{name}.
\end{description}
Both \texttt{pdefn} and \texttt{prsimp} take a dictionary argument
and list of sub-component (marked)predicates,
returning a string/predicate pair,
interpreted in a manner similar to \texttt{eval} above.


With sets, we can define a notion of equality as follows,
where the first parameter
\begin{code}
eqSet d es1 es2
 = let ns1 = nub $ sort $ es1                -- normalise sets
       ns2 = nub $ sort $ es2
   in if all (isGround d) (ns1++ns2)
      then Just (ns1==ns2)
      else Nothing
\end{code}


The data-structure that logs the complete calculation outcome
is a triple consisting of the final predicate,
a list of the steps, each with a justification string,
and the dictionary that was used.
\begin{verbatim}
type CalcLog s = (MPred s, [RWResult s], Dict s)
\end{verbatim}

