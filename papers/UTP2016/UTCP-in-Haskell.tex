
\subsection{UTCP Syntax}

We start by defining the syntax of our language
\[
   P ::= A \mid P \pseq P \mid P \parallel P \mid P \pcond P \mid \piter P
\]
and assign them pretty printing precedences,
so they work well with the definitions in modules
\texttt{StdPrecedences} and \texttt{StdUTPPrecedences}.
\begin{code}
precPCond = 5 + precSpacer  1
precPPar  = 5 + precSpacer  2
precPSeq  = 5 + precSpacer  3
precPIter = 5 + precSpacer  6
\end{code}
The \texttt{precSpacer} function is used to leave space between precedence
values so that new constructs can be given an in-between spacing,
if required. Currently it returns its argument times 1.


\subsection{UTCP Alphabet}

As already stated, the theory alphabet is $s,s',ls,ls',g,in,out$.
We declare each as a variable in our expression notation,
noting that Haskell allows identifiers to contain dashes,
which proves very convenient:
\begin{code}
s' = Var "s'"
\end{code}
Note, here \texttt{s'} is a Haskell variable of type \texttt{Expr},
while \texttt{"s'"} is a Haskell literal value of type \texttt{String}.

We have two ways to classify UTP observation variables.
Along one axis we distinguish observations of program variable
values (``script'' variables, e.g. $s$, $s'$) from those that record other
observations such as termination/stability,
or traces/refusals (``model'' variables, e.g. $ls$, $ls'$).
On the other axis we distinguish observations
that are dynamic, whose values change as the program runs
(e.g. $s$, $ls$ with $s'$ and $ls'$)
from those that are static,
unchanged during program execution (e.g. $g$, $in$ and $out$).
We have pre-defined names for these categories,
and a function \texttt{stdAlfDictGen} that
builds all the appropriate entries
given three lists of script, dynamic model and static variable strings.
We also declare that the predicate variables $A$, $B$ and $C$
will refer to atomic state-changes,
and so only have alphabet $\setof{s,s'}$.
\begin{code}
alfUTCPDict
 = dictMrg dictAlpha dictAtomic
 where
   dictAlpha = stdAlfDictGen ["s"] ["ls"] ["g","in","out"]
   dictAtomic = makeDict [ pvarEntry "A" ss'
                         , pvarEntry "B" ss'
                         , pvarEntry "C" ss' ]
   ss' = ["s", "s'"]
\end{code}
(See modules
\texttt{CalcAlphabets}
, \texttt{CalcPredicates}
, \texttt{StdPredicates}.)

\subsection{UTCP Expressions}

We have sets of labels
so we need a way to implement set-expressions.
To avoid long set expressions a number of shorthands are desirable,
so that a singleton set $\setof x$ is rendered as $x$
and the very common idiom $S \subseteq ls$
is rendered as $ls(S)$,
so that for example, $ls(in)$ is short for $\setof{in} \subseteq ls$.
So we encode a set construct as follows:
\begin{code}
set = App "set"                             -- set constructor
showSet d [elm] = edshow d elm      -- drop {,} from singleton
showSet d elms = "{" ++ dlshow d "," elms ++ "}"
\end{code}
We also define an equality tester for sets,
that gets the two element-lists
\begin{code}
eqSet d es1 es2
 = let ns1 = nub $ sort $ es1                -- normalise sets
       ns2 = nub $ sort $ es2
   in if all (isGround d) (ns1++ns2)
      then Just (ns1==ns2)
      else Nothing
\end{code}
The predicate \texttt{isGround} checks to see if an expression has no
dynamic variables.
For the purposes of this theory at least,
we know we can treat these expressions as values.
This is a common feature of encoding theories for this calculator%
---%
knowing when a particular simplification makes sense.
The dictionary entry for the set construct then looks like
\begin{code}
ExprEntry ["*"] showSet none eqSet
\end{code}
where we permit any substitutions directly on the elements,
and we use the special builtin function \texttt{none}
as an evaluator that does not make any changes,
since we regard these sets as evaluated, in this theory.

Similar tricks are used to code a very compact rendering
of a mechanism that involves unique label generators
that can also be split, so that an expression like
\[
 \pi_2(new(\pi_1(new(\pi_1(split(\pi_1(new(g))))))))
\]
can be displayed as $\ell_{g:1:}$, or,
within the calculator, as \texttt{lg:1:} .


\subsection{Coding Atomic Semantics}

Formally, using our shorthand notations, we define atomic behaviour as:
\[
    \A(A) \defs ls(in) \land A \land ls'=ls\ominus(in,out)
\]
where $A$ is a predicate whose alphabet is restricted to $s$ and $s'$.
We code this up as follows:
\begin{code}
patm mpr = comp "A" [mpr] -- we assume mpr has only s, s' free

defnAtomic d [a] = ldefn shPAtm $ mkAnd [lsin,a,ls'eqlsinout]

inp = Var "in" -- 'in' is a Haskell keyword
out = Var "out"
lsin = atm $ App "subset" [inp,ls]
lsinout = App "sswap" [ls,inp,out]
ls'eqlsinout = equal ls' lsinout

patmEntry
 = ( nPAtm
   , PredEntry [] ppPAtm [] defnAtomic (pNoChg nPAtm) )
\end{code}
Here \texttt{atm} lifts an expression to a marked predicate,
while \texttt{"sswap"} names the ternary operation $\_\ominus(\_,\_)$,
and \texttt{equal} is the marked form of \texttt{Equal}.


We won't show the encoding of the composite constructs,
or a predicate transformer called $run$ that actually
enables us to symbolically `execute' our semantics.
We will show how the \texttt{pprint} entry for
sequential composition in UTCP is defined,
just to show how easy support for infix notation is.
\begin{code}
ppPSeq d ms p [mpr1,mpr2]
 = paren p precPSeq -- parenthesise if precedence requires it
     $ ppopen  (pad ";;") [ mshowp d ms precPSeq mpr1
                          , mshowp d ms precPSeq mpr2 ]
\end{code}
Here \texttt{pad} puts spaces around its argument,
while \texttt{ppopen} uses its first argument as a
separator between all the elements of its second list argument.
The function \texttt{mshowp} is the top-level predicate printer.


\subsection{Coding UTCP Laws}

The definition of the semantics of the UTCP language
constructs, and of $run$,
make use of the (almost) standard notions of skip,
sequential composition
and iteration in UTP.
The versions used here are slightly non-standard because we have
non-homogeneous relations,
where the static parameters have no dashed counterparts.
In essence we ignore the parameters as far as flow-of-control is concerned:
\RLEQNS{
   \Skip &\defs& s'=s \land ls'=ls
\\ P ; Q
   &\defs&
   \exists s_m,ls_m @
     P[s_m,ls_m/s',ls']
     \land
     Q[s_m,ls_m/s,ls]
\\ c * P &\defs& \mu L @ (P ; L) \cond c \Skip
\\ P \cond c Q &\defs& c \land P \lor \lnot c \land Q
}
Here, the definition of $\cond\_$ is entirely standard, of course.

What is key here though,
is realising that we do not want to define the constructs
as above and use them directly, as it involves
quantifiers and explicit recursion,
both of which would introduce considerable complexity to the calculator.
Instead, we encode useful laws that they satisfy,
that do not require their definitions to be expanded.
Such laws might include the following:
\RLEQNS{
  \Skip \seq\ P & {} = P = {} & P \seq \Skip
\\ c * P &=& (P \seq c* P ) \cond c \Skip
\\ (c * P)[e/x] &=& P[e/x] \seq c * P, \qquad if c[e/x]
\\ (c * P)[e/x] &=& \Skip[e/x] , \qquad if \lnot c[e/x]
}
These laws need to be proven by hand (carefully),
by the theory developer, and then encoded into Haskell
(equally carefully), as we are about to describe.

We can easily give a definition of $\Skip$,
which is worth expanding.
\RLEQNS{
   \Skip &\defs& s'=s \land ls'=ls
}
\begin{code}
defnUTCPII = mkAnd[ equal s' s, equal ls' ls ]
\end{code}

For more complex laws,
the idea is that we pattern-match on predicate syntax
to see if a law is applicable (we have its lefthand-side),
and if so,
we then build an appropriate instance of the righthand-side.
The plan is that we gather all these pattern/outcome pairs
in one function definition,
which will try them in order.
This is a direct match for how Haskell pattern-matching works.
So for UTCP we have a function called \texttt{reduceUTCP},
structured as follows:
\begin{code}
reduceUTCP :: (Show s, Ord s) => DictRWFun s
reduceUTCP (...1st law pattern...) = 1st outcome
reduceUTCP (...2nd law pattern...) = 2nd outcome
...
reduceUTCP d mpr = lred "" mpr  -- catch-all at end, no change
\end{code}
The last clause matches any predicate
and simply returns it with a null string,
indicating no change took place.
The main idea is find a suitable collection of patterns,
in the right order,
to be most effective in performing calculations.
The best way to determine this is start with none,
run the calculator and when it stalls
(no change is happening for any command),
see what law would help make progress, and encode it.
This is the essence of the agile approach to theory calculator development.

A simple example of such a pattern is the following encoding
of $\Skip;P = P$ :
\begin{code}
reduceUTCP d
 (_,Comp "Seq" [(_,Comp "Skip" []), mpr]) = lred ";-lunit" mpr
\end{code}
The second argument has type marked-predicate (\texttt{MPred})
which is a marking/predicate pair.
We are not interested in the markings
so we use the wildcard pattern '\verb"_"'
for the first pair component.
The sub-pattern in the second pair component,
\verb'Comp "Seq" [(_,Comp "Skip" []), mpr])',
matches a composite called ``Seq'',
with a argument list containing two (marked) predicates.
The first argument predicate pattern \verb'(_,Comp "Skip" [])'
matches a ``Skip'' composite with no further subarguments.
The second argument pattern \verb'mpr' matches an arbitrary predicate
($P$ in the law above).
The righthand-side returns the application \verb'lred ";-lunit" mpr'
which simply constructs a string/predicate pair,
with the string being a justification note that says a reduction-step
using a law called ``$;$-lunit'' was applied.



\subsection{UTCP Recognisers}

Some laws require matching that is a bit more sophisticated.
For example,
consider a useful reduction for tidying up at the end,
assuming that $ls' \notin A$ and $ls \notin B$, and both $k$ and $h$
are ground:
\[
   (A \land ls'=k) ; (B \land ls'= h)
   \quad\equiv\quad
   (A;B) \land ls'=h
   \qquad \elabel{$ls'$-cleanup}
\]
However, we want this law to work when both $A$
and $B$ are themselves conjunctions, with the $ls'=\dots$
as part of the same conjunction, located at some arbitrary position.
We can break the problem into two parts.
First we do a top-level pattern match
to see that we have a top-level sequential composition
of two conjunctions,
then we use a function that will check both conjunction predicate-lists
for the existence of a $ls'=\dots$ component,
and that everything else also satisfies the requirements regarding
the occurrence, or otherwise of $ls$ or $ls'$:
\begin{code}
reduceUTCP d pr@(_,Comp "Seq" [ (_,Comp "And" pAs)
                              , (_,Comp "And" pBs)])
 = case isSafeLSDash d ls' ls' pAs of -- no ls' in rest of pAs
    Nothing -> lred "" pr
    Just (_,restA) ->
     case isSafeLSDash d ls' ls pBs of -- no ls in rest of pBs
      Nothing -> lred "" pr
      Just (eqB,restB)
       -> lred "ls'-cleanup" $   -- build RHS
             comp "And" [ comp "Seq" [ bAnd restA
                                     , bAnd restB ]
                        , eqB ]
\end{code}
The function \texttt{isSafeLSDash}
is designed to
(i) locate the $ls'=e$ conjunct and check that its rhs is a ground expression;
(ii) check that none of the remaining conjuncts make use of the
`unwanted' version of the label-set variable ($ls$ or $ls'$);
and (iii), if all ok, return a pair
whose first component is the ($ls'=\dots$) equality,
and whose second is the list of other conjuncts.
To achieve (i) above,
we make use of two functions provided by the \texttt{CalcRecogniser} module:
\begin{code}
mtchNmdObsEqToConst :: Ord s => String -> Dict s -> Recogniser s
matchRecog :: (Ord s, Show s)
           => Recogniser s -> [MPred s]
           -> Maybe ([MPred s],(MPred s,[MPred s]),[MPred s])
\end{code}
where
\begin{code}
type Recogniser s = MPred s -> (Bool, [MPred s])
\end{code}
A recogniser is a function that takes a predicate
and if it ``recognises'' it, returns \texttt{(True, parts)},
where parts are the subcomponents of the predicate in some order.
The recogniser \texttt{mtchNmdObsEqtoConst v d} matches a predicate of the form
\texttt{Equal (Var v) k}, returning a list with both parts.
The function \texttt{matchRecog} takes a recogniser and list of predicates
and looks in the list for the first predicate to satisfy
the recogniser, returning a triple of the form
(before,satisyingPred,after).
If the recogniser succeeds,
we then check the validity of the expression,
and the absence of the unwanted variable from the
rest of the conjuncts --- using boolean function
 \texttt{dftlyNotInP} (``definitely not in $P$''),
 so handling task (ii) above.
\begin{code}
isSafeLSDash d theLS unwanted prs
 = case matchRecog (mtchNmdObsEqToConst theLS d) prs of
    Nothing -> Nothing
    Just (pre,(eq@(_,Equal _ k),_),post) ->
     if notGround d k
      then Nothing
      else if all (dftlyNotInP d unwanted) rest
       then Just (eq,rest)
       else Nothing
     where rest = pre++post
\end{code}

\subsection{Conditional Reductions}

To avoid having to support a wide range of expression-related theories,
we provide a conditional reducer, that computes
a number of alternative outcomes, each guarded by some predicate
that is hard to evaluate.
The user elects which one to use by checking the conditions manually.
We define a function, similar to reduceUTCP,
that contains a series of patterns for each conditional reduction law.
\begin{code}
creduceUTCP :: (Show s, Ord s) => CDictRWFun s
\end{code}
Provided that $\vec x \subseteq in\alpha P$
 (which in this case is $\setof{s,ls}$):
\RLEQNS{
   c[\vec e/\vec x]
   &\implies&
   (c * P)[\vec e/\vec x] = P[\vec e/\vec x] ; c * P
\\ \lnot c[\vec e/\vec x]
   &\implies&
   (c * P)[\vec e/\vec x] = \Skip[\vec e/\vec x]
}
\begin{code}
creduceUTCP d (_,PSub w@(_,Comp "Iter" [c,p]) sub)
 | isCondition c        -- true if expr c is a UTP 'condition'
   && beforeSub d sub   -- true if subst-vars are all undashed
 = lcred "loop-substn" [ctrue,cfalse]
 where
   csub = psub c sub            --  psub builds a substitution
   ctrue  = ( psimp d csub          -- psimp runs a simplifier
            , bSeq (psub p sub) w )
   cfalse = ( psimp d $ bNot csub
            , psub bSkip sub )
\end{code}
If this succeeds, the user is presented with two options,
each of the form (side-condition, outcome)
The user can then identify which of those side-conditions is true,
resulting in the appropriate outcome.

We make these two reduction functions ``known'' to the calculator
by adding them into a dictionary.
\begin{code}
lawsUTCPDict
 = makeDict [("laws", LawEntry [reduceUTCP] [creduceUTCP] [])]
\end{code}
We then can take a number of partial dictionaries and use various
dictionary functions,
defined in \texttt{CalcPredicates}, to merge them together.
\begin{code}
dictUTCP = foldl1 dictMrg [ alfUTCPDict, ..., lawsUTCPDict]
\end{code}
