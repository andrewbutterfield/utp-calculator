\subsection{Before we dive in \dots}

The UTP Calculator is implemented as a series
of Haskell modules,
which are broken into two groups:
\begin{description}
  \item[Infrastructure]
    are modules that implement the calculator mechanics,
    pretty-printing, etc.
    These include \texttt{PrettyPrint},
    and all modules with names starting with \texttt{Calc}.
  \item[Builtin Theories]
    are pre-defined theory modules that cover standard logic,
    whose names start with \texttt{Std}, and modules that cover ``standard''
    UTP, whose names start with \texttt{StdUTP}.
    These theory modules typically come in three, covering
    \texttt{Predicates}, \texttt{Precedences} and \texttt{Laws}.
\end{description}
All the Haskell modules are found in the \texttt{src} directory
of the repository, with a \texttt{.lhs} file extension
(e.g., \texttt{CalcTypes.lhs}).

\subsection{UTCP Syntax}

We start by defining the syntax of our language
\RLEQNS{
   p,q \in UTCP
   &::=& Idle
  \mid  \A(A)
  \mid  p \lseq q
  \mid  p \lcond c q
  \mid  p \parallel q
  \mid  c \wdo p
}
\noindent
and assign them pretty printing precedences,
so they work well with the definitions in modules
\texttt{StdPrecedences} and \texttt{StdUTPPrecedences}.
\begin{code}
precPCond = 5 + precSpacer  1
precPPar  = 5 + precSpacer  2
precPSeq  = 5 + precSpacer  3
precPIter = 5 + precSpacer  6
\end{code}


\subsection{UTCP Alphabet}

As already stated, the theory alphabet is $s,s',ls,ls',g,in,out$.
We declare each as a variable in our expression notation,
noting that Haskell allows identifiers to contain dashes,
which proves very convenient:
\begin{code}
s' = Var "s'"
\end{code}
Note, here \texttt{s'} is a Haskell variable of type \texttt{Expr},
while \texttt{"s'"} is a Haskell literal value of type \texttt{String}.

We have two ways to classify UTP observation variables.
Along one axis we distinguish observations of program variable
values (``script'' variables, e.g. $s$, $s'$) from those that record other
observations such as termination/stability,
or traces/refusals (``model'' variables, e.g. $ls$, $ls'$).
On the other axis we distinguish observations
that are dynamic, whose values change as the program runs
(e.g. $s$, $ls$ with $s'$ and $ls'$)
from those that are static,
unchanged during program execution (e.g. $g$, $in$ and $out$).
We have pre-defined names for these categories,
and an function \texttt{stdAlfDictGen} that
builds all the appropriate entries
given three lists of script, dynamic model and static variable strings.
We also declare that the predicate variables $A$, $B$ and $C$
will refer to atomic state-changes,
and so only have alphabet $\setof{s,s'}$.
\begin{code}
alfUTCPDict
 = dictMrg dictAlpha dictAtomic
 where
   dictAlpha = stdAlfDictGen ["s"] ["ls"] ["g","in","out"]
   dictAtomic = makeDict [ pvarEntry "A" ss'
                         , pvarEntry "B" ss'
                         , pvarEntry "C" ss' ]
   ss' = ["s", "s'"]
\end{code}
(See modules
\texttt{CalcAlphabets}
, \texttt{CalcPredicates}
, \texttt{StdPredicates}.)

\subsection{UTPC Expressions}

We have sets of labels
so we need a way to implement set-expressions.
To avoid long set expressions a number of shorthands are desirable,
so that a singleton set $\setof x$ is rendered as $x$
and the very common idiom $S \subseteq ls$
is rendered as $ls(S)$,
so that for example, $ls(in)$ is short for $\setof{in} \subseteq ls$.
So we might encode a set construct as follows
\begin{code}
set = App "set"                             -- set constructor
showSet d [elm] = edshow d elm      -- drop {,} from singleton
showSet d elms = "{" ++ dlshow d "," elms ++ "}"
\end{code}
We also define an equality tester for sets,
that gets the two element-lists
\begin{code}
eqSet d es1 es2
 = let ns1 = nub $ sort $ es1                -- normalise sets
       ns2 = nub $ sort $ es2
   in if all (isGround d) (ns1++ns2)
      then Just (ns1==ns2)
      else Nothing
\end{code}
The predicate \texttt{isGround} checks to see if an expression has no
dynamic variables.
For the purposes of this theory at least,
we know we can treat these expressions as values.
This is a common feature of encoding theories for this calculator%
---%
knowing when a particular simplification makes sense.
The dictionary entry for the set construct then looks like
\begin{code}
ExprEntry ["*"] showSet none eqSet
\end{code}
where we permit any substitutions directly on the elements,
and we use the special builtin function \texttt{none}
as an evaluator that does not make any changes,
since we regard these sets as evaluated, in this theory.

Similar tricks are used to code a very compact rendering
of a mechanism that involves unique label generators
that can also be split, so that an expression like
\[
 \pi_2(new(\pi_1(new(\pi_1(split(\pi_1(new(g))))))))
\]
can be displayed as $\ell_{g:1:}$, or,
within the calculator, as \texttt{lg:1:} .


\subsection{Coding Atomic Semantics}

Formally, using our shorthand notations, define atomic behaviour as:
\RLEQNS{
    \A(A) &\defs& ls(in) \land A \land ls'=ls\ominus(in,out)
}
where $A$ is a predicate whose alphabet is restricted to $s$ and $s'$.
We code this up as follows:
\begin{code}
patm mpr = comp "A" [mpr] -- we assume mpr has only s, s' free

defnAtomic d [a] = ldefn shPAtm $ mkAnd [lsin,a,ls'eqlsinout]

inp = Var "in" -- 'in' is a Haskell keyword
out = Var "out"
lsin = atm $ App "subset" [inp,ls]
lsinout = App "sswap" [ls,inp,out]
ls'eqlsinout = equal ls' lsinout

patmEntry
 = ( nPAtm
   , PredEntry [] ppPAtm [] defnAtomic (pNoChg nPAtm) )
\end{code}
Here \texttt{atm} lifts an expression to a marked predicate,
while \texttt{"sswap"} names the ternary operation $\_\ominus(\_,\_)$,
and \texttt{equal} is the marked form of \texttt{Equal}.


We won't show the encoding of the composite constructs,
or a predicate transformer called $run$ that actually
enables us to symbolically `execute' our semantics.
We will show how the \texttt{pprint} entry for
sequential composition in UTCP is defined,
just to show how easy support for infix notation is.
\begin{code}
ppPSeq d ms p [mpr1,mpr2]
 = paren p precPSeq -- parenthesise if precedence requires it
     $ ppopen  (pad ";;") [ mshowp d ms precPSeq mpr1
                          , mshowp d ms precPSeq mpr2 ]
\end{code}
Here \texttt{pad} puts spaces around its argument,
while \texttt{ppopen} uses its first argument as a
separator between all the elements of its second list argument.
The function \texttt{mshowp} is the top-level predicate printer.


\subsection{Coding UTCP Laws}

The definition of the semantics of the UTCP language
constructs, and of $run$,
make use of the (almost) standard notions of skip,
sequential composition
and iteration in UTP.
The versions used here are slightly non-standard because we have
non-homogeneous relations,
where the static parameters have no dashed counterparts.
In essence we ignore the parameters as far as flow-of-control is concerned:
\RLEQNS{
   \Skip &\defs& s'=s \land ls'=ls
\\ P ; Q
   &\defs&
   \exists s_m,ls_m @
     P[s_m,ls_m/s',ls']
     \land
     Q[s_m,ls_m/s,ls]
\\ c * P &\defs& \mu L @ (P ; L) \cond c \Skip
\\ P \cond c Q &\defs& c \land P \lor \lnot c \land Q
}
Here, the definition of $\cond\_$ is entirely standard, of course.

What is key here though,
is realising that we do not want to define the constructs
as above and use them directly, as it involves
quantifiers and explicit recursion,
both of which would introduce considerable complexity to the calculator.
Instead, we encode useful laws that they satisfy,
that do not require their definitions to be expanded.
Such laws might include the following:
\RLEQNS{
  \Skip \seq\ P & {} = P = {} & P \seq \Skip
\\ c * P &=& (P \seq c* P ) \cond c \Skip
\\ (c * P)[e/x] &=& P[e/x] \seq c * P, \qquad if c[e/x]
\\ (c * P)[e/x] &=& \Skip[e/x] , \qquad if \lnot c[e/x]
}
These laws need to be proven by hand (carefully),
by the theory developer, and then encoded into Haskell
(equally carefully), as we are about to describe.

We can easily give a definition of $\Skip$,
which is worth expanding.
\RLEQNS{
   \Skip &\defs& s'=s \land ls'=ls
}
\begin{code}
defnUTCPII = mkAnd[ equal s' s, equal ls' ls ]
\end{code}

For more complex laws,
the idea is that we pattern-match on predicate syntax
to see if a law is applicable (we have its lefthands-side),
and if so,
we then build an appropriate instance of the righthand-side.
The plan is that we gather all these pattern/outcome pairs
in one function definition,
which will try them in order.
This is a direct match for how Haskell pattern-matching works.
So for UTCP we have a function called reduceUTCP,
structured as follows:
\begin{code}
reduceUTCP :: (Show s, Ord s) => DictRWFun s
reduceUTCP (...1st law pattern...) = 1st outcome
reduceUTCP (...2nd law pattern...) = 2nd outcome
...
reduceUTCP d mpr = lred "" mpr -- catch-all at end, no change
\end{code}
The last clause matches any predicate
and simply returns it with a null string,
indicating no change took place.
The main idea is find a suitable collection of patterns,
in the right order,
to be most effective in performing calculations.
The best way to determine this is start with none,
run the calculator and when it stalls
(no change is happening for any command),
see what law would help make progress, and encode it.

A simple example of such a pattern is the following encoding
of $\Skip;P = P$ :
\begin{code}
reduceUTCP d   -- the dictionary is 'd'
  (_, Comp "Seq" [(_,Comp "Skip" []), mpr]) = lred ";-lunit" mpr
\end{code}
The second argument has type marked-predicate (\texttt{MPred})
which is a marking/predicate pair.
We are not interested in the markings
so we use the wildcard pattern '\verb"_"'
for the first pair component.
The sub-pattern in the second pair component,
\verb'Comp "Seq" [(_,Comp "Skip" []), mpr])',
matches a composite called ``Seq'',
with a argument list containing two (marked) predicates.
The first argument predicate pattern \verb'(_,Comp "Skip" [])'
matches a ``Skip'' composite with no further subarguments.
The second argument pattern \verb'mpr' matches an arbitrary predicate
($P$ in the law above).
The righthand-side returns the application \verb'lred ";-lunit" mpr'
which simply constructs a string/predicate pair,
with the strign being a justification note that says a reduction-step
usimng a laws called ``;-lunit'' was applied.



\subsubsection{UTCP Recognisers}

Some laws require matching that is a bit more sophisticated,
often with side-conditions.

$s'=s$, or $s=s'$
\begin{code}
isIdle s1 s2 = s1=="s" && s2=="s'" || s1=="s'" && s2=="s"
\end{code}
$s'=s$ conjoined with $A$ whose alphabet is $\setof{s,s'}$.
\begin{code}
isIdleSeqAtom d s1 s2 pA
 | isIdle s1 s2
    = case plookup pA d of
       Just (PredEntry _ _ a_alf _ _)  ->  sort a_alf == ["s","s'"]
       _                               ->  False
 | otherwise  =  False
\end{code}




\subsubsection{Skip and Sequential Composition}


A law that tidies up a common situation towards
the end of calculations.
\RLEQNS{
   s' = e \land ls' = f ; A
   &=&
   A[e,f/s,ls]
   & \elabel{$s'$-$ls'$-$;$-prop}
}
\begin{code}
reduceUTCP d (_,Comp "Seq"
                [ (_,Comp "And" [ (_,Equal (Var "s'") e)
                                , (_,Equal (Var "ls'") f) ])
                , mpA ])
 = lred "s'ls'-;-prop" $ psub mpA [("s",e),("ls",f)]
reduceUTCP d (_,Comp "Seq"
                [ (_,Comp "And" [ (_,Equal (Var ls'@"ls'") f)
                                , (_,Equal (Var s'@"s'") e) ])
                , mpA])
 = lred "s'ls'-;-prop" $ psub mpA [("s",e),("ls",f)]
\end{code}

A useful reduction for tidying up at the end,
assuming that $ls' \notin A$ and $ls \notin B$, and both $k$ and $h$
are ground:
\RLEQNS{
   A \land ls'=k ; B \land ls'= h
   &\equiv&
   (A;B) \land ls'=h
   & \elabel{$ls'$-cleanup}
}
\begin{code}
reduceUTCP d pr@(_,Comp "Seq" [ (_,Comp "And" pAs)
                              , (_,Comp "And" pBs)])
 = case isSafeLSDash d ls' ls' pAs of
    Nothing -> lred "" pr
    Just (_,restA) ->
     case isSafeLSDash d ls' ls pBs of
      Nothing -> lred "" pr
      Just (eqB,restB)
       -> lred "ls'-cleanup" $
             comp "And" [ comp "Seq" [ bAnd restA
                                     , bAnd restB ]
                        , eqB ]
 where
   ls = "ls"
   ls' = "ls'"

   isSafeLSDash d theLS unwanted prs
    = case matchRecog (mtchNmdObsEqToConst theLS d) prs of
       Nothing -> Nothing
       Just (pre,(eq@(_,Equal _ k),_),post) ->
        if notGround d k
         then Nothing
         else if all (dftlyNotInP d unwanted) rest
          then Just (eq,rest)
          else Nothing
        where rest = pre++post
\end{code}


That's all folks!


%%%%  CREDUCE


To avoid having to support a wide range of expression-related theories,
we provide a conditional reducer, that computes
a number of alternative outcomes, each guarded by some predicate
that is hard to evaluate.
The user elects which one to use by checking the conditions manually.

\begin{code}
creduceUTCP :: (Show s, Ord s) => CDictRWFun s
\end{code}

\paragraph{pre- and before-substitutions}
A pre-substitution is one that replaces undashed variables with
undashed expressions, while a before-substitution is further restricted
to replacing undashed observables only.
\begin{code}
preSublet :: Ord s => ( String, Expr s ) -> Bool
preSublet (v,e) = notDash v && notDashed e

preSub :: Ord s => Substn s -> Bool
preSub = all preSublet

beforeSublet :: Ord s => Dict s -> ( String, Expr s ) -> Bool
beforeSublet d (v,e) = isDyn d v && notDashed e

beforeSub :: Ord s => Dict s -> Substn s -> Bool
beforeSub d = all (beforeSublet d)
\end{code}

\paragraph{Predicate Simplifier}
 Sometimes we want to simplify a predicate without fuss
(marking or comment):
\begin{code}
psimp :: (Ord s, Show s)
      => Dict s -> MPred s -> Pred s
psimp d = snd . thd . simplify d startm

thd (_,_,z) = z
\end{code}


\paragraph{Atomic Enablement}

\RLEQNS{
   ns(\ell_0)
   &\implies&
   \A(A)[\ell_0,\ell_1,ns/in,out,ls]
    = A \land ls'=ns\ominus(\ell_0,\ell_1)
\\ \lnot ns(\ell_0)
   &\implies&
   \A(A)[\ell_0,\ell_1,ns/in,out,ls]
    = \false
}
\begin{code}
creduceUTCP d (_,PSub (_,Comp "PAtm" [pA])
                      [("in",l0),("out",l1),("ls",ns)] )
 = lcred "atm-substn" [doA,nowt]
 where
   nsl0 = atm $ subset l0 ns
   doA  = ( psimp d nsl0
          , bAnd [pA, equal ls' $ sswap ns l0 l1 ] )
   nowt = ( psimp d $ bNot nsl0
          , false )
\end{code}


Other cases, do nothing:
\begin{code}
creduceUTCP d mpr = lcred "" [(T,mpr)]

lcred nm cmprs = ( nm, cmprs )
\end{code}


\begin{code}
lawsUTCPDict :: (Ord s, Show s) => Dict s
lawsUTCPDict
 = makeDict
    [ ( "laws"
      , LawEntry [reduceUTCP] [creduceUTCP] [])
    ]
\end{code}


%%%% UTCP MAIN




\subsubsection{The UTCP Theory}
Our theory:
\begin{code}
dictUTCP :: (Eq s, Ord s, Show s) => Dict s
dictUTCP
 = foldl1 dictMrg [ makeDict [(version,AlfEntry [versionUTCP])]
                  , alfUTCPDict
                  , setUTCPDict
                  , genUTCPDict
                  , semUTCPDict
                  , lawsUTCPDict
                  ]

showUTCP (_,pr)  = pdshow 80 dictUTCP pr
\end{code}
