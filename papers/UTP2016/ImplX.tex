\section{Implementation}\label{sec:Impl}

\subsection{Before we dive in \dots}

The UTP Calculator is implemented as a series
of Haskell modules,
which are broken into two groups:
\begin{description}
  \item[Infrastructure]
    are modules that implement the calculator mechanics,
    pretty-printing, etc.
    These include \texttt{PrettyPrint},
    and all modules with names starting with \texttt{Calc}.
  \item[Builtin Theories]
    are pre-defined theory modules that cover standard logic,
    whose names start with \texttt{Std}, and modules that cover ``standard''
    UTP, whose names start with \texttt{StdUTP}.
    These theory modules typically come in threes, covering
    \texttt{Predicates}, \texttt{Precedences} and \texttt{Laws}.
\end{description}
All the Haskell modules are found in the \texttt{src} directory
of the repository, with a \texttt{.lhs} file extension
(e.g., \texttt{CalcTypes.lhs}).

\subsection{Development Process}\label{ssec:development}

A very early decision was made to adhere to Agile Software development
principles \cite{Fowl01a}
in developing this calculator
(to the extent possible given that the roles of Software Engineer, Scrum Manager
and Customer were all rolled into one).
In particular we stuck close to the YAGNI (``Ya Ain't Gonna Need It'') principle%
\footnote{More formally, ``Simplicity---the art of maximizing the amount
of work not done---is essential.''}
which requires us to only write software for a function
that is required at that time.
This does not prevent advanced design planning but does keep
the development focussed on immediate needs.
So initially the focus was on being able to use the calculator
to expand the UTP definition of an atomic action.
Once that was working, then the focus shifted to additional code to
support the calculation of the sequential composition of two atomic actions,
and so on.
For example, the feature to take a final calculation and output it to a file
was only developed when this paper was being written,
because there was no need for it until this point.

\subsection{Software Architecture}\label{ssec:architecture}

All the code described here is available online
at
\\\url{https://bitbucket.org/andrewbutterfield/utp-calculator.git}
as Literate Haskell Script files (\texttt{.lhs})
in the \texttt{src} sub-directory.

Taking into account the repetitive nature of the calculations,
as mentioned at the end of \S\ref{ssec:plan},
and the need for shorthand notations we very rapidly converged
on four initial design decisions:
\begin{enumerate}
  \item No parsers! All calculation objects are written
  directly in Haskell.
  \item We would keep the expression and predicate datatype declarations
   very simple, with only equality being singled out.
  \item We would need to have a good way to pretty-print long predicates
    that made it easy to see their overall structure
  \item We would rely on a dictionary based system to
    make it easy to customise how specific constructs
    were to be handled.
\end{enumerate}
From our experience with the \UTP2 theorem-prover we also decided
the following regarding the calculation steps that would be supported:
\begin{itemize}
  \item
    We would not support full  propositional calculus
    or theories of numbers or sets.
    Instead we would support the use of hard-coded relevant laws,
    typically derived from  a handwritten proof.
  \item
    We would avoid, at all costs,
    any use of quantifiers or binding constructs.
  \item
    The calculator user interface would be very simple,
    supporting a few high level commands such as ``simplify''
    or ``reduce''.
    In particular,
    no facility would be provided for the user to identify
    the relevant sub-part of the current goal to which any operation
    should be applied.
\end{itemize}

We will now expand on some of the points above,
and in so doing expose some of the concrete architecture of the calculator code.
