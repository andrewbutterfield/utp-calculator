\section{Building Theories}\label{sec:Theorising}

Sometimes learning by doing is the best approach,
so here we sketch out how a theory like UTCP
is captured in way that this calculator
can be usefully used to perform validating calculations.

\subsection*{Paper Plan - 20 page limit}

\begin{itemize}
  \item
     Present a development of a theory using the calculator,
     step by step.(\S\ref{sec:Theorising})
\end{itemize}

Show how the $ls(L)$ shorthand gets encoded,
and a predicate definition.

See also \texttt{CalcAlphabets.lhs}

\subsection{Coding Laws}\label{ssec:coding-laws}

Explain the \texttt{Rewrite}/\texttt{RWResult} distinction
and give an example of a \texttt{DictRWFun}.

\subsection{No Quantifiers}\label{ssec:no-quant}

    They bring a world of pain with them,
    and as it turns out, are not required
    for the kinds of calculations we wish to do.
    Handwritten proofs of laws involving concepts
    defined using quantifier and binders
    are required to validate the laws,
    but these are easy to do---the pain is automating these proofs,
    not doing them by hand.

    Substitutability

\subsection{No Targetting}\label{ssec:no-target}

First-come, first-served.

\subsection{Dictionary Libraries}

explain how dictionaries are managed and constructed,
and describe the ``builtin'' dictionaries (a.k.a. \texttt{StdXXX})
