\section{Conclusions}\label{sec:Conc}

\subsubsection{Using the Calculator}

The first property of interest for this calculator
was calculating the effect of $run(A \pseq B)$,
where $A$ and $B$ where atomic action predicates with alphabet $\setof{s,s'}$.
For convenience we predefined the predicate $A \pseq B$, as
\begin{code}
athenb = pseq [patm (pvar "A"),patm (pvar "B")]
\end{code}
We then invoke the calculator as follows, 
\begin{code}
calcREPL dictUTCP (run athenb)
\end{code}
and proceed to interact 
(here the prompt  ``\texttt{ ?,d,r,l,s,c,u,x :-}'' 
shows the available commands)
\begin{verbatim}
UTCP-0.7, UTP-Calc v0.0.1
run(A(A) ;; A(B))
 ?,d,r,l,s,c,u,x :- d
 = "defn. of run.3"
   (A(A) ;; A(B))[g::,lg,lg,lg:/g,in,ls,out]
 ; ~ls(lg:) * (A(A) ;; A(B))[g::,lg,lg:/g,in,out]
 ?,d,r,l,s,c,u,x :- d
 = "defn. of ;;"
   (A(A)[g:1,lg/g,out] \/
   A(B)[g:2,lg/g,in])[g::,lg,lg,lg:/g,in,ls,out]
 ; ~ls(lg:) * (A(A) ;; A(B))[g::,lg,lg:/g,in,out]
 ?,d,r,l,s,c,u,x :- s
 = "simplify"
   A(A)[g:::1,lg,lg,lg::/g,in,ls,out] \/
   A(B)[g:::2,lg::,lg,lg:/g,in,ls,out]
 ; ~ls(lg:) * (A(A) ;; A(B))[g::,lg,lg:/g,in,out]
.... 10 more steps
 ?,d,r,l,s,c,u,x :- r
 = "ls'-cleanup"
(A ; B) /\ ls' = {lg:}
\end{verbatim}
A text transcript is produced,
which is essentially the above
without the prompts.


\subsection{Soundness, and other nice stuff}

No such guarantees---quite possible to have a broken dictionary
or a duff law.

\subsection{Related Work}

In \cite{Bird14} we find a calculator
for point-free equational proofs as a final case-study.

Already cited several times but \cite{Gibbons:2014:FDS}
seems to be quite relevant.

Shared  vars and probability \cite{DBLP:conf/utp/ZhuSHQ12}.
This has a p-tree model with tags, with notions of probability and time.

Interference in Aspect Orientation \cite{DBLP:conf/utp/ChenYD10}.
Here we have a particular form of shared-state concurrency
that emerges because of the way that aspect-oriented languages
end up ``weaving together'' the various pieces of aspect code.
It is hard to relate this to a general concurrency view,
but it does have a rely-guarantee feel (c.f. their reference [8]).

Pomset semantics for shared-var parallel language\cite{DBLP:conf/utp/ZhaoWZ10}.
The focus here is on non-interleaving (or ``true'') concurrency.

\subsection*{Paper Plan - 20 page limit}

\begin{itemize}
  \item
    Talk about how an agile approach really is important,
    not just in writing the calculator software and Haskell models,
    but also when deciding what definitions and laws to focus on.
    Discuss finding the ``sweet spot'' between the use of definition
    expansion, and hard-coded laws based on by-hand proofs.
    Note the key observation that the calculator approach
    leads almost inexorably towards a strong emphasis
    on some kind of (combinator?) algebra. (\S\ref{sec:Conc})
\end{itemize}
