\section{Conclusions}\label{sec:Conc}

We have presented a description of a calculator written in Haskell,
that allows the encoding of an UTP theory under development,
in order to be able to rapidly perform test calculations
in order to check that predictions of the theory match expectations.
The tool was not designed to be a complete and sound theory development
system,
but instead to act as a rapid-prototype tool to help smoke out problems
with a developing theory. This approach relies on the  developer
to be checking and scrutinising everything.


As far as the development of the UTCP theory is concerned,
and the ongoing work to develop a fully composition UTP theory
of shared-state concurrency that does not require $run$,
the costs of developing and customising the calculator
have been well worth the benefits.

We note a few observations based on our experience
using the calculator:
\begin{itemize}
  \item
   First-come, first-served, just works.
   The approach of giving a blunt specification of what is to be done
   and then let the calculator do a top-down, left-to-right search
   for the first applicable location,
   has proven to be surprisingly effective.
   One reason why this has turned out to be the case,
   is that the kind of calculations that are needed for UTPC
   and related theories unfold very effectively with this strategy.
  \item
   The main idea is find a suitable collection of patterns,
   in the right order,
   to be most effective in performing calculations.
   The best way to determine this is start with none,
   run the calculator and when it stalls
   (no change is happening for any command),
   see what law would help make progress, and encode it.
   This leads to an unexpected side-effect of calculator,
   in that we learnt what laws we needed,
   rather than what we thought we would need.
  \item
    An issue that can be raised,
    given the customisation and lack of soundness guarantees,
    is how well has the calculator been tested?
    The answer is basically that the process of using it ensures
    that the whole system is comprehensively tested.
    This is because calculations fail repeatedly.
    Such failures lead to a post-mortem to identify the reason.
    Early in the calculator development,
    the reason would be traced to a bug in the calculator infrastructure.
    The next phase has failures that can be attributed to bugs in the encoding
    of laws in Haskell, or poor ordering in the dictionary.
    What makes the above tolerable is that the time taken to identify
    and fix each code problem is relatively short,
    often a matter of five to ten minutes.
    The final phase is where calculation failures arise because of errors
    the proposed theory---this is the real payback.
    In the author's experience,
    the cost of all the above failures
    is considerably outweighed by the cost of tryimg to do the check calculations
    manually.
  \item
    Effective use of the calculator results in an inexorable
    push towards algebras. By this we do not mean the Kleene algebras,
    or similar, that might characterise the language being formalised.
    Rather we mean that the most effective use of the calculator results
    when we define predicate functions that encapsulate some simple
    behaviour, and demonstrate, by proofs done without the calculator,
    some laws they obey, particularly with respect to sequential composition.
    In fact, one of the `algebras' under development for the fully
    compositional theory, is so effective, that many of the test calculations
    can be done manually
    (however some, most notably involving parallel,
    still require the calculator).
\end{itemize}


There are no guarantees of soundness.
But working on a theory by hand faces exactly the same issues
--- a proof or calculation by hand always raises the issue
of the correctness of a law, or the validity of a ``proof-step''
that is really a number of simpler steps all rolled into one.
In either case, by hand or by calculator,
the theory developer has a responsibility to carefully check every line.
This is one reason why so much effort was put into pretty-printing
and marking.
The calculator's real benefit, and\emph{ main design purpose},
is the ease with which
it can produce a calculation and transcript.


In effect, this UTP Calculator is a tool that assists
with the validation of UTP semantic definitions,
and is designed for use by someone with expertise
in UTP theory building,
and a good working knowledge of Haskell.

%A key lesson learnt during the development
%of both the calculator, and the UTCP theory,
%is the value of the agile approach.
%By focussing development of both on what
%was the immediate need at any given moment,
%we found that the calculator, and its dictionaries,
%were prevented from excessive bloat
%e.g., coding up a common, useful, obvious law,
%that actually wasn't needed.
%What was important was
%finding the ``sweet spot'' between the use of definition
%expansion, and hard-coded laws based on by-hand proofs.


We plan a formal release of this calculator as a Haskell package.
A key part of this would be comprehensive
user documentation of the key parts of the calculator API,
the standard built-in dictionaries,
as well as a complete worked example of a theory encoding.
There are many enhancements that are also being considered,
that include better transcript rendering options
(e.g. \LaTeX) or ways to customise the REPL
(e.g. always do a simplify step after any other REPL command).
Also of interest would be finding
a way of connecting the calculator
to either the \UTP2 theorem-prover\cite{DBLP:conf/utp/Butterfield10}
or the Isabelle/UPT encoding\cite{DBLP:conf/utp/FosterZW14}
in order to be able to validate the dictionary entries.

All the code described here is available online
at
\\\url{https://bitbucket.org/andrewbutterfield/utp-calculator.git}
as Literate Haskell Script files (\texttt{.lhs})
in the \texttt{src} sub-directory.
