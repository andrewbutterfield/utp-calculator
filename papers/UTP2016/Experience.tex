\section{Experience}\label{sec:Experience}

\DRAFT{Described, in math notation, not haskell, the laws
used to develop UTCP.}

\DRAFT{the inexorable push toward algebras}

\DRAFT{Notation as CRUCIAL}

They bring a world of pain with them,
    and as it turns out, are not required
    for the kinds of calculations we wish to do.
    Handwritten proofs of laws involving concepts
    defined using quantifier and binders
    are required to validate the laws,
    but these are easy to do---the pain is automating these proofs,
    not doing them by hand.

First-come, first-served, just works.
We don't provide a facility
to allow the user to specify
precisely which law to apply,
or which sub-predicate should be the target of a rewrite attempt.
Instead we allow a blunt command that simply chooses between
definition-expansion, reduction, simplification or conditional-reduction.
The calculator engine then simply looks for the first location
were the command succeeds.
To date this has had no major effect on the ability of the calculator
to work, but instead makes it very easy and fast to explore the calculation space.
