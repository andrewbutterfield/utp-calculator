\section{Related Work}\label{sec:Related}

Get related work from the UTP2 paper sources


There is very little work similar to this in the literature.
In \cite{Bird14} we find a calculator
for point-free equational proofs as a final case-study.
Also of interest is the discussion of deep/shallow embedding
in \cite{Gibbons:2014:FDS},
which suggests, that although our calculator is based on deep embedding,
the way it uses hand-coded laws from dictionaries
it more like shallow-embedding in character.



\subsection{The decision}

We briefly considered using the \UTP2 theorem prover
\cite{DBLP:conf/utp/Butterfield10,DBLP:conf/utp/Butterfield12},
but it would have required a lot of setup effort,
and it is currently not in an ideal state%
\footnote{The issue has to do with difficulties installing
the relevant software libraries
on more recent versions of Haskell.}
.
However, as part of our work on the formal semantics of PML,
we had developed a parser and some initial analysis tools
in Haskell\cite{Haskell2010},
and this software contained abstract syntax and support
for general predicates.
It became really obvious that this could be quickly adapted,
to mechanise the checking calculations, that were being performed
during each attempt.
In particular,
the key inspiration was the observation,
over all of those calculation attempts (five in all!),
that the pattern of each calculation was very uniform and similar.
So a decision was taken to construct the calculator described in this paper.
