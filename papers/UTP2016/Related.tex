\section{Related Work}\label{sec:Related}

Get related work from the UTP2 paper sources

See \cite{conf/utp/Butterfield10}.

There is very little work similar to this in the literature.
In \cite{Bird14} we find a calculator
for point-free equational proofs as a final case-study.
Also of interest is the discussion of deep/shallow embedding
in \cite{Gibbons:2014:FDS},
which suggests, that although our calculator is based on deep embedding,
the way it uses hand-coded laws from dictionaries
it more like shallow-embedding in character.



\subsection{The decision}

We briefly considered using the \UTP2 theorem prover
\cite{DBLP:conf/utp/Butterfield10,DBLP:conf/utp/Butterfield12},
but it would have required a lot of setup effort,
and it is currently not in an ideal state%
\footnote{The issue has to do with difficulties installing
the relevant software libraries
on more recent versions of Haskell.}
.
However, as part of our work on the formal semantics of PML,
we had developed a parser and some initial analysis tools
in Haskell\cite{Haskell2010},
and this software contained abstract syntax and support
for general predicates.
It became really obvious that this could be quickly adapted,
to mechanise the checking calculations, that were being performed
during each attempt.
In particular,
the key inspiration was the observation,
over all of those calculation attempts (five in all!),
that the pattern of each calculation was very uniform and similar.
So a decision was taken to construct the calculator described in this paper.

\subsection{From reviewer 3}

\begin{verbatim}
Some relevant systems are:

- LCF: tactics and tacticals much more general than the fixed bottom-up or
top-down, depth-first mentioned in the last 2 paragraphs of Page 9; see e.g.
(Paulson 1983)

- Isabelle: architecture with small logical core designed to make proofs
trustworthy; supports quantifiers, higher-order unification, automatic proof
checking and proof finding, proof replay

- Isabelle/Isar supports equational reasoning; see also other readable proof
languages such as Mizar (Wenzel/Wiedijk 2002)

- tactics in Coq and Sledgehammer in Isabelle/HOL: automation of proof steps
(i.e. finding proofs, not just checking)

- term rewriting systems, e.g. Maude: AC rewriting, AC matching are
extensively studied problems; Section 3.7 seems to try to solve a special
case of this

- program transformation systems (e.g. Unfold/Fold: Burstall/Darlington 1977;
ZAP: Feather 1982; CIP-S: Bauer et al. 1987; ASF+SDF: Bergstra et al. 1989;
KIDS: Smith, 1991; HATS: Winter, 1999; Stratego: Visser 2001) and many
systems based on unfold/fold supporting not only programs, but also
specifications involving quantifiers and binding constructs, which could be
applied to logical formulas and their interactive exploration; moreover
support of unfold (pdefn on page 8), fold, simplifications, user-defined
transformation rules (cf. law reduce on page 10)
\end{verbatim}

\subsection{From UTP2010}

There are a lot of theorem provers in existence,
of which the most prominent feature in \cite{conf/tphol/2006provers}.
Of these, the most obvious candidates for consideration for UTP prover support
are Isabelle/HOL\cite{books/sp/NipkowPW02},
PVS\cite{conf/fmcad/Shankar96},
and Coq \cite{bk:Coq'Art:04}.
They are powerful, well-supported,
with decades of development experience
and large active user communities.
They all support higher-order logic of some form, with a command-line interface,
typically based around tactics of some form. All three require functions to be total,
but support some kind of mechanism for handling partial functions
(e.g. dependent types in PVS).
Their reasoning frameworks are based on some form of sequent calculus,
and do not support equational reasoning in a native fashion.

There has been work done on improving the user interfaces
of the above theorem provers.
An interesting example was ``proof by pointing'' \cite{conf/tacs/BertotKT94}
for CoQ which allowed the user to select a subterm,
whereupon it would generate and apply a tactic based on the subterm's top-level
operator.
Whilst proof-by-pointing is not supported in more recent versions of CoQ,
it has been incorporated into ``Proof General'' \cite{conf/tacas/Aspinall00},
a general purpose user interface for theorem provers, built on top of Emacs.
It supports Isabelle and Coq, among others,
and is basically a proof-script management system.
In essence it supports the command-line tactics of the provers,
allowing the user to edit proof scripts at will,
whilst maintaining prover consistency behind the scenes.

Within the UTP community,
there has been considerable work using Proof{\-}Power-Z to
build models of UTP theories in Z in order to mechanise proofs.
Early work looked at deep embedding into Z of an imperative language whose semantics
were given using UTP \cite{conf/utp/NukaW06}.
Work extending this to a mechanisation of UTP itself was also undertaken,
driven by a desire to mechanically verify the semantics of Circus \cite{journals/fac/OliveiraCW09}.
Recent work has looked at re-working the mechanisation of UTP in order to better support
the hierarchical nature of UTP theory building \cite{journals/entcs/ZeydaC09},
where alphabetised predicates are restricted to relations, then designs, and so on.
Some support for Z-like theories from UTP (such as Circus)
can be found as extensions implemented in the Community Z Tools project
\cite{conf/zum/MalikU05}.


Whilst all of good pedigree, CoQ, Isabelle/HOL, ProofPower-Z and PVS
all have in common that they work best when used in the manner
for which they were designed%
---in none of these cases does this manner match the way
we wish to work in UTP, as described in the introduction.
Ironically, the key inspiration for the design of UTP2 came not from the
above provers, but instead from the one provided as part
of the RAISE Development Method \cite{bk:Raise-dev-mthd:95}.
That theorem prover had mechanisms for selecting sub-expressions
and identifying applicable laws for that sub-expression,
a feature very close to that required for the proof style that UTP2 supports.
