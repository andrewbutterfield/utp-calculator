\begin{abstract}
The developments of a Unifying Theory of Programming (UTP)
can involve a number of false starts,
as alphabet variables are chosen
and semantics and healthiness conditions are defined.
Typically, some calculations done just to check that everything
is fine in fact reveal problems with the theory.
So we iterate by revising the basic definitions,
and attempting the calculations again.
Hopefully, these eventually converge to what becomes a sound and useful
UTP Theory.
In a recent bout of such theory revision and re-calculation,
which required five iterations in total,
the author noted that common patterns of proof-steps kept occurring
in each iteration.
This inspired the development of the UTP-Calculator:
a tool, written in Haskell,
that supports rapid prototyping of new theories
by supporting an easy way to very quickly perform calculations.
The tool is designed for someone who is both very familiar with UTP
theory construction, and familiar enough with Haskell to be able to write
pattern-matching code. In this paper we describe how this tool can be used
to assist in theory development, by walking through how such a theory
might be encoded.
\end{abstract}
