\section{Expressions}\label{sec:Expressions}

In this section we focus on how to specify the properties and
rendering of expressions of interest.

\DRAFT{Put this somewhere useful: One big win in using a functional language like Haskell,
in which functions are first class data values,
is that we can easily define datatypes that
contain function-valued components.
We make full use of this in three of the entry kinds discussed below.
}

\subsection{UTCP Expressions}

We have sets of labels
so we need a way to implement set-expressions.
To avoid long set expressions a number of shorthands are desirable,
so that a singleton set $\setof x$ is rendered as $x$
and the very common idiom $S \subseteq ls$
is rendered as $ls(S)$,
so that for example, $ls(in)$ is short for $\setof{in} \subseteq ls$.
So we encode a set construct as follows:
\begin{code}
set = App "set"                             -- set constructor
showSet d [elm] = edshow d elm      -- drop {,} from singleton
showSet d elms = "{" ++ dlshow d "," elms ++ "}"
\end{code}
We also define an equality tester for sets,
that gets the two element-lists
\begin{code}
eqSet d es1 es2
 = let ns1 = nub $ sort $ es1                -- normalise sets
       ns2 = nub $ sort $ es2
   in if all (isGround d) (ns1++ns2)
      then Just (ns1==ns2)
      else Nothing
\end{code}
The predicate \texttt{isGround} checks to see if an expression has no
dynamic variables.
For the purposes of this theory at least,
we know we can treat these expressions as values.
This is a common feature of encoding theories for this calculator%
---%
knowing when a particular simplification makes sense.
The dictionary entry for the set construct then looks like
\begin{code}
ExprEntry ["*"] showSet none eqSet
\end{code}
where we permit any substitutions directly on the elements,
and we use the special builtin function \texttt{none}
as an evaluator that does not make any changes,
since we regard these sets as evaluated, in this theory.

Similar tricks are used to code a very compact rendering
of a mechanism that involves unique label generators
that can also be split, so that an expression like
\[
 \pi_2(new(\pi_1(new(\pi_1(split(\pi_1(new(g))))))))
\]
can be displayed as $\ell_{g:1:}$, or,
within the calculator, as \texttt{lg:1:} .


\subsubsection{Expression Entries}~

\begin{code}
Entry s
  = ExprEntry
    { ecansub :: [String]
    , eprint :: Dict s -> [Expr s] -> String
    , eval :: Dict s -> [Expr s] -> (String, Expr s)
    , isEqual :: Dict s -> [Expr s] -> [Expr s] -> Maybe Bool}
  | ...
\end{code}
Imagine that \texttt{App name args }is being processed
and that a dictionary lookup with key \texttt{name}
has returned an expression entry.
The entry contains four pieces of information:
\begin{description}
  \item[\texttt{ecansub}]
     is a list of variable names,
      over which it is safe to perform substitutions.
  \item[\texttt{eprint}]
    is a function that takes a dictionary as first argument,
    the argument-list \texttt{args} as its second argument
    and returns a string rendering of the expression.
    Currently we view expressions as atomic one-line texts
    for output purposes.
  \item[\texttt{eval}]
    takes similar arguments to \texttt{eprint},
    but returns a string/expression pair,
    that denotes a possible evaluation or simplification
    of the original expression.
    The string is empty if no change occurred,
    otherwise it is a short description of/name for the eval/simplify step.
  \item[\texttt{isEqual}]
    has a dictionary argument, followed by two sub-expression lists.
    it tests for (in)equality, returning \texttt{Just True} or \texttt{Just False}
    if it can establish (in)equality,
    and \texttt{Nothing} if unable to give a definitive answer.
\end{description}
The dictionary argument supplied to the three functions
is always the same as the dictionary in which the entry was found.

To understand the need for \texttt{ecansub},
consider the following shorthand definition for an expression:
\[
  D(L) \defs L \subseteq ls
\]
in a context where we know that $L$ is a set expression defined
only over variables $g$, $in$ and $out$.
Now, consider the following instance, with a substitution,
and two attempts to calculate a full expansion:
\[\begin{array}{l@{\qquad}l}
   D(\setof{out})[\setof{\ell_1},\ell_2/ls,out]
 & D(\setof{out})[\setof{\ell_1},\ell_2/ls,out]
\\ {} = (\setof{out} \subseteq ls)[\setof{\ell_1},\ell_2/ls,out]
 & {}= D(\setof{\ell_2})
\\ {} = \setof{\ell_2} \subseteq \setof{\ell_1}
 & {} = \setof{\ell_2} \subseteq ls
\end{array}\]
The lefthand calculation is correct, the righthand is not.
The substitution refers to variables (e.g. $ls$)
that are hidden when the $D$ shorthand is used.
The \texttt{ecansub} entry lists the variables for which substitution
is safe with the expression as-is.
With the definition above, the value of this entry
 should be \texttt{["g","in","out"]}.
Given that entry, the calculator would simplify (correctly) as follows:
\[
  D(\setof{\ell_2})[\setof{\ell_1}/ls]
\]
The righthand side of the definition is what should be returned
by the \texttt{eval} component.
If we want to state that any substitution is safe,
then we use the ``wildcard'' form: \texttt{["*"]}.
