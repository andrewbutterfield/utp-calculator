\documentclass{llncs}
\usepackage{amssymb}
\usepackage{amsmath}
\usepackage{views}
\usepackage{graphicx}
\usepackage{xcolor}
\usepackage{UTPCalc}
\usepackage{listings}
\usepackage{PML}
\usepackage{mathpartir}
\usepackage{epstopdf}
\usepackage{hyperref}
\allowdisplaybreaks[2]
\newif\ifColour
%\Colourtrue
\ifColour
\usepackage{clstlhs}
\else
\usepackage{lstlhs}
\fi
% ----------------------------------------------------------------


\newif\ifDraft
\Drafttrue  % comment out to turn off note/draft-mode
\ifDraft
  \def\NOTE#1{\textbf{Note: }\textsl{#1}}
  \def\DRAFT#1{\textbf{Draft: }\textit{#1}}
\else
  \def\NOTE#1{}
  \def\DRAFT#1{}
\fi

\title{UTPCalc --- A calculator for UTP Predicates}%
\author{
Andrew Butterfield
\thanks{%
This work was supported, in part,
by Science Foundation Ireland grant 10/CE/I1855
to Lero - the Irish Software Engineering Research Centre (www.lero.ie)}%
}
\institute{
Trinity College Dublin
}
\date{\today}%

\mathchardef\spot="320F
\mathcode`\@=\spot
\mathcode`\|=\mid


\begin{document}

\maketitle

\begin{abstract}
We present the development of the UTP-Calculator:
a tool, written in Haskell,
that supports rapid prototyping of new theories
by supporting an easy way to very quickly perform calculations.
The emphasis during the calculator development
was keeping it simple but effective,
and relying on the user to have the expertise to check its output.
It is not intended to supplant existing theorem prover or language
transformation technology.
The tool is designed for someone who is both very familiar with UTP
theory construction, and familiar enough with Haskell to be able to write
pattern-matching code. In this paper we describe how this tool can be used
to assist in theory development, by describing the key components
of the calculator and how various aspects such a theory
might be encoded.
We finish with a discussion of our experience in using the tool.
\end{abstract}


\section{Introduction}\label{sec:Intro}


\subsection*{Paper Plan}

\begin{itemize}
  \item motivate by talking about
  concurrency theory calculation travails
  \item explain decision to build a calculator
   (noting emerging tool work on PML as the seed)
   explain why UTP2 was not suitable.
  \item
     explain the key design decisions,
     and how these were motivated by agile software development
     principles (esp. YAGNI).
     Such decisions include
     \begin{itemize}
       \item dictionaries --- originally to support tailored rendering
       \item taking pretty-printing seriously
       \item relying on simple hand-proved (algebraic!) laws
       \item the avoidance of quantifiers,
        and how this forces us to deal with substitutability
     \end{itemize}
  \item
     Present a development of a theory using the calculator,
     step by step.
  \item
    Talk about how an agile approach really is important,
    not just in writing the calculator software and haskell models,
    but also when deciding what definitions and laws to focus on.
    Discuss finding the ``sweet spot'' between the use definition
    expansion, and hard-coded laws based on by-hand proofs.
    Note the key observation that the calculator approach
    leads almost inexorably towards a strong emphasis
    on some kind of (combinator?) algebra.
  \item
    Discuss future plans: formal release as a Haskell library;
    REPL customization; rendering options: \LaTeX, UTP2, Isabelle/UTP.
\end{itemize}

\section{Design \& Architecture}\label{sec:Design}

\subsection{Development Process}\label{ssec:development}

A very early decision was made to adhere to Agile Software development
principles \cite{Fowl01a}
in developing this calculator
(to the extent possible given that the roles of Software Engineer, Scrum Manager
and Customer were all rolled into one).
In particular we stuck close to the YAGNI (``Ya Ain't Gonna Need It'') principle%
\footnote{More formally, ``Simplicity---the art of maximizing the amount
of work not done---is essential.''}
which requires us to only write software for a function
that is required at that time.
This does not prevent advanced design planning but does keep
the development focussed on immediate needs.
So initially the focus was on being able to use the calculator
to expand the UTP definition of an atomic action.
Once that was working, then the focus shifted to additional code to
support the calculation of the sequential composition of two atomic actions,
and so on.
For example, the feature to take a final calculation and output it to a file
was only developed when this paper was being written,
because there was no need for it until this point.

\subsection{Software Architecture}\label{ssec:architecture}

All the code described here is available online as a git repo
at
\\\url{https://andrewbutterfield@bitbucket.org/andrewbutterfield/utp-calculator.git}
as Literate Haskell Script%
\footnote{
Note, this paper is not itself a literate Haskell script
(you'll be relieved to know).
}
(\texttt{.lhs}) files in the \texttt{src} sub-directory.

Taking into account the repetitive nature of the calculations,
as mentioned at the end of \S\ref{ssec:plan},
and the need for shorthand notations we very rapidly converged
on four initial design decisions:
\begin{enumerate}
  \item No parsers! All calculation objects are written
  directly in Haskell.
  \item We would keep the expression and predicate datatype declarations
   very simple, with only equality being singled out.
  \item We would need to have a good way to pretty-print long predicates
    that made it easy to see their overall structure
  \item We would reply on a dictionary based system to enable default

\end{enumerate}
From our experience with the \UTP2 theorem-prover we also decided
the following regarding the calculation steps that would be supported:
\begin{itemize}
  \item
    We would not support full  propositional calculus
    or theories of numbers or sets.
    Instead we would support the use of hard-coded relevant laws,
    typically derived from  a handwritten proof.
  \item
    We would avoid, at all costs,
    any use of quantifiers or binding constructs.
  \item
    The calculator use interface would be very simple,
    supporting a few high level commands such as ``simplify''
    or ``reduce''.
    In particular,
    no facility would be provided for the user to identify
    the relevant sub-part of the current goal to which any operation
    should be applied.
\end{itemize}

We will now expand on some of the points above,
and in so doing expose some of the concrete architecture of the calculator code.

\subsection{Expressions and Predicates}\label{ssec:expr-pred}

In Fig. \ref{fig:expr-pred-types} we show the Haskell declarations
of the datatypes to represent both expressions and predicates.
Of course this means we are developing a deep embedding\cite{Gibbons:2014:FDS},
which is required in order to be able to do the kind of calculations we require.
\begin{figure}[tb]
\begin{verbatim}
data Expr s
  = St s | B Bool | Z Int | Var String
  | App String [Expr s] | Sub (Expr s) (Substn s) | Undef

data Pred s
  = T | F | PVar String | Equal (Expr s) (Expr s) | Atm (Expr s)
  | Comp String [MPred s] | PSub (MPred s) (Substn s) | PUndef

type MPred s = (Marks, Pred s)
\end{verbatim}
  \caption{Expression and Predicate Datatypes (\texttt{CalcTypes.lhs})}
  \label{fig:expr-pred-types}
\end{figure}
Both datatypes are parameterised on a generic state type \texttt{s},
which allows us to be able to handle different concrete types
of shared state with one piece of code.
Both types have basic values (\texttt{St},\texttt{B},\texttt{Z},\texttt{T},\texttt{F}),
variables (\texttt{Var},\texttt{PVar}),
generic composites (\texttt{App},\texttt{Comp}),
substitution (\texttt{Sub},\texttt{PSub})
and undefined values (\texttt{Undef},\texttt{PUndef}).
The predicate datatype has a way to embed a (boolean-valued)
expression to form an atomic predicate (\texttt{Atm}),
as well as an explicit form for atomic predicates that take
the form of an equality (\texttt{Equal}).
We give a specialised form for equality simply because
its use in laws is widespread and worth optimising.

Also important to point out is the fact that \texttt{Pred} is
defined in terms of \texttt{MPred}, which in turn is defined
in terms of \texttt{Pred}.
This is done to facilitate the association of markings (lists of integers)
with predicate constructs.
These markings are used to indicate which parts of a predicate
were changed at each calculation step.


\subsection{Pretty-Printing}\label{ssec:pp}

For the calculator output,
it is very important that it be readable,
as many of the predicates get very large,
particularly at intermediate points of the calculation.
For this reason, a lot of effort was put into the development
of both good pretty-printing,
and ways to highlight old and new parts of predicates as changes are made.
The key principle was to ensure that whenever a predicate
had to split over multiple lines,
that the breaks are always around the top-most operator or composition
symbol, with sub-components indented in.
An example of such pretty-printing in action is
{\small
\begin{verbatim}
    D(out)
 \/ (      ~ls(out)
        /\ (D(lg) \/ A(in,lg,a,in,lg,lg) \/ D(out) \/ A(lg,out,b,lg,out,out))
     ; W(D(lg) \/ A(in,lg,a,in,lg,lg) \/ D(out) \/ A(lg,out,b,lg,out,out)))
\end{verbatim}
}
The top-level structure of this is
\[D(out) \lor ( (\lnot ls(out) \land \dots) ; \W(\dots) )\]

The pretty printing support can be found in \texttt{PrettyPrint.lhs}.

\subsection{Dictionaries}\label{ssec:dict}

The abstract syntax is very simple,
and uses names as Haskell Strings to distinguish between
different composites.
So for example logical operators $\land$ and $\lor$
might be represented as \texttt{Comp "And"} and \texttt{Comp "Or"}
respectively.
We have a default way to pretty-print composites,
to that the predicate $\true \land \false$,
represented in Haskell by \texttt{Comp "And" [bT,bF]},
would render as \texttt{And(true,false)}.
However it would be nice to be able to render it as the usual
infix operator, giving \texttt{true /\BS\ false}.
In addition, we need some way to associate laws and definitions,
as appropriate, with composites.

The approach taken is to implement a dictionary that maps names
to entries that supply extra information.
The names can be those of expression or predicate composites,
or correspond to variables, and a few other features of note.
All of the main calculator functions are driven by this
dictionary,
and the correct definition of dictionary entries
is the primary way for users to set up calculations.
There are four kinds of dictionary entries,
one each for expressions and predicates,
one for laws of various kinds,
and one for alphabet handling.
We will discuss each kind in turn.

\subsubsection{Expression Entries}

\begin{verbatim}
Entry s
  = ExprEntry { ecansub :: [String]
              , eprint :: Dict s -> [Expr s] -> String
              , eval :: Dict s -> [Expr s] -> (String, Expr s)
              , isEqual :: Dict s -> [Expr s] -> [Expr s] -> Maybe Bool }
  | ...
\end{verbatim}

\subsubsection{Predicate Entries}

\begin{verbatim}
Entry s
  = ...
  | PredEntry { pcansub :: [String]
              , pprint :: Dict s -> MarkStyle -> Int -> [MPred s] -> PP
              , alfa :: [String]
              , pdefn :: Rewrite s
              , prsimp :: Rewrite s }
type Rewrite s = Dict s -> [MPred s] -> (String, Pred s)
\end{verbatim}


\subsubsection{Law Entries}

\begin{verbatim}
Entry s
  = ...
  | LawEntry  { reduce :: [DictRWFun s]
              , creduce :: [CDictRWFun s]
              , unroll :: [String -> DictRWFun s] }
type DictRWFun s = Dict s -> RWFun sdata
type RWFun s = MPred s -> RWResult s
type RWResult s = (String, MPred s)
\end{verbatim}


\subsubsection{Alphabet Entries}

\begin{verbatim}
Entry s
  = ...
  | AlfEntry  { avars :: [String]}
\end{verbatim}


One big win in using a functional language like Haskell,
in which functions are first class data values,
is that we can easily define datatypes that
contain function-valued components.

Also point out how this is a form of shallow-embedding.
So we can get the best (?) of both worlds\cite{Gibbons:2014:FDS}.

Show how the $ls(L)$ shorthand gets encoded,
and a predicate definition.

See also \texttt{CalcAlphabets.lhs}

\subsection{Coding Laws}\label{ssec:coding-laws}

Explain the \texttt{Rewrite}/\texttt{RWResult} distinction
and give an example of a \texttt{DictRWFun}.

\subsection{No Quantifiers}\label{ssec:no-quant}

    They bring a world of pain with them,
    and as it turns out, are not required
    for the kinds of calculations we wish to do.
    Handwritten proofs of laws involving concepts
    defined using quantifier and binders
    are required to validate the laws,
    but these are easy to do---the pain is automating these proofs,
    not doing them by hand.

    Substitutability

\subsection{No Targetting}\label{ssec:no-target}

First-come, first-served.

\subsection{Dictionary Libraries}

explain how dictionaries are managed and constructed,
and describe the ``builtin'' dictionaries (a.k.a. \texttt{StdXXX})

\section{Building Theories}\label{sec:Theorising}

\subsection*{Paper Plan - 20 page limit}

\begin{itemize}
  \item
     Present a development of a theory using the calculator,
     step by step.(\S\ref{sec:Theorising})
\end{itemize}

\section{Implementation}\label{sec:Impl}

\subsection{Before we dive in \dots}

The UTP Calculator is implemented as a series
of Haskell modules,
which are broken into two groups:
\begin{description}
  \item[Infrastructure]
    are modules that implement the calculator mechanics,
    pretty-printing, etc.
    These include \texttt{PrettyPrint},
    and all modules with names starting with \texttt{Calc}.
  \item[Builtin Theories]
    are pre-defined theory modules that cover standard logic,
    whose names start with \texttt{Std}, and modules that cover ``standard''
    UTP, whose names start with \texttt{StdUTP}.
    These theory modules typically come in threes, covering
    \texttt{Predicates}, \texttt{Precedences} and \texttt{Laws}.
\end{description}
All the Haskell modules are found in the \texttt{src} directory
of the repository, with a \texttt{.lhs} file extension
(e.g., \texttt{CalcTypes.lhs}).

\subsection{Development Process}\label{ssec:development}

A very early decision was made to adhere to Agile Software development
principles \cite{Fowl01a}
in developing this calculator
(to the extent possible given that the roles of Software Engineer, Scrum Manager
and Customer were all rolled into one).
In particular we stuck close to the YAGNI (``Ya Ain't Gonna Need It'') principle%
\footnote{More formally, ``Simplicity---the art of maximizing the amount
of work not done---is essential.''}
which requires us to only write software for a function
that is required at that time.
This does not prevent advanced design planning but does keep
the development focussed on immediate needs.
So initially the focus was on being able to use the calculator
to expand the UTP definition of an atomic action.
Once that was working, then the focus shifted to additional code to
support the calculation of the sequential composition of two atomic actions,
and so on.
For example, the feature to take a final calculation and output it to a file
was only developed when this paper was being written,
because there was no need for it until this point.

\subsection{Software Architecture}\label{ssec:architecture}

All the code described here is available online
at
\\\url{https://bitbucket.org/andrewbutterfield/utp-calculator.git}
as Literate Haskell Script files (\texttt{.lhs})
in the \texttt{src} sub-directory.

Taking into account the repetitive nature of the calculations,
as mentioned at the end of \S\ref{ssec:plan},
and the need for shorthand notations we very rapidly converged
on four initial design decisions:
\begin{enumerate}
  \item No parsers! All calculation objects are written
  directly in Haskell.
  \item We would keep the expression and predicate datatype declarations
   very simple, with only equality being singled out.
  \item We would need to have a good way to pretty-print long predicates
    that made it easy to see their overall structure
  \item We would rely on a dictionary based system to
    make it easy to customise how specific constructs
    were to be handled.
\end{enumerate}
From our experience with the \UTP2 theorem-prover we also decided
the following regarding the calculation steps that would be supported:
\begin{itemize}
  \item
    We would not support full  propositional calculus
    or theories of numbers or sets.
    Instead we would support the use of hard-coded relevant laws,
    typically derived from  a handwritten proof.
  \item
    We would avoid, at all costs,
    any use of quantifiers or binding constructs.
  \item
    The calculator user interface would be very simple,
    supporting a few high level commands such as ``simplify''
    or ``reduce''.
    In particular,
    no facility would be provided for the user to identify
    the relevant sub-part of the current goal to which any operation
    should be applied.
\end{itemize}

We will now expand on some of the points above,
and in so doing expose some of the concrete architecture of the calculator code.

\section{Conclusions}\label{sec:Conc}

\subsubsection{Using the Calculator}

The first property of interest for this calculator
was calculating the effect of $run(A \pseq B)$,
where $A$ and $B$ where atomic action predicates with alphabet $\setof{s,s'}$.
For convenience we predefined the predicate $A \pseq B$, as
\begin{code}
athenb = pseq [patm (pvar "A"),patm (pvar "B")]
\end{code}
We then invoke the calculator as follows, 
\begin{code}
calcREPL dictUTCP (run athenb)
\end{code}
and proceed to interact 
(here the prompt  ``\texttt{ ?,d,r,l,s,c,u,x :-}'' 
shows the available commands)
\begin{verbatim}
UTCP-0.7, UTP-Calc v0.0.1
run(A(A) ;; A(B))
 ?,d,r,l,s,c,u,x :- d
 = "defn. of run.3"
   (A(A) ;; A(B))[g::,lg,lg,lg:/g,in,ls,out]
 ; ~ls(lg:) * (A(A) ;; A(B))[g::,lg,lg:/g,in,out]
 ?,d,r,l,s,c,u,x :- d
 = "defn. of ;;"
   (A(A)[g:1,lg/g,out] \/
   A(B)[g:2,lg/g,in])[g::,lg,lg,lg:/g,in,ls,out]
 ; ~ls(lg:) * (A(A) ;; A(B))[g::,lg,lg:/g,in,out]
 ?,d,r,l,s,c,u,x :- s
 = "simplify"
   A(A)[g:::1,lg,lg,lg::/g,in,ls,out] \/
   A(B)[g:::2,lg::,lg,lg:/g,in,ls,out]
 ; ~ls(lg:) * (A(A) ;; A(B))[g::,lg,lg:/g,in,out]
.... 10 more steps
 ?,d,r,l,s,c,u,x :- r
 = "ls'-cleanup"
(A ; B) /\ ls' = {lg:}
\end{verbatim}
A text transcript is produced,
which is essentially the above
without the prompts.


\subsection{Soundness, and other nice stuff}

No such guarantees---quite possible to have a broken dictionary
or a duff law.

\subsection{Related Work}

In \cite{Bird14} we find a calculator
for point-free equational proofs as a final case-study.

Already cited several times but \cite{Gibbons:2014:FDS}
seems to be quite relevant.

Shared  vars and probability \cite{DBLP:conf/utp/ZhuSHQ12}.
This has a p-tree model with tags, with notions of probability and time.

Interference in Aspect Orientation \cite{DBLP:conf/utp/ChenYD10}.
Here we have a particular form of shared-state concurrency
that emerges because of the way that aspect-oriented languages
end up ``weaving together'' the various pieces of aspect code.
It is hard to relate this to a general concurrency view,
but it does have a rely-guarantee feel (c.f. their reference [8]).

Pomset semantics for shared-var parallel language\cite{DBLP:conf/utp/ZhaoWZ10}.
The focus here is on non-interleaving (or ``true'') concurrency.

\subsection*{Paper Plan - 20 page limit}

\begin{itemize}
  \item
    Talk about how an agile approach really is important,
    not just in writing the calculator software and Haskell models,
    but also when deciding what definitions and laws to focus on.
    Discuss finding the ``sweet spot'' between the use of definition
    expansion, and hard-coded laws based on by-hand proofs.
    Note the key observation that the calculator approach
    leads almost inexorably towards a strong emphasis
    on some kind of (combinator?) algebra. (\S\ref{sec:Conc})
\end{itemize}


%\subsection{Before we dive in \dots}

The UTP Calculator is implemented as a series
of Haskell modules,
which are broken into two groups:
\begin{description}
  \item[Infrastructure]
    are modules that implement the calculator mechanics,
    pretty-printing, etc.
    These include \texttt{PrettyPrint},
    and all modules with names starting with \texttt{Calc}.
  \item[Builtin Theories]
    are pre-defined theory modules that cover standard logic,
    whose names start with \texttt{Std}, and modules that cover ``standard''
    UTP, whose names start with \texttt{StdUTP}.
    These theory modules typically come in three, covering
    \texttt{Predicates}, \texttt{Precedences} and \texttt{Laws}.
\end{description}
All the Haskell modules are found in the \texttt{src} directory
of the repository, with a \texttt{.lhs} file extension
(e.g., \texttt{CalcTypes.lhs}).

\subsection{UTCP Syntax}

We start by defining the syntax of our language
\RLEQNS{
   p,q \in UTCP
   &::=& Idle
  \mid  \A(A)
  \mid  p \lseq q
  \mid  p \lcond c q
  \mid  p \parallel q
  \mid  c \wdo p
}
\noindent
and assign them pretty printing precedences,
so they work well with the definitions in modules
\texttt{StdPrecedences} and \texttt{StdUTPPrecedences}.
\begin{code}
precPCond = 5 + precSpacer  1
precPPar  = 5 + precSpacer  2
precPSeq  = 5 + precSpacer  3
precPIter = 5 + precSpacer  6
\end{code}


\subsection{UTCP Alphabet}

As already stated, the theory alphabet is $s,s',ls,ls',g,in,out$.
We declare each as a variable in our expression notation,
noting that Haskell allows identifiers to contain dashes,
which proves very convenient:
\begin{code}
s' = Var "s'"
\end{code}
Note, here \texttt{s'} is a Haskell variable of type \texttt{Expr},
while \texttt{"s'"} is a Haskell literal value of type \texttt{String}.

We have two ways to classify UTP observation variables.
Along one axis we distinguish observations of program variable
values (``script'' variables, e.g. $s$, $s'$) from those that record other
observations such as termination/stability,
or traces/refusals (``model'' variables, e.g. $ls$, $ls'$).
On the other axis we distinguish observations
that are dynamic, whose values change as the program runs
(e.g. $s$, $ls$ with $s'$ and $ls'$)
from those that are static,
unchanged during program execution (e.g. $g$, $in$ and $out$).
We have pre-defined names for these categories,
and an function \texttt{stdAlfDictGen} that
builds all the appropriate entries
given three lists of script, dynamic model and static variable strings.
We also declare that the predicate variables $A$, $B$ and $C$
will refer to atomic state-changes,
and so only have alphabet $\setof{s,s'}$.
\begin{code}
alfUTCPDict
 = dictMrg dictAlpha dictAtomic
 where
   dictAlpha = stdAlfDictGen ["s"] ["ls"] ["g","in","out"]
   dictAtomic = makeDict [ pvarEntry "A" ss'
                         , pvarEntry "B" ss'
                         , pvarEntry "C" ss' ]
   ss' = ["s", "s'"]
\end{code}
(See modules
\texttt{CalcAlphabets}
, \texttt{CalcPredicates}
, \texttt{StdPredicates}.)

\subsection{UTPC Expressions}

We have sets of labels
so we need a way to implement set-expressions.
To avoid long set expressions a number of shorthands are desirable,
so that a singleton set $\setof x$ is rendered as $x$
and the very common idiom $S \subseteq ls$
is rendered as $ls(S)$,
so that for example, $ls(in)$ is short for $\setof{in} \subseteq ls$.
So we might encode a set construct as follows
\begin{code}
set = App "set"                             -- set constructor
showSet d [elm] = edshow d elm      -- drop {,} from singleton
showSet d elms = "{" ++ dlshow d "," elms ++ "}"
\end{code}
We also define an equality tester for sets,
that gets the two element-lists
\begin{code}
eqSet d es1 es2
 = let ns1 = nub $ sort $ es1                -- normalise sets
       ns2 = nub $ sort $ es2
   in if all (isGround d) (ns1++ns2)
      then Just (ns1==ns2)
      else Nothing
\end{code}
The predicate \texttt{isGround} checks to see if an expression has no
dynamic variables.
For the purposes of this theory at least,
we know we can treat these expressions as values.
This is a common feature of encoding theories for this calculator%
---%
knowing when a particular simplification makes sense.
The dictionary entry for the set construct then looks like
\begin{code}
ExprEntry ["*"] showSet none eqSet
\end{code}
where we permit any substitutions directly on the elements,
and we use the special builtin function \texttt{none}
as an evaluator that does not make any changes,
since we regard these sets as evaluated, in this theory.

Similar tricks are used to code a very compact rendering
of a mechanism that involves unique label generators
that can also be split, so that an expression like
\[
 \pi_2(new(\pi_1(new(\pi_1(split(\pi_1(new(g))))))))
\]
can be displayed as $\ell_{g:1:}$, or,
within the calculator, as \texttt{lg:1:} .


\subsection{Coding Atomic Semantics}

Formally, using our shorthand notations, define atomic behaviour as:
\RLEQNS{
    \A(A) &\defs& ls(in) \land A \land ls'=ls\ominus(in,out)
}
where $A$ is a predicate whose alphabet is restricted to $s$ and $s'$.
We code this up as follows:
\begin{code}
patm mpr = comp "A" [mpr] -- we assume mpr has only s, s' free

defnAtomic d [a] = ldefn shPAtm $ mkAnd [lsin,a,ls'eqlsinout]

inp = Var "in" -- 'in' is a Haskell keyword
out = Var "out"
lsin = atm $ App "subset" [inp,ls]
lsinout = App "sswap" [ls,inp,out]
ls'eqlsinout = equal ls' lsinout

patmEntry
 = ( nPAtm
   , PredEntry [] ppPAtm [] defnAtomic (pNoChg nPAtm) )
\end{code}
Here \texttt{atm} lifts an expression to a marked predicate,
while \texttt{"sswap"} names the ternary operation $\_\ominus(\_,\_)$,
and \texttt{equal} is the marked form of \texttt{Equal}.


We won't show the encoding of the composite constructs,
or a predicate transformer called $run$ that actually
enables us to symbolically `execute' our semantics.
We will show how the \texttt{pprint} entry for
sequential composition in UTCP is defined,
just to show how easy support for infix notation is.
\begin{code}
ppPSeq d ms p [mpr1,mpr2]
 = paren p precPSeq -- parenthesise if precedence requires it
     $ ppopen  (pad ";;") [ mshowp d ms precPSeq mpr1
                          , mshowp d ms precPSeq mpr2 ]
\end{code}
Here \texttt{pad} puts spaces around its argument,
while \texttt{ppopen} uses its first argument as a
separator between all the elements of its second list argument.
The function \texttt{mshowp} is the top-level predicate printer.


\subsection{Coding UTCP Laws}

The definition of the semantics of the UTCP language
constructs, and of $run$,
make use of the (almost) standard notions of skip,
sequential composition
and iteration in UTP.
The versions used here are slightly non-standard because we have
non-homogeneous relations,
where the static parameters have no dashed counterparts.
In essence we ignore the parameters as far as flow-of-control is concerned:
\RLEQNS{
   \Skip &\defs& s'=s \land ls'=ls
\\ P ; Q
   &\defs&
   \exists s_m,ls_m @
     P[s_m,ls_m/s',ls']
     \land
     Q[s_m,ls_m/s,ls]
\\ c * P &\defs& \mu L @ (P ; L) \cond c \Skip
\\ P \cond c Q &\defs& c \land P \lor \lnot c \land Q
}
Here, the definition of $\cond\_$ is entirely standard, of course.

What is key here though,
is realising that we do not want to define the constructs
as above and use them directly, as it involves
quantifiers and explicit recursion,
both of which would introduce considerable complexity to the calculator.
Instead, we encode useful laws that they satisfy,
that do not require their definitions to be expanded.
Such laws might include the following:
\RLEQNS{
  \Skip \seq\ P & {} = P = {} & P \seq \Skip
\\ c * P &=& (P \seq c* P ) \cond c \Skip
\\ (c * P)[e/x] &=& P[e/x] \seq c * P, \qquad if c[e/x]
\\ (c * P)[e/x] &=& \Skip[e/x] , \qquad if \lnot c[e/x]
}
These laws need to be proven by hand (carefully),
by the theory developer, and then encoded into Haskell
(equally carefully), as we are about to describe.

We can easily give a definition of $\Skip$,
which is worth expanding.
\RLEQNS{
   \Skip &\defs& s'=s \land ls'=ls
}
\begin{code}
defnUTCPII = mkAnd[ equal s' s, equal ls' ls ]
\end{code}

For more complex laws,
the idea is that we pattern-match on predicate syntax
to see if a law is applicable (we have its lefthands-side),
and if so,
we then build an appropriate instance of the righthand-side.
The plan is that we gather all these pattern/outcome pairs
in one function definition,
which will try them in order.
This is a direct match for how Haskell pattern-matching works.
So for UTCP we have a function called reduceUTCP,
structured as follows:
\begin{code}
reduceUTCP :: (Show s, Ord s) => DictRWFun s
reduceUTCP (...1st law pattern...) = 1st outcome
reduceUTCP (...2nd law pattern...) = 2nd outcome
...
reduceUTCP d mpr = lred "" mpr -- catch-all at end, no change
\end{code}
The last clause matches any predicate
and simply returns it with a null string,
indicating no change took place.
The main idea is find a suitable collection of patterns,
in the right order,
to be most effective in performing calculations.
The best way to determine this is start with none,
run the calculator and when it stalls
(no change is happening for any command),
see what law would help make progress, and encode it.

A simple example of such a pattern is the following encoding
of $\Skip;P = P$ :
\begin{code}
reduceUTCP d   -- the dictionary is 'd'
  (_, Comp "Seq" [(_,Comp "Skip" []), mpr]) = lred ";-lunit" mpr
\end{code}
The second argument has type marked-predicate (\texttt{MPred})
which is a marking/predicate pair.
We are not interested in the markings
so we use the wildcard pattern '\verb"_"'
for the first pair component.
The sub-pattern in the second pair component,
\verb'Comp "Seq" [(_,Comp "Skip" []), mpr])',
matches a composite called ``Seq'',
with a argument list containing two (marked) predicates.
The first argument predicate pattern \verb'(_,Comp "Skip" [])'
matches a ``Skip'' composite with no further subarguments.
The second argument pattern \verb'mpr' matches an arbitrary predicate
($P$ in the law above).
The righthand-side returns the application \verb'lred ";-lunit" mpr'
which simply constructs a string/predicate pair,
with the strign being a justification note that says a reduction-step
usimng a laws called ``;-lunit'' was applied.



\subsubsection{UTCP Recognisers}

Some laws require matching that is a bit more sophisticated,
often with side-conditions.

$s'=s$, or $s=s'$
\begin{code}
isIdle s1 s2 = s1=="s" && s2=="s'" || s1=="s'" && s2=="s"
\end{code}
$s'=s$ conjoined with $A$ whose alphabet is $\setof{s,s'}$.
\begin{code}
isIdleSeqAtom d s1 s2 pA
 | isIdle s1 s2
    = case plookup pA d of
       Just (PredEntry _ _ a_alf _ _)  ->  sort a_alf == ["s","s'"]
       _                               ->  False
 | otherwise  =  False
\end{code}




\subsubsection{Skip and Sequential Composition}


A law that tidies up a common situation towards
the end of calculations.
\RLEQNS{
   s' = e \land ls' = f ; A
   &=&
   A[e,f/s,ls]
   & \elabel{$s'$-$ls'$-$;$-prop}
}
\begin{code}
reduceUTCP d (_,Comp "Seq"
                [ (_,Comp "And" [ (_,Equal (Var "s'") e)
                                , (_,Equal (Var "ls'") f) ])
                , mpA ])
 = lred "s'ls'-;-prop" $ psub mpA [("s",e),("ls",f)]
reduceUTCP d (_,Comp "Seq"
                [ (_,Comp "And" [ (_,Equal (Var ls'@"ls'") f)
                                , (_,Equal (Var s'@"s'") e) ])
                , mpA])
 = lred "s'ls'-;-prop" $ psub mpA [("s",e),("ls",f)]
\end{code}

A useful reduction for tidying up at the end,
assuming that $ls' \notin A$ and $ls \notin B$, and both $k$ and $h$
are ground:
\RLEQNS{
   A \land ls'=k ; B \land ls'= h
   &\equiv&
   (A;B) \land ls'=h
   & \elabel{$ls'$-cleanup}
}
\begin{code}
reduceUTCP d pr@(_,Comp "Seq" [ (_,Comp "And" pAs)
                              , (_,Comp "And" pBs)])
 = case isSafeLSDash d ls' ls' pAs of
    Nothing -> lred "" pr
    Just (_,restA) ->
     case isSafeLSDash d ls' ls pBs of
      Nothing -> lred "" pr
      Just (eqB,restB)
       -> lred "ls'-cleanup" $
             comp "And" [ comp "Seq" [ bAnd restA
                                     , bAnd restB ]
                        , eqB ]
 where
   ls = "ls"
   ls' = "ls'"

   isSafeLSDash d theLS unwanted prs
    = case matchRecog (mtchNmdObsEqToConst theLS d) prs of
       Nothing -> Nothing
       Just (pre,(eq@(_,Equal _ k),_),post) ->
        if notGround d k
         then Nothing
         else if all (dftlyNotInP d unwanted) rest
          then Just (eq,rest)
          else Nothing
        where rest = pre++post
\end{code}


That's all folks!


%%%%  CREDUCE


To avoid having to support a wide range of expression-related theories,
we provide a conditional reducer, that computes
a number of alternative outcomes, each guarded by some predicate
that is hard to evaluate.
The user elects which one to use by checking the conditions manually.

\begin{code}
creduceUTCP :: (Show s, Ord s) => CDictRWFun s
\end{code}

\paragraph{pre- and before-substitutions}
A pre-substitution is one that replaces undashed variables with
undashed expressions, while a before-substitution is further restricted
to replacing undashed observables only.
\begin{code}
preSublet :: Ord s => ( String, Expr s ) -> Bool
preSublet (v,e) = notDash v && notDashed e

preSub :: Ord s => Substn s -> Bool
preSub = all preSublet

beforeSublet :: Ord s => Dict s -> ( String, Expr s ) -> Bool
beforeSublet d (v,e) = isDyn d v && notDashed e

beforeSub :: Ord s => Dict s -> Substn s -> Bool
beforeSub d = all (beforeSublet d)
\end{code}

\paragraph{Predicate Simplifier}
 Sometimes we want to simplify a predicate without fuss
(marking or comment):
\begin{code}
psimp :: (Ord s, Show s)
      => Dict s -> MPred s -> Pred s
psimp d = snd . thd . simplify d startm

thd (_,_,z) = z
\end{code}


\paragraph{Atomic Enablement}

\RLEQNS{
   ns(\ell_0)
   &\implies&
   \A(A)[\ell_0,\ell_1,ns/in,out,ls]
    = A \land ls'=ns\ominus(\ell_0,\ell_1)
\\ \lnot ns(\ell_0)
   &\implies&
   \A(A)[\ell_0,\ell_1,ns/in,out,ls]
    = \false
}
\begin{code}
creduceUTCP d (_,PSub (_,Comp "PAtm" [pA])
                      [("in",l0),("out",l1),("ls",ns)] )
 = lcred "atm-substn" [doA,nowt]
 where
   nsl0 = atm $ subset l0 ns
   doA  = ( psimp d nsl0
          , bAnd [pA, equal ls' $ sswap ns l0 l1 ] )
   nowt = ( psimp d $ bNot nsl0
          , false )
\end{code}


Other cases, do nothing:
\begin{code}
creduceUTCP d mpr = lcred "" [(T,mpr)]

lcred nm cmprs = ( nm, cmprs )
\end{code}


\begin{code}
lawsUTCPDict :: (Ord s, Show s) => Dict s
lawsUTCPDict
 = makeDict
    [ ( "laws"
      , LawEntry [reduceUTCP] [creduceUTCP] [])
    ]
\end{code}


%%%% UTCP MAIN




\subsubsection{The UTCP Theory}
Our theory:
\begin{code}
dictUTCP :: (Eq s, Ord s, Show s) => Dict s
dictUTCP
 = foldl1 dictMrg [ makeDict [(version,AlfEntry [versionUTCP])]
                  , alfUTCPDict
                  , setUTCPDict
                  , genUTCPDict
                  , semUTCPDict
                  , lawsUTCPDict
                  ]

showUTCP (_,pr)  = pdshow 80 dictUTCP pr
\end{code}


\bibliographystyle{splncs03}
\bibliography{UTPCalc-UTP2016-MAIN}

\end{document}
