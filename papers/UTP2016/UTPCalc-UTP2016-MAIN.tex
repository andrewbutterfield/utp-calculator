\documentclass{llncs}
\usepackage{amssymb}
\usepackage{amsmath}
\usepackage{views}
\usepackage{graphicx}
\usepackage{xcolor}
\usepackage{UTPCalc}
\usepackage{listings}
\usepackage{PML}
\usepackage{mathpartir}
\usepackage{epstopdf}
\usepackage{hyperref}
\usepackage{listings}
\allowdisplaybreaks[2]
\newif\ifColour
%\Colourtrue
\ifColour
\usepackage{clstlhs}
\else
\usepackage{lstlhs}
\fi
% ----------------------------------------------------------------


\newif\ifDraft
\Drafttrue  % comment out to turn off note/draft-mode
\ifDraft
  \def\NOTE#1{\textbf{Note: }\textsl{#1}}
  \def\DRAFT#1{\textbf{[Draft: }\textsc{#1}\textbf{]}}
\else
  \def\NOTE#1{}
  \def\DRAFT#1{}
\fi

\title{UTPCalc --- A calculator for UTP Predicates}%
\author{
Andrew Butterfield
\thanks{%
This work was supported, in part,
by Science Foundation Ireland grant 10/CE/I1855
to Lero - the Irish Software Engineering Research Centre (www.lero.ie)}%
}
\institute{
Trinity College Dublin
}
\date{\today}%

\mathchardef\spot="320F
\mathcode`\@=\spot
\mathcode`\|=\mid


\begin{document}

\maketitle

\begin{abstract}
The developments of a UTP Theory can involve a number of false starts,
as alphabet variables are chosen
and semantics and healthiness conditions are defined.
Typically, some calculations done just to check that everything
is fine in fact reveal problems with the theory.
So we iterate by revising the basic definitions,
and attempting the calculations again.
Hopefully, these eventually converge to what becomes a sound and useful
UTP Theory.
In a recent bout of such theory revision and re-calculation,
which required five iterations in total,
the author noted that common patterns of proof-steps kept occurring
in each iteration.
This inspired the development of the UTP-Calculator:
a tool, written in Haskell,
that supports rapid prototyping of new theories
by supporting an easy way to very quickly perform calculations.
The tool is designed for someone who is both very familiar with UTP
Theory construction, and familiar enough with Haskell to be able to write
pattern-matching code. In this paper we describe how this tool can be used
to assist in theory development, by walking through how such a theory
might be encoded.
\end{abstract}


\section{Introduction}\label{sec:Intro}


\subsection*{Paper Plan - 20 page limit}

\begin{itemize}
  \item motivate by talking about
  concurrency theory calculation travails (\S\ref{sec:Intro})
  \item explain decision to build a calculator
   (noting emerging tool work on PML as the seed)
   explain why UTP2 was not suitable.(\S\ref{sec:Intro})
\end{itemize}


\subsection{Research Context}

We have recently started to explore using UTP
to do the  formal modelling of a language,
called ``Process Modelling Language'' (PML),
designed to describe software development and similar business
processes \cite{DBLP:journals/infsof/AtkinsonWN07}.
The main objective is to give PML a formal semantics,
as the basis for a number of analysis tools that could be made available
to process designers and users---with one key application
area being the modelling of clinical healthcare pathways.
It quickly became apparent that PML and similar (business) process
notations essentially involve concurrency with global shared mutable state.

There has been work using UTP
to model concurrent programs with shared mutable state,
most notably
with an encoding into actions systems having been done by
Woodcock and Hughes\cite{DBLP:conf/icfem/WoodcockH02}.
We have been looking at adapting this work to provide a UTP semantics
for PML.
In addition, we have started to take an interest in the ``Views'' paper
by Dinsdale-Young and colleagues\cite{conf/popl/Dinsdale-YoungBGPY13},
that provides a framework within which it is possible
to construct instances of many different concurrency theories,
ranging from
type-theory \cite{tal-toplas,Smit00b,journals/fuin/AhmedFM07},
Owicki-Gries\cite{Owicki76},
separation logic\cite{conf/lics/CalcagnoOY07}
and rely-guarantee\cite{Jones83}
approaches,
among others.


\subsection{Theory Construction Difficulties}

\subsection{A Plan}

Haskell\cite{Haskell2010}

A book by Bird \cite{Bird14}

Fun \cite{conf/esop/Rudiak-GouldMJ06}.


\subsection{Structure of this paper}

\begin{itemize}
  \item
     explain the key design decisions,
     and how these were motivated by agile software development
     principles (esp. YAGNI). (\S\ref{sec:Design})
     Such decisions include
     \begin{itemize}
       \item dictionaries --- originally to support tailored rendering
       \item taking pretty-printing seriously
       \item relying on simple hand-proved (algebraic!) laws
       \item the avoidance of quantifiers,
        and how this forces us to deal with substitutability
       \item deciding to add a save-to-file option
        just in time to produce material for this paper
     \end{itemize}
  \item
     Present a development of a theory using the calculator,
     step by step.(\S\ref{sec:Theorising})
  \item
    Talk about how an agile approach really is important,
    not just in writing the calculator software and Haskell models,
    but also when deciding what definitions and laws to focus on.
    Discuss finding the ``sweet spot'' between the use of definition
    expansion, and hard-coded laws based on by-hand proofs.
    Note the key observation that the calculator approach
    leads almost inexorably towards a strong emphasis
    on some kind of (combinator?) algebra. (\S\ref{sec:Conc})
  \item
    Discuss future plans: formal release as a Haskell library;
    REPL customization; rendering options: \LaTeX, UTP2, Isabelle/UTP.
    (\S\ref{sec:Future})
\end{itemize}

\section{Related Work}\label{sec:Related}

There are lots of tools for assisting with the kinds of
calculations we are trying to perform,
ranging from calculators
\cite{Bird14},
through rewrite/transformation
systems
(CIP-S\cite{DBLP:books/sp/BauerEH87},
Stratego\cite{conf/rta/Visser01},
ASF+DSF\cite{VanDenBrand:2002:CLD}
Maude\cite{conf/rta/ClavelDELMMT03}
HATS\cite{conf/gttse/WinterB06})
to full-blown theorem provers
(Isabelle/HOL\cite{books/sp/NipkowPW02},
CoQ\cite{bk:Coq'Art:04},
PVS\cite{conf/fmcad/Shankar96})
including those that support equational reasoning
(Isabelle/ISAR\cite{man:Isabelle:Isar:Wenzel:10}).

Most of the above have a considerable body of work behind them,
both in terms of theory and tool development,
and provide very comprehensive coverage of their problem domain,
be it rewriting, program transformation or theorem proving.
However many are tied to specific languages,
or have limited ability to allow the user to customise the target language.
In particular,
it is not clear in any of them,
how to achieve the ability to do rapid calculation
with a high degree of ease in proof-reading its
output.

Within the UTP community,
there has been considerable work using Proof{\-}Power-Z,
such as the deep embedding into Z of an imperative language whose semantics
were given using UTP \cite{conf/utp/NukaW06}
and
 re-working the mechanisation of UTP in order to better support
the hierarchical nature of UTP theory building \cite{journals/entcs/ZeydaC09}.
%Some support for Z-like theories from UTP (such as Circus)
%can be found as extensions implemented in the Community Z Tools project
%\cite{conf/zum/MalikU05}.
%
There is also work on embedding UTP into Isabelle/HOL\cite{DBLP:conf/utp/FosterZW14}.
This contains a considerable amount of infrastructure to support UTP's
alphabetised predicates in a general way, with UTP forming a third
sub-syntax in addition to Isabelle/HOL's inner and outer syntaxes.
It continues to undergo continuous improvement\cite{DBLP:series/natosec/FosterW15}.

Like all high-quality state-of-the-art tools,
CoQ, Isabelle/HOL, ProofPower-Z and PVS
all have in common that they work best when used in the manner
for which they were designed%
---in none of these cases does this manner match the way
we wish to work in UTP, as described in the introduction,
without at least a steep learning curve.


We briefly considered using the \UTP2 theorem prover
\cite{DBLP:conf/utp/Butterfield10,DBLP:conf/utp/Butterfield12},
which does support both equational reasoning,
plus a mode in which calculations can be from a starting predicate,
as we require.
However, it would have required a lot of setup effort,
in particular to build the support theories about sets and labels
and generators.
Also, it is currently not in an ideal state,
due to difficulties installing
the relevant third-party software libraries
on more recent versions of Haskell.

However, as part of other ongoing work,
we had developed a parser and some initial analysis tools
in Haskell\cite{Haskell2010},
and this software contained abstract syntax and support
for general predicates.
It became really obvious that this could be quickly adapted,
to mechanise the checking calculations, that were being performed
during each attempt.
In particular,
the key inspiration was the observation,
that the pattern of each calculation was very uniform and similar.
So a decision was taken to construct the calculator described in this paper.
It also has the advantage that it runs on standard Haskell,
and hence it is much easier to future-proof.

\section{Design \& Architecture}\label{sec:Design}

\subsection{Development Process}\label{ssec:development}

A very early decision was made to adhere to Agile Software development
principles \cite{Fowl01a}
in developing this calculator
(to the extent possible given that the roles of Software Engineer, Scrum Manager
and Customer were all rolled into one).
In particular we stuck close to the YAGNI (``Ya Ain't Gonna Need It'') principle%
\footnote{More formally, ``Simplicity---the art of maximizing the amount
of work not done---is essential.''}
which requires us to only write software for a function
that is required at that time.
This does not prevent advanced design planning but does keep
the development focussed on immediate needs.
So initially the focus was on being able to use the calculator
to expand the UTP definition of an atomic action.
Once that was working, then the focus shifted to additional code to
support the calculation of the sequential composition of two atomic actions,
and so on.
For example, the feature to take a final calculation and output it to a file
was only developed when this paper was being written,
because there was no need for it until this point.

\subsection{Software Architecture}\label{ssec:architecture}

All the code described here is available online as a git repo
at
\\\url{https://andrewbutterfield@bitbucket.org/andrewbutterfield/utp-calculator.git}
as Literate Haskell Script%
\footnote{
Note, this paper is not itself a literate Haskell script
(you'll be relieved to know).
}
(\texttt{.lhs}) files in the \texttt{src} sub-directory.

Taking into account the repetitive nature of the calculations,
as mentioned at the end of \S\ref{ssec:plan},
and the need for shorthand notations we very rapidly converged
on four initial design decisions:
\begin{enumerate}
  \item No parsers! All calculation objects are written
  directly in Haskell.
  \item We would keep the expression and predicate datatype declarations
   very simple, with only equality being singled out.
  \item We would need to have a good way to pretty-print long predicates
    that made it easy to see their overall structure
  \item We would reply on a dictionary based system to enable default

\end{enumerate}
From our experience with the \UTP2 theorem-prover we also decided
the following regarding the calculation steps that would be supported:
\begin{itemize}
  \item
    We would not support full  propositional calculus
    or theories of numbers or sets.
    Instead we would support the use of hard-coded relevant laws,
    typically derived from  a handwritten proof.
  \item
    We would avoid, at all costs,
    any use of quantifiers or binding constructs.
  \item
    The calculator use interface would be very simple,
    supporting a few high level commands such as ``simplify''
    or ``reduce''.
    In particular,
    no facility would be provided for the user to identify
    the relevant sub-part of the current goal to which any operation
    should be applied.
\end{itemize}

We will now expand on some of the points above,
and in so doing expose some of the concrete architecture of the calculator code.

\subsection{Expressions and Predicates}\label{ssec:expr-pred}

In Fig. \ref{fig:expr-pred-types} we show the Haskell declarations
of the datatypes to represent both expressions and predicates.
Of course this means we are developing a deep embedding\cite{Gibbons:2014:FDS},
which is required in order to be able to do the kind of calculations we require.
\begin{figure}[tb]
\begin{code}
data Expr s
  = St s | B Bool | Z Int | Var String
  | App String [Expr s] | Sub (Expr s) (Substn s) | Undef

data Pred s
  = T | F | PVar String | Equal (Expr s) (Expr s) | Atm (Expr s)
  | Comp String [MPred s] | PSub (MPred s) (Substn s) | PUndef

type MPred s = (Marks, Pred s)
\end{code}
  \caption{Expression and Predicate Datatypes (\texttt{CalcTypes.lhs})}
  \label{fig:expr-pred-types}
\end{figure}
Both datatypes are parameterised on a generic state type \texttt{s},
which allows us to be able to handle different concrete types
of shared state with one piece of code.
Both types have basic values (\texttt{St},\texttt{B},\texttt{Z},\texttt{T},\texttt{F}),
variables (\texttt{Var},\texttt{PVar}),
generic composites (\texttt{App},\texttt{Comp}),
substitution (\texttt{Sub},\texttt{PSub})
and undefined values (\texttt{Undef},\texttt{PUndef}).
The predicate datatype has a way to embed a (boolean-valued)
expression to form an atomic predicate (\texttt{Atm}),
as well as an explicit form for atomic predicates that take
the form of an equality (\texttt{Equal}).
We give a specialised form for equality simply because
its use in laws is widespread and worth optimising.

Also important to point out is the fact that \texttt{Pred} is
defined in terms of \texttt{MPred}, which in turn is defined
in terms of \texttt{Pred}.
This is done to facilitate the association of markings (lists of integers)
with predicate constructs.
These markings are used to indicate which parts of a predicate
were changed at each calculation step.


\subsection{Pretty-Printing}\label{ssec:pp}

For the calculator output,
it is very important that it be readable,
as many of the predicates get very large,
particularly at intermediate points of the calculation.
For this reason, a lot of effort was put into the development
of both good pretty-printing,
and ways to highlight old and new parts of predicates as changes are made.
The key principle was to ensure that whenever a predicate
had to split over multiple lines,
that the breaks are always around the top-most operator or composition
symbol, with sub-components indented in.
An example of such pretty-printing in action is
{\small
\begin{verbatim}
    D(out)
 \/ (      ~ls(out)
        /\ (D(lg) \/ A(in,lg,a,in,lg,lg) \/ D(out) \/ A(lg,out,b,lg,out,out))
     ; W(D(lg) \/ A(in,lg,a,in,lg,lg) \/ D(out) \/ A(lg,out,b,lg,out,out)))
\end{verbatim}
}
The top-level structure of this is
\[D(out) \lor ( (\lnot ls(out) \land \dots) ; \W(\dots) )\]

The pretty printing support can be found in \texttt{PrettyPrint.lhs}.

\subsection{Dictionaries}\label{ssec:dict}

The abstract syntax is very simple,
and uses names as Haskell Strings to distinguish between
different composites.
So for example logical operators $\land$ and $\lor$
might be represented as \texttt{Comp "And"} and \texttt{Comp "Or"}
respectively.
We have a default way to pretty-print composites,
so that the predicate $\true \land \false$,
represented in Haskell by \texttt{Comp "And" [bT,bF]},
would render as \texttt{And(true,false)}.
However it would be nice to be able to render it as the usual
infix operator, giving \texttt{true /\BS\ false}.
In addition, we need some way to associate laws and definitions,
as appropriate, with composites.

The approach taken is to implement a dictionary that maps names
to entries that supply extra information.
The names can be those of expression or predicate composites,
or correspond to variables, and a few other features of note.
All of the main calculator functions are driven by this
dictionary,
and the correct definition of dictionary entries
is the primary way for users to set up calculations.

The standard approach is that the calculator
will seek information regarding some name that is
currently of interest.
It will perform a dictionary lookup which can result
in either failure, for which a suitable default action will be used,
or success, in which case the dictionary entry will supply
information to guide the required name-specific behaviour.


Also point out how this is a form of shallow-embedding.
So we can get the best (?) of both worlds\cite{Gibbons:2014:FDS}.

There are four kinds of dictionary entries,
one each for expressions and predicates,
one for laws of various kinds,
and one for alphabet handling.
We will discuss each kind in turn,
but leave examples of their use until Sect.\ref{sec:Theorising}.

One big win in using a functional language like Haskell,
in which functions are first class data values,
is that we can easily define datatypes that
contain function-valued components.
We make full use of this in three of the entry kinds discussed below.


\subsubsection{Expression Entries}~

\begin{code}
Entry s
  = ExprEntry 
    { ecansub :: [String]
    , eprint :: Dict s -> [Expr s] -> String
    , eval :: Dict s -> [Expr s] -> (String, Expr s)
    , isEqual :: Dict s -> [Expr s] -> [Expr s] -> Maybe Bool}
  | ...
\end{code}
Imagine that \texttt{App name args }is being processed
and that a dictionary lookup with key \texttt{name}
has returned an expression entry.
The entry contains four pieces of information:
\begin{description}
  \item[\texttt{ecansub}]
     is a list of variable names,
      over which it is safe to perform substitutions.
  \item[\texttt{eprint}]
    is a function that takes a dictionary as first argument,
    the argument-list \texttt{args} as its second argument
    and returns a string rendering of the expression.
    Currently we view expressions as atomic one-line texts
    for output purposes.
  \item[\texttt{eval}]
    takes similar arguments to \texttt{eprint},
    but returns a string/expression pair,
    that denotes a possible evaluation or simplification
    of the original expression.
    The string is empty if no change occurred,
    otherwise it is a short description of/name for the eval/simplify step.
  \item[\texttt{isEqual}]
    has a dictionary argument, followed by two sub-expression lists.
    it tests for (in)equality, returning \texttt{Just True} or \texttt{Just False}
    if it can establish (in)equality,
    and \texttt{Nothing} if unable to give a definitive answer.
\end{description}
The dictionary argument supplied to the three functions
is always the same as the dictionary in which the entry was found.

To understand the need for \texttt{ecansub},
consider the following shorthand definition for an expression:
\[
  D(L) \defs L \subseteq ls
\]
in a context where we know that $L$ is a set expression defined
only over variables $g$, $in$ and $out$.
Now, consider the following instance, with a substitution,
and two attempts to calculate a full expansion:
\begin{eqnarray*}
   & D(\setof{out})[\setof{\ell_1},\ell_2/ls,out] &
\\ (\setof{out \subseteq ls})[\setof{\ell_1},\ell_2/ls,out]=
 && = D(\setof{\ell_2})
\\ \setof{\ell_2} \subseteq \setof{\ell_1}=
 && = \setof{\ell_2} \subseteq ls
\end{eqnarray*}
The lefthand calculation is correct, the righthand is not.
The substitution refers to variables (e.g. $ls$)
that are hidden when the $D$ shorthand is used.
The \texttt{ecansub} entry lists the variables for which substitution
is safe with the expression as-is.
With the definition above, the value of this entry
 should be \texttt{["g","in","out"]}.
Given that entry, the calculator would simplify (correctly) as follows:
\[
  D(\setof{\ell_2})[\setof{\ell_1}/ls]
\]
The righthand side of the definition is what should be returned
by the \texttt{eval} component.
If we want to state that any substitution is safe,
then we use the ``wildcard'' form: \texttt{["*"]}.


\subsubsection{Predicate Entries}~

\begin{code}
Entry s
  = ...
  | PredEntry 
    { pcansub :: [String]
    , pprint :: Dict s -> MarkStyle -> Int -> [MPred s] -> PP
    , alfa :: [String]
    , pdefn :: Rewrite s
    , prsimp :: Rewrite s }
type Rewrite s = Dict s -> [MPred s] -> (String, Pred s)
\end{code}
The predicate entry associated with Comp name pargs
has five fields:
\begin{description}
  \item[\texttt{pcansub}]
    does for predicates what \texttt{ecansub} does for expressions.
  \item[\texttt{pprint}]
    is similar to \texttt{eprint} except it has two extra arguments,
    used to help with special renderings for predicates marked as changed,
    and precedence levels for managing bracketing.
    We do pretty-printing tricks such as indenting and adding line-breaks
    at the predicate level.
  \item[\texttt{alfa}]
    is used to identify the alphabet of the predicate.
  \item[\texttt{pdefn}]
    is a function invoked when \texttt{name} has a definition,
    and we want to expand it.
  \item[\texttt{prsimp}]
    is called by the simplifier when processing \texttt{name}.
\end{description}
Both \texttt{pdefn} and \texttt{prsimp} take a dictionary argument
and list of sub-component (marked)predicates,
returning a string/predicate pair,
interpreted in a manner similar to \texttt{eval} above.

\subsubsection{Law Entries}~

\begin{code}
Entry s
  = ...
  | LawEntry  { reduce :: [DictRWFun s]
              , creduce :: [CDictRWFun s]
              , unroll :: [String -> DictRWFun s] }
\end{code}
Law entries are currently only associated with one key,
namely the string \texttt{"laws"},
which is what the calculator uses to find such entries.
There are three parts,
each consisting of lists of functions.
These lists are intended to be be applied in order to the
``current predicate'', until either one succeeds,
or none do.
All the functions take a dictionary
and (marked) predicate as arguments,
and return either
an single-outcome indicator (\texttt{reduce},\texttt{unroll}):
\begin{code}
type DictRWFun s = Dict s -> RWFun sdata
type RWFun s = MPred s -> RWResult s
type RWResult s = (String, MPred s)
\end{code}
or a conditional  multiple-outcome result (\texttt{creduce}).
\begin{code}
type CDictRWFun s = Dict s -> CRWFun s
type CRWFun s = MPred s -> CRWResult s
type CRWResult s = (String, [(Pred s, MPred s)])
\end{code}
\begin{description}
  \item[\texttt{reduce}]
     is a list of reduction laws,
     to be tried out when a reduction step is invoked by the user.
  \item[\texttt{creduce}]
    is a list of conditional reduction laws,
    which have multiple outcomes dependent on a side-condition.
    Rather than try to resolve conditions automatically,
    we prefer to let the user choose the appropriate outcome.
  \item[\texttt{loop}]
    is a list of loop-unrolling functions.
    The extra string argument is to give the user finer control
    of how much unrolling is done, and how it is presented
    (see later).
\end{description}


\subsubsection{Alphabet Entries}~

\begin{code}
Entry s
  = ...
  | AlfEntry  { avars :: [String]}
\end{code}
Most simply put, an alphabet is simply a set of variables.
In any UTP theory we typically have well-defined alphabets,
often with particular subsets of interest,
such as all the ``before'' observations (undashed variables),
or ``after'' observations (dashed variables).
We give these subsets standardised names,
and use the dictionary to list the relevant variables.


\subsection{The Calculator REPL}

The way the calculator is designed to be used is
that a function implementing a calculator Read-Execute-Print-Loop (REPL)
is given a dictionary and starting predicate as inputs.
It then offers the user the opportunity to invoke various
commands to perform calculation steps.
The user can then indicate when they are finished,
at which point the calculator function returns
a data-structure that logs the complete calculation outcome.
This is a triple consisting of the final predicate,
a list of the steps, each with a justification string,
and the dictionary that was used.
\begin{code}
type CalcLog s = (MPred s, [RWResult s], Dict s)
\end{code}

Calculator commands include an ability to undo previous steps ('\texttt{u}'),
request help ('\texttt{?}'),
and to signal an exit from the calculator ('\texttt{x}').
However,
of most interest are the five calculation commands.
The first is a global simplify command ('s'),
that scans the entire predicate from the bottom-up
looking for simplifiers for each composite and applying them.
Simplifiers are captured as \texttt{eval} or \texttt{prsimp} components
in dictionary entries.

The other four commands work by searching top-down, depth-first for
the first sub-component for which the relevant dictionary calculator
function returns a changed result.
Here is where we have a reduced degree of control,
which simplifies the REPL dramatically,
but turns out to be strikingly effective in practise.
This is because these kinds of semantic ``smoke-test'' calculations
tend to go in phases: expand all definitions; simplify; reduce; simplify;
perhaps unroll a loop a bit; etc...
\begin{description}
  \item[Defn. Expand ('\texttt{d}')]
    Find the first predicate changed by applying its
     \texttt{pdefn} dictionary entry, which should unfold its definition.
  \item[Law Reduce ('\texttt{r}')]
    Find the first predicate transformed by a function
    in the \texttt{reduce} list of the \texttt{LawEntry} indexed
     by the string \texttt{"laws"}.
     This function captures an equational law,
     that is only applied in a left to right direction.
  \item[Loop Unroll ('\texttt{l}')]
    Find the first predicate transformed by a function
    in the \texttt{unroll} list of the \texttt{LawEntry} indexed
     by the string \texttt{"laws"}.
    The remainder of the command string after the initial '\texttt{l}'
    is passed to the function to control the nature and degree
    of unrolling. How this string is interpreted is entirely
    up to the user---see Sect. \ref{sec:Theorising} for an example.
  \item[Conditional Reduce ('\texttt{c}')]
    Find the first predicate transformed by a function
    in the \texttt{creduce} list of the \texttt{LawEntry} indexed
     by the string \texttt{"laws"}.
     This function captures an equational law with a side condition,
    which returns a list of alternatives,
    each alternative being a side-condition predicate
    paired with a result.
    These are presented to the user,
    who then selects which outcome is appropriate.
    In effect the user has to look at each condition
    and select the one (if any) that evaluates to true.
    This is done to prevent the calculator from having
    to embody predicate simplifiers ``modulo various expression theories''.
\end{description}

\section{Types}\label{sec:Types}

We give an overview of key types in the calculator here,
focussing on Expressions, Predicates and the Dictionary.

\subsection{Expressions}

In Fig. \ref{fig:expr-types} we show the Haskell declarations
of the datatypes used to represent expressions and substitution.
\begin{figure}[tb]
\begin{code}
data Expr s
  = St s | B Bool | Z Int | Var String
  | App String [Expr s] | Sub (Expr s) (Substn s) | Undef

type Substn s = [(String,Expr s)]
\end{code}
  \caption{Expression and Sunstitution Datatypes (\texttt{CalcTypes.lhs})}
  \label{fig:expr-types}
\end{figure}
Both datatypes are parameterised on a generic state type \texttt{s},
which allows us to be able to handle different concrete types
of shared state with one piece of code.
Both types have basic values (\texttt{St},\texttt{B},\texttt{Z},\texttt{T},\texttt{F}),
variables (\texttt{Var},\texttt{PVar}),
generic composites (\texttt{App},\texttt{Comp}),
substitution (\texttt{Sub},\texttt{PSub})
and undefined values (\texttt{Undef},\texttt{PUndef}).
The predicate datatype has a way to embed a (boolean-valued)
expression to form an atomic predicate (\texttt{Atm}),
as well as an explicit form for atomic predicates that take
the form of an equality (\texttt{Equal}).
We give a specialised form for equality simply because
its use in laws is widespread and worth optimising.

The abstract syntax is very simple,
and uses names as Haskell Strings to distinguish between
different composites.

\subsection{Predicates}

In Fig. \ref{fig:pred-types} we show the Haskell declarations
of the datatypes to represent predicates.
\begin{figure}[tb]
\begin{code}
data Pred s
  = T | F | PVar String | Equal (Expr s) (Expr s) | Atm (Expr s)
  | Comp String [MPred s] | PSub (MPred s) (Substn s) | PUndef

type MPred s = (Marks, Pred s)
\end{code}
  \caption{Predicate Datatypes (\texttt{CalcTypes.lhs})}
  \label{fig:pred-types}
\end{figure}
Both datatypes are parameterised on a generic state type \texttt{s},
which allows us to be able to handle different concrete types
of shared state with one piece of code.
Both types have basic values (\texttt{St},\texttt{B},\texttt{Z},\texttt{T},\texttt{F}),
variables (\texttt{Var},\texttt{PVar}),
generic composites (\texttt{App},\texttt{Comp}),
substitution (\texttt{Sub},\texttt{PSub})
and undefined values (\texttt{Undef},\texttt{PUndef}).
The predicate datatype has a way to embed a (boolean-valued)
expression to form an atomic predicate (\texttt{Atm}),
as well as an explicit form for atomic predicates that take
the form of an equality (\texttt{Equal}).
We give a specialised form for equality simply because
its use in laws is widespread and worth optimising.

Also important to point out is the fact that \texttt{Pred} is
defined in terms of \texttt{MPred}, which in turn is defined
in terms of \texttt{Pred}.
This is done to facilitate the association of markings (lists of integers)
with predicate constructs.
These markings are used to indicate which parts of a predicate
were changed at each calculation step.
We will not discuss marking further in this document
as it runs completely ``under the hood'',
and its only impact on the users of this calculator
is their need to be aware of the interplay involving \texttt{Pred} and \texttt{MPred}.

The abstract syntax is very simple,
and uses names as Haskell Strings to distinguish between
different composites.
So for example logical operators $\land$ and $\lor$
might be represented as \texttt{Comp "And"} and \texttt{Comp "Or"}
respectively.
We have a default way to pretty-print composites,
so that the predicate $\true \land \false$,
represented in Haskell by \texttt{Comp "And" [bT,bF]},
would render as \texttt{And(true,false)}.
However it is nice to be able to render it as the usual
infix operator, giving \texttt{true /\BS\ false}.
In addition, we need some way to associate laws and definitions,
as appropriate, with composites.

\subsection{Dictionary}





The data-structure that logs the complete calculation outcome
is a triple consisting of the final predicate,
a list of the steps, each with a justification string,
and the dictionary that was used.
\begin{code}
type CalcLog s = (MPred s, [RWResult s], Dict s)
\end{code}

\section{Expressions}\label{sec:Expressions}

In Fig. \ref{fig:expr-types} we show the Haskell declarations
of the datatypes used to represent expressions and substitution.
\begin{figure}[tb]
\begin{verbatim}
data Expr s
  = St s | B Bool | Z Int | Var String
  | App String [Expr s] | Sub (Expr s) (Substn s) | Undef
  deriving (Eq, Ord, Show)
type Substn s = [(String,Expr s)]
\end{verbatim}
  \caption{Expression and Substitution Datatypes (\texttt{CalcTypes.lhs})}
  \label{fig:expr-types}
\end{figure}
Both types are parameterised on a generic state type \texttt{s},
which allows us to be able to reason independently
of any particular notion of state.
We provide booleans (\texttt{B}),
integers (\texttt{Z}),
values of the generic state type (\texttt{St}),
named function application (\texttt{App}).
We also have substitution (\texttt{Sub}), which pairs an expression
with a substitution (\texttt{Substn}),
which is a list of variable/expression pairs.
The \texttt{deriving} clause for \texttt{Expr} 
enables the Haskell default notions
of equality, ordering and display for the type.

\subsection{Set Expressions}

We shall explore the use of the \texttt{Expr}  datatype
by indicating how the notions of sets and some basic operators
could be defined with the calculator.
We shall represent sets as instances of \texttt{App} with the name ``set'',
and the subset relation as an \texttt{App} with name ``subset'',
so the set $\setof{1,2}$  and predicate $S \subseteq T$
would be represented by
\verb$App "set" [Z 1,Z 2]$,
and
\verb$App "subset" [Var "S",Var "T"]$ respectively.
In practice, we would define constructor functions to build
these:
\begin{code}
set es = App "set" es
subset s1 s2 = App "subset" [s1,s2]
\end{code}
There is a standard interface for defining expression simplifiers:
define a function with the following type:
\begin{verbatim}
Dict s -> [Expr s] -> (String, Expr s)
\end{verbatim}
The first argument, of type \texttt{Dict},
is the dictionary currently in use.
The second argument is the list of sub-expressiosn of
the \texttt{App} construct for which the simplifier is intended.
The result is a pair consisting of a string and an expression.
If the simplification succeeds, then the string is non-empty
and gives some indication for the user
of what was simplified.
In this case the expression component is the simplified result.
If the simplification has no effect, then the string is empty,
and the expression returned is not defined.

The following code defines a simplifier for subset,
which expects it to have precisely two set components:
\begin{code}
evalSubset d [App "set" s1,App "set" s2] = dosubset d s1 s2
evalSubset _ _ = none -- predefined shorthand for ("",Undef)
\end{code}
The two underscores in the second line are pattern matching
wildcards, so this catches all other possibilities.
It makes use of the following helper,
which gets the two lists of expressions associated with each set:
\begin{code}
dosubset d es1 es2 -- is es1 a subset of es2 ?
  | null (es1 \\ es2)  =  ( "subset", B True )
  | all (isGround d) ((es1 \\ es2) ++ es2)
                       =  ( "subset", B False )
  | otherwise          =  none
\end{code}
If the result of removing \texttt{es2} from \texttt{es1} is null 
it then returns true.
If not, then if all elements remaining are ``ground'',
i.e., contain no variables, returns false.
Otherwise, we cannot infer anything, so return \texttt{none}.

\subsection{Rendering Expressions}

The UTCP theory definitions and calculations
involve a lot of reasoning about sets,
leading to quite complicated expressions.
To avoid complex set expressions that are hard to parse visually,
a number of simplifying notations are desirable,
so that a singleton set $\setof x$ is rendered as $x$
and the very common idiom $S \subseteq ls$
is rendered as $ls(S)$,
so that for example, $ls(in)$ is short for $\setof{in} \subseteq ls$.
This shrinks the expressions to a much more readable form,
mainly by reducing the number of infix operators and set brackets.


When rendering expressions,
if an \texttt{App} construct is found, then its name
is looked up in the dictionary.
If an \texttt{ExprEntry} is not found, then the default rendering is used,
in which \verb$App "f" [e1,e2,..,en]$
is converted into \verb$f(e1,e2,..,en)$.
Otherwise, a function of type \verb$Dict s -> [Expr s] -> String$,
in that entry, is used to render the construct.

As far as expressions are concerned,
they become strings, and so are viewed as atomic
by the predicate pretty-printer (see Sect. \ref{sec:Predicates}).
So, we could show singleton sets without enclosing braces
by defining:
\begin{code}
showSet d [elm] = edshow d elm   -- drop {,} from singleton
showSet d elms = "{" ++ dlshow d "," elms ++ "}"
\end{code}
Here \texttt{edshow} (expression-dict-show)
displays its \texttt{elm} argument,
while \texttt{dlshow} (dictionary-list-show) displays the expressions
in \texttt{elms} separated by the \verb@","@ string.
%Similar tricks are used to code a very compact rendering
%of a mechanism that involves unique label generator expressions
%that involve very deep nesting, such as:
%\[
% \pi_2(new(\pi_1(new(\pi_2(split(\pi_1(new(g))))))))
%\]
%This can be displayed as \texttt{lg:2:},
%using a very compact shorthand described in \cite{conf/tase/BMN16}
%which we do not explain here.


\subsection{Expression Equality}

In contrast to the way that the subset predicate
is captured as an expression above,
the notion of expression equality is hardwired in,
as part of the predicate abstract syntax (see Sect. \ref{sec:Predicates}).
The simplifier will look at the two expression
arguments of that construct,
and if they are both instances of \texttt{App} with the same name,
will do a dictionary lookup, to see if there
is an entry, from
which an equality checking function can be obtained (\texttt{isEqual} component).
This has the following signature:
\begin{verbatim}
Dict s -> [Expr s] -> [Expr s] -> Maybe Bool
\end{verbatim}
The \texttt{Maybe} type constructor is standard Haskell, defined as
\begin{verbatim}
data Maybe t = Nothing | Just t
\end{verbatim}
It converts a type \texttt{t} into one which is now ``optional'',
or equivalently has a undefined value added.

The equality testing function takes a dictionary and the two expression
lists from the two \texttt{App} instances
and either returns \texttt{Nothing},
if it cannot establish the truth or falsity of the equality,
or \texttt{Just} the appropriate result.
Suitable code for \verb$"set"$ is the following
\begin{code}
eqSet d es1 es2
 = let ns1 = nub $ sort $ es1 ; ns2 = nub $ sort $ es2
   in if all (isGround d) (ns1++ns2)
      then Just (ns1==ns2) else Nothing
\end{code}
The standard function \texttt{nub} removes duplicates,
which we do after we \texttt{sort}.
If both lists are ground we just do an equality comparison
and return \texttt{Just} it. Otherwise, we return \texttt{Nothing}.




\subsection{The Expression Entry}~

The dictionary entry for expressions has the following form:
\begin{verbatim}
ExprEntry
    { ecansub :: [String]
    , eprint :: Dict s -> [Expr s] -> String
    , eval :: Dict s -> [Expr s] -> (String, Expr s)
    , isEqual :: Dict s -> [Expr s] -> [Expr s] -> Maybe Bool}
\end{verbatim}
One big win in using a functional language like Haskell,
in which functions are first class data values,
is that we can easily define datatypes that
contain function-valued components.
We make full use of this in three of the entry kinds,
for expressions, predicates and laws.

The \texttt{eprint}, \texttt{eval} and \texttt{isEqual} components correspond
to the various examples we have seen above.
The \texttt{ecansub} component
indicates those variables occurring in the \texttt{App} expression
list for which it is safe to replace in substitutions.

To understand the need for \texttt{ecansub},
consider the following shorthand definition for an expression:
\[
  D(L) \defs L \subseteq ls
\]
in a context where we know that $L$ is a set expression defined
only over variables $g$, $in$ and $out$.
The variable $ls$ is not free in the lhs,
but does occur in the rhs.
A substitution of the form $[E/ls]$ say,
would leave the lhs unchanged, but alter the rhs to $L \subseteq E$.
For this reason the entry for $D$ would need to disallow substitution
for $ls$.
The \texttt{ecansub} entry lists the variables for which substitution
is safe with the expression as-is.
With the definition above, the value of this entry
 should be \texttt{["g","in","out"]}.
If we want to state that any substitution is safe,
then we use the ``wildcard'' form: \texttt{["*"]}.
We choose to list the substitutable variables
rather than those that are non-substitutable,
because the former is always easy to determine,
whereas the latter can be very open ended.

Given all of the above,
we can define dictionary entries for set and subset as
\begin{code}
setUTCPDict = makeDict
 [ ("set",(ExprEntry ["*"] showSet evalSet eqSet))
 , ("subset",(ExprEntry ["*"] showSubSet evalSubset noEq)) ]
\end{code}
Here \texttt{noEq} is an equality test function that always returns \texttt{Nothing}.

\section{Predicates}\label{sec:Predicates}

In Fig. \ref{fig:pred-types} we show the Haskell declarations
of the datatypes to represent predicates.
\begin{figure}[tb]
\begin{verbatim}
data Pred s
  = T | F | PVar String | Equal (Expr s) (Expr s) | Atm (Expr s)
  | Comp String [Pred s] | PSub (Pred s) (Substn s)
\end{verbatim}
  \caption{Predicate Datatype (\texttt{CalcTypes.lhs})}
  \label{fig:pred-types}
\end{figure}

Similar to expressions we have basic values such as true (\texttt{T})
and false (\texttt{F}),
 with predicate-valued variables (\texttt{PVar}),
and composite predicates (\texttt{Comp}) which are the predicate equivalent
of \texttt{App} (see Sect. \ref{sec:Expressions}).
We also have two ways to turn expressions into predicates.
One (\texttt{Atm}) lifts an expression, which should be boolean-valued
into an (atomic) predicate,
while the other is an explicit representation (\texttt{Equal})
for expression equality.
We can also substitute over predicates (\texttt{PSub}).

In many ways,
we define our predicates of interest
in much the same was as done for expressions.
Basic logic features such as negation, conjunction, etc.,
are not built in,
but have to be implemented using \texttt{Comp}.
A collection of these are pre-defined as part of the calculator,
in the Haskell module \texttt{StdPredicates}.

There are a few ways in which the treatment of predicates
differ from expressions:
\begin{itemize}
  \item
    The simplifier and some of the infrastructure for handling
    laws treats \texttt{PVar} in a special way.
    It is possible to associate an \texttt{AlfEntry} in the dictionary
    with a \texttt{PVar}, so defining its alphabet.
    This can be useful when reasoning about atomic state-change
    actions which only depend on $s$ and $s'$.
    Such entries will be looked up when certain
    side-conditions are being checked.
  \item
    We distinguish between having a definition associated with a \texttt{Comp},
    and having a way to simplify one.
  \item
    Rendering predicates involves the pretty printer
    so the interface is more complex.
    We explain this below.
\end{itemize}



\subsection{Coding Atomic Semantics}

Formally, using our shorthand notations, we define atomic behaviour as:
\[
    \A(A) \defs ls(in) \land A \land ls'=ls\ominus(in,out)
\]
where $A$ is a predicate whose alphabet is restricted to $s$ and $s'$,
$\A(\_)$ lifts this to the full alphabet,
and $S\ominus(T,V)$ is notation from \cite{DBLP:conf/icfem/WoodcockH02}
that stands for $(S\setminus T)\cup V$.


\subsubsection{Coding a Definition}

We want to define a composite, called "A" (representing $\A$).
We  define a function that takes a single predicate argument
and applies $\A$ to it
\begin{code}
patm pr = Comp "A" [pr] -- we assume pr has only s, s' free
\end{code}
We can now code up its definition,
which takes a dictionary, and a list of its subcomponents
and returns a string/predicate pair,
interpreted in the same manner as the string/expression pair
returned by the expression simplifier.

One way to code this is as follows.
First define our variables and expressions,
because these get used in a variety of places.
\begin{code}
ls = Var "ls" ; ls' = Var "ls'"
inp = Var "in" -- 'in' is a Haskell keyword
out = Var "out"
lsinout = App "sswap" [ls,inp,out]
\end{code}
Here, \verb@"sswap"@ is our name for $\ominus$,
and note that Haskell identifiers can contain
the prime (\verb@'@) character.
We then define our atomic predicates ($ls(in)$ and $ls'=ls\ominus(in,out)$)
\begin{code}
lsin = Atm $ App "subset" [inp,ls]
ls'eqlsinout = Equal ls' lsinout
\end{code}
Finally we can define $\A(a)$ as their conjunction,
where \texttt{mkAnd} is a smart constructor for \verb@Comp "And"@,
defined in \texttt{StdPredicates.lhs},
that does some optimisations on the fly (e.g. \verb$mkAnd [pr] = pr$).
\begin{code}
defnAtomic d [a] = Just ("A",mkAnd [lsin,a,ls'eqlsinout],True)
\end{code}
Here \texttt{a} represents the single ($A$) argument we expect.

\subsubsection{Coding for Pretty Printing}

For rendering \texttt{Comp} predicates, we are going to generate an instance
of the pretty-printer type \texttt{PP}, using a dictionary
and list of sub-predicates,
with two additonal arguments:
one of type \texttt{MarkStyle} which describes how markings should be rendered,
and one of type \texttt{Int} which gives a precedence level.
The type signature is
\begin{verbatim}
SubCompPrint s -> Dict s -> Int -> [Pred s] -> PP
\end{verbatim}
The first argument is a function,
and it is a bit of Haskell higher-order programming designed
to make it easier to track predicate contexts in order
to facilitating the generation of highlighting.
The user does not need to write this,
as it is supplied as the first argument by the pretty-printing
infrastructure.
What the user should do is use it in any recursive call
to pretty-print a sub-predicate.
\begin{verbatim}
type SubCompPrint s
 = Int       -- precedence level for sub-component
   -> Int    -- sub-component number, starts at 1
   -> Pred s -- sub-component
   -> PP
\end{verbatim}
It takes two integer arguments to begin.
The first is the precedence level to be used to render the
sub-component,
while the secound should denote the position of the sub-component
in the sub-component list, counting from 1.

Such code, for example, would be like the following
to render our atomic construct:
\begin{code}
ppPAtm sCP d p [pr]
 = pplist [ ppa "A"
          , ppbracket "(" (sCP 0 1 pr) ")"]
\end{code}
The functions \texttt{pplist}, \texttt{ppa} and \texttt{ppbracket}
build instances of \texttt{PP} respectively,
that represent lists of \texttt{PP},
atomic strings,
and an occurence of \texttt{PP} surrounded by the designated brackets.
Note that the \texttt{SubCompPrint} argument (\texttt{sCP})
is applied to \texttt{pr},
with the precedence is set to zero as it is bracketed,
and the sub-component number set to one as \texttt{pr} is the first
(and only) sub-component.



We will show how the pretty-printing for
sequential composition  ($\lseq$) in UTCP is defined,
to illustrate the support for infix notation.
\begin{code}
ppPSeq sCP d  p [pr1,pr2]
 = paren p precPSeq
     $ ppopen  (pad ";;") [ sCP precPSeq 1 pr1
                          , sCP precPSeq 2 pr2 ]
\end{code}
Here \texttt{pad} puts spaces around its argument,
and so its user here is equivalent to \verb$ppa " ;; "$,
while \texttt{ppopen} uses its first argument as a
separator between all the elements of its second list argument.
The \texttt{paren} function takes two precedence values,
and a \texttt{PP} value, and puts parentheses around it if the first precedence
number is greater than the second.
The variable \texttt{precPSeq} is the precedence level of sequential composition,
here defined to be tighter than disjunction,
but looser than conjunction, as defined in module \texttt{StdPrecedences}.
Note once more, the use of \texttt{sCP}, and how the 2nd integer argument
corresponds to the position of the sub-predicate involved.

\subsubsection{The Predicate Entry}~

The dictionary entry for predicates has the following form:
\begin{verbatim}
 | PredEntry
   { pcansub :: [String]
   , pprint  :: SubCompPrint s -> Dict s -> Int -> [Pred s] -> PP
   , alfa :: [String]
   , pdefn   :: Rewrite s
   , prsimp  :: Rewrite s
   }
type Rewrite s = Dict s -> [Pred s] -> RWResult s
type RWResult s
 = Maybe ( String  -- rewrite justification
         , Pred s  -- result predicate
         , Bool )  -- True if top-level modified
\end{verbatim}
Fields \texttt{pcansub} and \texttt{prsimp} are the predicate analoges
of \texttt{ecansub} and \texttt{eval} in the expression entry.
Here \texttt{pprint} plays the same role as eprint,
but is oriented towards pretty printing.
The \texttt{alfa} component allows an specific alphabet to
be associated with a composite
---if empty then the dictionary alphabet entries apply.

The \texttt{pdefn} component, of the same type as \texttt{prsimp},
is used when the user invokes the Definition Expansion
command from the REPL.
The calculator searches top-down, left-right
    for the first \texttt{Comp} whose \texttt{pdefn} function
    returns a changed outcome.

A \texttt{RWResult} can be \texttt{Nothing},
in which case this definition expansion or simplifier
was unable to make any changes.
If it was able to change its target then it returns
\texttt{Just(reason,newPred,isTopLevel)}.
The string \texttt{reason} is used to display the justification for the
calculation step to the user.
The \texttt{isTopLevel} flag is a hint to the change highlighting facilities
of the pretty-printer infrastructure.


The dictionary entry for our atomic semantics is then:
\begin{code}
patmEntry=("A",PredEntry [] ppPAtm [] defnAtomic (pNoChg "A"))
\end{code}
The function \texttt{pNoChg} creates a simplifer that returns \texttt{Nothing}.

\section{Laws}\label{sec:Laws}

\DRAFT{Talk about reduce, creduce and unroll}

\subsection{Coding UTCP Laws}

The definition of the semantics of the UTCP language
constructs, and of $run$,
make use of the (almost) standard notions of skip,
sequential composition
and iteration in UTP.
The versions used here are slightly non-standard because we have
non-homogeneous relations,
where the static parameters have no dashed counterparts.
In essence we ignore the parameters as far as flow-of-control is concerned:
\RLEQNS{
   \Skip &\defs& s'=s \land ls'=ls
\\ P ; Q
   &\defs&
   \exists s_m,ls_m @
     P[s_m,ls_m/s',ls']
     \land
     Q[s_m,ls_m/s,ls]
\\ c * P &\defs& \mu L @ (P ; L) \cond c \Skip
\\ P \cond c Q &\defs& c \land P \lor \lnot c \land Q
}
Here, the definition of $\cond\_$ is entirely standard, of course.

What is key here though,
is realising that we do not want to define the constructs
as above and use them directly, as it involves
quantifiers and explicit recursion,
both of which would introduce considerable complexity to the calculator.
Instead, we encode useful laws that they satisfy,
that do not require their definitions to be expanded.
Such laws might include the following:
\RLEQNS{
  \Skip \seq\ P & {} = P = {} & P \seq \Skip
\\ c * P &=& (P \seq c* P ) \cond c \Skip
\\ (c * P)[e/x] &=& P[e/x] \seq c * P, \qquad if c[e/x]
\\ (c * P)[e/x] &=& \Skip[e/x] , \qquad if \lnot c[e/x]
}
These laws need to be proven by hand (carefully),
by the theory developer, and then encoded into Haskell
(equally carefully), as we are about to describe.

We can easily give a definition of $\Skip$,
which is worth expanding.
\RLEQNS{
   \Skip &\defs& s'=s \land ls'=ls
}
\begin{code}
defnUTCPII = mkAnd[ equal s' s, equal ls' ls ]
\end{code}

For more complex laws,
the idea is that we pattern-match on predicate syntax
to see if a law is applicable (we have its lefthand-side),
and if so,
we then build an appropriate instance of the righthand-side.
The plan is that we gather all these pattern/outcome pairs
in one function definition,
which will try them in order.
This is a direct match for how Haskell pattern-matching works.
So for UTCP we have a function called \texttt{reduceUTCP},
structured as follows:
\begin{code}
reduceUTCP :: (Show s, Ord s) => DictRWFun s
reduceUTCP (...1st law pattern...) = 1st outcome
reduceUTCP (...2nd law pattern...) = 2nd outcome
...
reduceUTCP d mpr = lred "" mpr  -- catch-all at end, no change
\end{code}
The last clause matches any predicate
and simply returns it with a null string,
indicating no change took place.
The main idea is find a suitable collection of patterns,
in the right order,
to be most effective in performing calculations.
The best way to determine this is start with none,
run the calculator and when it stalls
(no change is happening for any command),
see what law would help make progress, and encode it.
This is the essence of the agile approach to theory calculator development.

A simple example of such a pattern is the following encoding
of $\Skip;P = P$ :
\begin{code}
reduceUTCP d
 (_,Comp "Seq" [(_,Comp "Skip" []), mpr]) = lred ";-lunit" mpr
\end{code}
The second argument has type marked-predicate (\texttt{MPred})
which is a marking/predicate pair.
We are not interested in the markings
so we use the wildcard pattern '\verb"_"'
for the first pair component.
The sub-pattern in the second pair component,
\verb'Comp "Seq" [(_,Comp "Skip" []), mpr])',
matches a composite called ``Seq'',
with a argument list containing two (marked) predicates.
The first argument predicate pattern \verb'(_,Comp "Skip" [])'
matches a ``Skip'' composite with no further subarguments.
The second argument pattern \verb'mpr' matches an arbitrary predicate
($P$ in the law above).
The righthand-side returns the application \verb'lred ";-lunit" mpr'
which simply constructs a string/predicate pair,
with the string being a justification note that says a reduction-step
using a law called ``$;$-lunit'' was applied.



\subsection{UTCP Recognisers}

Some laws require matching that is a bit more sophisticated.
For example,
consider a useful reduction for tidying up at the end,
assuming that $ls' \notin A$ and $ls \notin B$, and both $k$ and $h$
are ground:
\[
   (A \land ls'=k) ; (B \land ls'= h)
   \quad\equiv\quad
   (A;B) \land ls'=h
   \qquad \elabel{$ls'$-cleanup}
\]
However, we want this law to work when both $A$
and $B$ are themselves conjunctions, with the $ls'=\dots$
as part of the same conjunction, located at some arbitrary position.
We can break the problem into two parts.
First we do a top-level pattern match
to see that we have a top-level sequential composition
of two conjunctions,
then we use a function that will check both conjunction predicate-lists
for the existence of a $ls'=\dots$ component,
and that everything else also satisfies the requirements regarding
the occurrence, or otherwise of $ls$ or $ls'$:
\begin{code}
reduceUTCP d pr@(_,Comp "Seq" [ (_,Comp "And" pAs)
                              , (_,Comp "And" pBs)])
 = case isSafeLSDash d ls' ls' pAs of -- no ls' in rest of pAs
    Nothing -> lred "" pr
    Just (_,restA) ->
     case isSafeLSDash d ls' ls pBs of -- no ls in rest of pBs
      Nothing -> lred "" pr
      Just (eqB,restB)
       -> lred "ls'-cleanup" $   -- build RHS
             comp "And" [ comp "Seq" [ bAnd restA
                                     , bAnd restB ]
                        , eqB ]
\end{code}
The function \texttt{isSafeLSDash}
is designed to
(i) locate the $ls'=e$ conjunct and check that its rhs is a ground expression;
(ii) check that none of the remaining conjuncts make use of the
`unwanted' version of the label-set variable ($ls$ or $ls'$);
and (iii), if all ok, return a pair
whose first component is the ($ls'=\dots$) equality,
and whose second is the list of other conjuncts.
To achieve (i) above,
we make use of two functions provided by the \texttt{CalcRecogniser} module:
\begin{code}
mtchNmdObsEqToConst :: Ord s => String -> Dict s -> Recogniser s
matchRecog :: (Ord s, Show s)
           => Recogniser s -> [MPred s]
           -> Maybe ([MPred s],(MPred s,[MPred s]),[MPred s])
\end{code}
where
\begin{code}
type Recogniser s = MPred s -> (Bool, [MPred s])
\end{code}
A recogniser is a function that takes a predicate
and if it ``recognises'' it, returns \texttt{(True, parts)},
where parts are the subcomponents of the predicate in some order.
The recogniser \texttt{mtchNmdObsEqtoConst v d} matches a predicate of the form
\texttt{Equal (Var v) k}, returning a list with both parts.
The function \texttt{matchRecog} takes a recogniser and list of predicates
and looks in the list for the first predicate to satisfy
the recogniser, returning a triple of the form
(before,satisyingPred,after).
If the recogniser succeeds,
we then check the validity of the expression,
and the absence of the unwanted variable from the
rest of the conjuncts --- using boolean function
 \texttt{dftlyNotInP} (``definitely not in $P$''),
 so handling task (ii) above.
\begin{code}
isSafeLSDash d theLS unwanted prs
 = case matchRecog (mtchNmdObsEqToConst theLS d) prs of
    Nothing -> Nothing
    Just (pre,(eq@(_,Equal _ k),_),post) ->
     if notGround d k
      then Nothing
      else if all (dftlyNotInP d unwanted) rest
       then Just (eq,rest)
       else Nothing
     where rest = pre++post
\end{code}

\subsection{Conditional Reductions}

To avoid having to support a wide range of expression-related theories,
we provide a conditional reducer, that computes
a number of alternative outcomes, each guarded by some predicate
that is hard to evaluate.
The user elects which one to use by checking the conditions manually.
We define a function, similar to reduceUTCP,
that contains a series of patterns for each conditional reduction law.
\begin{code}
creduceUTCP :: (Show s, Ord s) => CDictRWFun s
\end{code}
Provided that $\vec x \subseteq in\alpha P$
 (which in this case is $\setof{s,ls}$):
\RLEQNS{
   c[\vec e/\vec x]
   &\implies&
   (c * P)[\vec e/\vec x] = P[\vec e/\vec x] ; c * P
\\ \lnot c[\vec e/\vec x]
   &\implies&
   (c * P)[\vec e/\vec x] = \Skip[\vec e/\vec x]
}
\begin{code}
creduceUTCP d (_,PSub w@(_,Comp "Iter" [c,p]) sub)
 | isCondition c        -- true if expr c is a UTP 'condition'
   && beforeSub d sub   -- true if subst-vars are all undashed
 = lcred "loop-substn" [ctrue,cfalse]
 where
   csub = psub c sub            --  psub builds a substitution
   ctrue  = ( psimp d csub          -- psimp runs a simplifier
            , bSeq (psub p sub) w )
   cfalse = ( psimp d $ bNot csub
            , psub bSkip sub )
\end{code}
If this succeeds, the user is presented with two options,
each of the form (side-condition, outcome)
The user can then identify which of those side-conditions is true,
resulting in the appropriate outcome.

We make these two reduction functions ``known'' to the calculator
by adding them into a dictionary.
\begin{code}
lawsUTCPDict
 = makeDict [("laws", LawEntry [reduceUTCP] [creduceUTCP] [])]
\end{code}
We then can take a number of partial dictionaries and use various
dictionary functions,
defined in \texttt{CalcPredicates}, to merge them together.
\begin{code}
dictUTCP = foldl1 dictMrg [ alfUTCPDict, ..., lawsUTCPDict]
\end{code}

\subsubsection{Law Entries}~

\begin{code}
Entry s
  = ...
  | LawEntry  { reduce :: [DictRWFun s]
              , creduce :: [CDictRWFun s]
              , unroll :: [String -> DictRWFun s] }
\end{code}
Law entries are currently only associated with one key,
namely the string \texttt{"laws"},
which is what the calculator uses to find such entries.
There are three parts,
each consisting of lists of functions.
These lists are intended to be be applied in order to the
``current predicate'', until either one succeeds,
or none do.
All the functions take a dictionary
and (marked) predicate as arguments,
and return either
an single-outcome indicator (\texttt{reduce},\texttt{unroll}):
\begin{code}
type DictRWFun s = Dict s -> RWFun s
type RWFun s = MPred s -> RWResult s
type RWResult s = (String, MPred s)
\end{code}
or a conditional  multiple-outcome result (\texttt{creduce}).
\begin{code}
type CDictRWFun s = Dict s -> CRWFun s
type CRWFun s = MPred s -> CRWResult s
type CRWResult s = (String, [(Pred s, MPred s)])
\end{code}
\begin{description}
  \item[\texttt{reduce}]
     is a list of reduction laws,
     to be tried out when a reduction step is invoked by the user.
  \item[\texttt{creduce}]
    is a list of conditional reduction laws,
    which have multiple outcomes dependent on a side-condition.
    Rather than try to resolve conditions automatically,
    we prefer to let the user choose the appropriate outcome.
  \item[\texttt{unroll}]
    is a list of loop-unrolling functions.
    The extra string argument is to give the user finer control
    of how much unrolling is done, and how it is presented.
\end{description}

\section{Experience}\label{sec:Experience}

\section{Conclusions}\label{sec:Conc}

We have presented a description of a calculator written in Haskell,
that allows the encoding of an UTP theory under development,
in order to be able to rapidly perform test calculations.
This helps to check that predictions of the theory match expectations.
The tool was not designed to be a complete and sound theory development
system,
but instead to act as a rapid-prototype tool to help smoke out problems
with a developing theory. This approach relies on the  developer
to be checking and scrutinising everything.

\subsection{Costs vs. Benefits}

As far as the development of the UTCP theory is concerned,
the costs of developing and customising the calculator
have been well compensated for by the benefits we encountered.
This also applies to ongoing work to develop a fully compositional UTP theory
of shared-state concurrency that does not require $run$.
We note a few observations based on our experience
using the calculator.


   The ``first-come, first-served'' approach
   used by the calculator is surprisingly effective.
   We support a system of equational reasoning
   where reductions and definitions replace predicates with ones that
   are equal.
   In effect this means that the order in which most of these steps
   take place is immaterial.
   Some care needs to be taken when several rules apply to one construct,
   but this can be managed by re-arranging the order in which various
   patterns and their side-conditions can be checked.


   The main idea in using the calculator
   is to find a suitable collection of patterns,
   in the right order,
   to be most effective in performing calculations.
   The best way to determine this is to start with none,
   run the calculator and when it stalls
   (no change is happening for any command),
   see what law would help make progress, and encode it.
   This leads to an unexpected side-effect of this calculator,
   in that we learnt what laws we needed,
   rather than what we thought we would need.




    Effective use of the calculator results in an inexorable
    push towards algebras. By this we do not mean the Kleene algebras,
    or similar, that might characterise the language being formalised.
    Rather we mean that the most effective use of the calculator results
    when we define predicate functions that encapsulate some simple
    behaviour, and demonstrate, by proofs done without the calculator,
    some laws they obey,
    particularly with respect to sequential composition.
    In fact, one of the `algebras' under development for the fully
    compositional theory, is so effective,
    that many of the test calculations
    can actually be done manually.
    However some, most notably involving parallel composition,
    still require the calculator in order to be feasible.

\subsection{Correctness}

    An issue that can be raised,
    given the customisation and lack of soundness guarantees,
    is how well has the calculator been tested?
    The answer is basically that the process of using it ensures
    that the whole system is comprehensively tested.
    This is because calculations fail repeatedly.
    Such failures lead to a post-mortem to identify the reason.
    Early in the calculator development,
    the reason would be traced to a bug in the calculator infrastructure.
    The next phase has failures that can be attributed
    to bugs in the encoding of laws in Haskell,
    or poor ordering in the dictionary.
    What makes the above tolerable is that the time taken to identify
    and fix each code problem is relatively short,
    often a matter of five to ten minutes.
    The final phase is where calculation failures arise because of errors
    in the proposed theory---this is the real payback,
    as this is the intended purpose of the tool.
    The outcome of all of this iterative development
    is a high degree of confidence in the end result.
    In the author's experience,
    the cost of all the above failures
    is considerably outweighed by the cost of
    trying to do the check calculations manually.

There are no guarantees of soundness.
But working on any theory by hand faces exactly the same issues
--- a proof or calculation by hand always raises the issue
of the correctness of a law, or the validity of a ``proof-step''
that is really a number of simpler steps all rolled into one.
In either case, by hand or by calculator,
the theory developer has a responsibility to carefully check every line.
This is one reason why so much effort was put into pretty-printing
and marking.
The calculator's real benefit, and\emph{ main design purpose},
is the ease with which
it can produce a calculation and transcript.


In effect, this UTP Calculator is a tool that assists
with the validation of UTP semantic definitions,
and is designed for use by someone with expertise
in UTP theory building,
and a good working knowledge of Haskell.

%A key lesson learnt during the development
%of both the calculator, and the UTCP theory,
%is the value of the agile approach.
%By focussing development of both on what
%was the immediate need at any given moment,
%we found that the calculator, and its dictionaries,
%were prevented from excessive bloat
%e.g., coding up a common, useful, obvious law,
%that actually wasn't needed.
%What was important was
%finding the ``sweet spot'' between the use of definition
%expansion, and hard-coded laws based on by-hand proofs.

\subsection{Future Work}

We plan a formal release of this calculator as a Haskell package.
A key part of this would be comprehensive
user documentation of the key parts of the calculator API,
the standard built-in dictionaries,
as well as a complete worked example of a theory encoding.
There are many enhancements that are also being considered,
that include better transcript rendering options
(e.g. \LaTeX) or ways to customise the REPL
(e.g. always do a simplify step after any other REPL command).
Also of interest would be finding
a way of connecting the calculator
to either the \UTP2 theorem-prover\cite{DBLP:conf/utp/Butterfield10}
or the Isabelle/UTP encoding\cite{DBLP:conf/utp/FosterZW14}
in order to be able to validate the dictionary entries.
All the code described here is available online
at
\\\url{https://bitbucket.org/andrewbutterfield/utp-calculator.git}
as Literate Haskell Script files (\texttt{.lhs})
in the \texttt{src} sub-directory.


%\subsection{Before we dive in \dots}

The UTP Calculator is implemented as a series
of Haskell modules,
which are broken into two groups:
\begin{description}
  \item[Infrastructure]
    are modules that implement the calculator mechanics,
    pretty-printing, etc.
    These include \texttt{PrettyPrint},
    and all modules with names starting with \texttt{Calc}.
  \item[Builtin Theories]
    are pre-defined theory modules that cover standard logic,
    whose names start with \texttt{Std}, and modules that cover ``standard''
    UTP, whose names start with \texttt{StdUTP}.
    These theory modules typically come in threes, covering
    \texttt{Predicates}, \texttt{Precedences} and \texttt{Laws}.
\end{description}
All the Haskell modules are found in the \texttt{src} directory
of the repository, with a \texttt{.lhs} file extension
(e.g., \texttt{CalcTypes.lhs}).

\subsection{UTCP Syntax}

We start by defining the syntax of our language
\[
   P ::= A \mid P \pseq P \mid P \parallel P \mid P \pcond P \mid \piter P
\]
and assign them pretty printing precedences,
so they work well with the definitions in modules
\texttt{StdPrecedences} and \texttt{StdUTPPrecedences}.
\begin{code}
precPCond = 5 + precSpacer  1
precPPar  = 5 + precSpacer  2
precPSeq  = 5 + precSpacer  3
precPIter = 5 + precSpacer  6
\end{code}
The \texttt{precSpacer} function is used to leave space between precedence
values so that new constructs can be given an in-between spacing,
if required. Currently it returns its argument times 1.


\subsection{UTCP Alphabet}

As already stated, the theory alphabet is $s,s',ls,ls',g,in,out$.
We declare each as a variable in our expression notation,
noting that Haskell allows identifiers to contain dashes,
which proves very convenient:
\begin{code}
s' = Var "s'"
\end{code}
Note, here \texttt{s'} is a Haskell variable of type \texttt{Expr},
while \texttt{"s'"} is a Haskell literal value of type \texttt{String}.

We have two ways to classify UTP observation variables.
Along one axis we distinguish observations of program variable
values (``script'' variables, e.g. $s$, $s'$) from those that record other
observations such as termination/stability,
or traces/refusals (``model'' variables, e.g. $ls$, $ls'$).
On the other axis we distinguish observations
that are dynamic, whose values change as the program runs
(e.g. $s$, $ls$ with $s'$ and $ls'$)
from those that are static,
unchanged during program execution (e.g. $g$, $in$ and $out$).
We have pre-defined names for these categories,
and a function \texttt{stdAlfDictGen} that
builds all the appropriate entries
given three lists of script, dynamic model and static variable strings.
We also declare that the predicate variables $A$, $B$ and $C$
will refer to atomic state-changes,
and so only have alphabet $\setof{s,s'}$.
\begin{code}
alfUTCPDict
 = dictMrg dictAlpha dictAtomic
 where
   dictAlpha = stdAlfDictGen ["s"] ["ls"] ["g","in","out"]
   dictAtomic = makeDict [ pvarEntry "A" ss'
                         , pvarEntry "B" ss'
                         , pvarEntry "C" ss' ]
   ss' = ["s", "s'"]
\end{code}
(See modules
\texttt{CalcAlphabets}
, \texttt{CalcPredicates}
, \texttt{StdPredicates}.)

\subsection{UTCP Expressions}

We have sets of labels
so we need a way to implement set-expressions.
To avoid long set expressions a number of shorthands are desirable,
so that a singleton set $\setof x$ is rendered as $x$
and the very common idiom $S \subseteq ls$
is rendered as $ls(S)$,
so that for example, $ls(in)$ is short for $\setof{in} \subseteq ls$.
So we encode a set construct as follows:
\begin{code}
set = App "set"                             -- set constructor
showSet d [elm] = edshow d elm      -- drop {,} from singleton
showSet d elms = "{" ++ dlshow d "," elms ++ "}"
\end{code}
We also define an equality tester for sets,
that gets the two element-lists
\begin{code}
eqSet d es1 es2
 = let ns1 = nub $ sort $ es1                -- normalise sets
       ns2 = nub $ sort $ es2
   in if all (isGround d) (ns1++ns2)
      then Just (ns1==ns2)
      else Nothing
\end{code}
The predicate \texttt{isGround} checks to see if an expression has no
dynamic variables.
For the purposes of this theory at least,
we know we can treat these expressions as values.
This is a common feature of encoding theories for this calculator%
---%
knowing when a particular simplification makes sense.
The dictionary entry for the set construct then looks like
\begin{code}
ExprEntry ["*"] showSet none eqSet
\end{code}
where we permit any substitutions directly on the elements,
and we use the special builtin function \texttt{none}
as an evaluator that does not make any changes,
since we regard these sets as evaluated, in this theory.

Similar tricks are used to code a very compact rendering
of a mechanism that involves unique label generators
that can also be split, so that an expression like
\[
 \pi_2(new(\pi_1(new(\pi_1(split(\pi_1(new(g))))))))
\]
can be displayed as $\ell_{g:1:}$, or,
within the calculator, as \texttt{lg:1:} .


\subsection{Coding Atomic Semantics}

Formally, using our shorthand notations, we define atomic behaviour as:
\[
    \A(A) \defs ls(in) \land A \land ls'=ls\ominus(in,out)
\]
where $A$ is a predicate whose alphabet is restricted to $s$ and $s'$.
We code this up as follows:
\begin{code}
patm mpr = comp "A" [mpr] -- we assume mpr has only s, s' free

defnAtomic d [a] = ldefn shPAtm $ mkAnd [lsin,a,ls'eqlsinout]

inp = Var "in" -- 'in' is a Haskell keyword
out = Var "out"
lsin = atm $ App "subset" [inp,ls]
lsinout = App "sswap" [ls,inp,out]
ls'eqlsinout = equal ls' lsinout

patmEntry
 = ( nPAtm
   , PredEntry [] ppPAtm [] defnAtomic (pNoChg nPAtm) )
\end{code}
Here \texttt{atm} lifts an expression to a marked predicate,
while \texttt{"sswap"} names the ternary operation $\_\ominus(\_,\_)$,
and \texttt{equal} is the marked form of \texttt{Equal}.


We won't show the encoding of the composite constructs,
or a predicate transformer called $run$ that actually
enables us to symbolically `execute' our semantics.
We will show how the \texttt{pprint} entry for
sequential composition in UTCP is defined,
just to show how easy support for infix notation is.
\begin{code}
ppPSeq d ms p [mpr1,mpr2]
 = paren p precPSeq -- parenthesise if precedence requires it
     $ ppopen  (pad ";;") [ mshowp d ms precPSeq mpr1
                          , mshowp d ms precPSeq mpr2 ]
\end{code}
Here \texttt{pad} puts spaces around its argument,
while \texttt{ppopen} uses its first argument as a
separator between all the elements of its second list argument.
The function \texttt{mshowp} is the top-level predicate printer.


\subsection{Coding UTCP Laws}

The definition of the semantics of the UTCP language
constructs, and of $run$,
make use of the (almost) standard notions of skip,
sequential composition
and iteration in UTP.
The versions used here are slightly non-standard because we have
non-homogeneous relations,
where the static parameters have no dashed counterparts.
In essence we ignore the parameters as far as flow-of-control is concerned:
\RLEQNS{
   \Skip &\defs& s'=s \land ls'=ls
\\ P ; Q
   &\defs&
   \exists s_m,ls_m @
     P[s_m,ls_m/s',ls']
     \land
     Q[s_m,ls_m/s,ls]
\\ c * P &\defs& \mu L @ (P ; L) \cond c \Skip
\\ P \cond c Q &\defs& c \land P \lor \lnot c \land Q
}
Here, the definition of $\cond\_$ is entirely standard, of course.

What is key here though,
is realising that we do not want to define the constructs
as above and use them directly, as it involves
quantifiers and explicit recursion,
both of which would introduce considerable complexity to the calculator.
Instead, we encode useful laws that they satisfy,
that do not require their definitions to be expanded.
Such laws might include the following:
\RLEQNS{
  \Skip \seq\ P & {} = P = {} & P \seq \Skip
\\ c * P &=& (P \seq c* P ) \cond c \Skip
\\ (c * P)[e/x] &=& P[e/x] \seq c * P, \qquad if c[e/x]
\\ (c * P)[e/x] &=& \Skip[e/x] , \qquad if \lnot c[e/x]
}
These laws need to be proven by hand (carefully),
by the theory developer, and then encoded into Haskell
(equally carefully), as we are about to describe.

We can easily give a definition of $\Skip$,
which is worth expanding.
\RLEQNS{
   \Skip &\defs& s'=s \land ls'=ls
}
\begin{code}
defnUTCPII = mkAnd[ equal s' s, equal ls' ls ]
\end{code}

For more complex laws,
the idea is that we pattern-match on predicate syntax
to see if a law is applicable (we have its lefthand-side),
and if so,
we then build an appropriate instance of the righthand-side.
The plan is that we gather all these pattern/outcome pairs
in one function definition,
which will try them in order.
This is a direct match for how Haskell pattern-matching works.
So for UTCP we have a function called \texttt{reduceUTCP},
structured as follows:
\begin{code}
reduceUTCP :: (Show s, Ord s) => DictRWFun s
reduceUTCP (...1st law pattern...) = 1st outcome
reduceUTCP (...2nd law pattern...) = 2nd outcome
...
reduceUTCP d mpr = lred "" mpr  -- catch-all at end, no change
\end{code}
The last clause matches any predicate
and simply returns it with a null string,
indicating no change took place.
The main idea is find a suitable collection of patterns,
in the right order,
to be most effective in performing calculations.
The best way to determine this is start with none,
run the calculator and when it stalls
(no change is happening for any command),
see what law would help make progress, and encode it.
This is the essence of the agile approach to theory calculator development.

A simple example of such a pattern is the following encoding
of $\Skip;P = P$ :
\begin{code}
reduceUTCP d
 (_,Comp "Seq" [(_,Comp "Skip" []), mpr]) = lred ";-lunit" mpr
\end{code}
The second argument has type marked-predicate (\texttt{MPred})
which is a marking/predicate pair.
We are not interested in the markings
so we use the wildcard pattern '\verb"_"'
for the first pair component.
The sub-pattern in the second pair component,
\verb'Comp "Seq" [(_,Comp "Skip" []), mpr])',
matches a composite called ``Seq'',
with a argument list containing two (marked) predicates.
The first argument predicate pattern \verb'(_,Comp "Skip" [])'
matches a ``Skip'' composite with no further subarguments.
The second argument pattern \verb'mpr' matches an arbitrary predicate
($P$ in the law above).
The righthand-side returns the application \verb'lred ";-lunit" mpr'
which simply constructs a string/predicate pair,
with the string being a justification note that says a reduction-step
using a law called ``$;$-lunit'' was applied.



\subsection{UTCP Recognisers}

Some laws require matching that is a bit more sophisticated.
For example,
consider a useful reduction for tidying up at the end,
assuming that $ls' \notin A$ and $ls \notin B$, and both $k$ and $h$
are ground:
\[
   (A \land ls'=k) ; (B \land ls'= h)
   \quad\equiv\quad
   (A;B) \land ls'=h
   \qquad \elabel{$ls'$-cleanup}
\]
However, we want this law to work when both $A$
and $B$ are themselves conjunctions, with the $ls'=\dots$
as part of the same conjunction, located at some arbitrary position.
We can break the problem into two parts.
First we do a top-level pattern match
to see that we have a top-level sequential composition
of two conjunctions,
then we use a function that will check both conjunction predicate-lists
for the existence of a $ls'=\dots$ component,
and that everything else also satisfies the requirements regarding
the occurrence, or otherwise of $ls$ or $ls'$:
\begin{code}
reduceUTCP d pr@(_,Comp "Seq" [ (_,Comp "And" pAs)
                              , (_,Comp "And" pBs)])
 = case isSafeLSDash d ls' ls' pAs of -- no ls' in rest of pAs
    Nothing -> lred "" pr
    Just (_,restA) ->
     case isSafeLSDash d ls' ls pBs of -- no ls in rest of pBs
      Nothing -> lred "" pr
      Just (eqB,restB)
       -> lred "ls'-cleanup" $   -- build RHS
             comp "And" [ comp "Seq" [ bAnd restA
                                     , bAnd restB ]
                        , eqB ]
\end{code}
The function \texttt{isSafeLSDash}
is designed to
(i) locate the $ls'=e$ conjunct and check that its rhs is a ground expression;
(ii) check that none of the remaining conjuncts make use of the
`unwanted' version of the label-set variable ($ls$ or $ls'$);
and (iii), if all ok, return a pair
whose first component is the ($ls'=\dots$) equality,
and whose second is the list of other conjuncts.
To achieve (i) above,
we make use of two functions provided by the \texttt{CalcRecogniser} module:
\begin{code}
mtchNmdObsEqToConst :: Ord s => String -> Dict s -> Recogniser s
matchRecog :: (Ord s, Show s)
           => Recogniser s -> [MPred s]
           -> Maybe ([MPred s],(MPred s,[MPred s]),[MPred s])
\end{code}
where
\begin{code}
type Recogniser s = MPred s -> (Bool, [MPred s])
\end{code}
A recogniser is a function that takes a predicate
and if it ``recognises'' it, returns \texttt{(True, parts)},
where parts are the subcomponents of the predicate in some order.
The recogniser \texttt{mtchNmdObsEqtoConst v d} matches a predicate of the form
\texttt{Equal (Var v) k}, returning a list with both parts.
The function \texttt{matchRecog} takes a recogniser and list of predicates
and looks in the list for the first predicate to satisfy
the recogniser, returning a triple of the form
(before,satisyingPred,after).
If the recogniser succeeds,
we then check the validity of the expression,
and the absence of the unwanted variable from the
rest of the conjuncts --- using boolean function
 \texttt{dftlyNotInP} (``definitely not in $P$''),
 so handling task (ii) above.
\begin{code}
isSafeLSDash d theLS unwanted prs
 = case matchRecog (mtchNmdObsEqToConst theLS d) prs of
    Nothing -> Nothing
    Just (pre,(eq@(_,Equal _ k),_),post) ->
     if notGround d k
      then Nothing
      else if all (dftlyNotInP d unwanted) rest
       then Just (eq,rest)
       else Nothing
     where rest = pre++post
\end{code}

\subsection{Conditional Reductions}

To avoid having to support a wide range of expression-related theories,
we provide a conditional reducer, that computes
a number of alternative outcomes, each guarded by some predicate
that is hard to evaluate.
The user elects which one to use by checking the conditions manually.
We define a function, similar to reduceUTCP,
that contains a series of patterns for each conditional reduction law.
\begin{code}
creduceUTCP :: (Show s, Ord s) => CDictRWFun s
\end{code}
Provided that $\vec x \subseteq in\alpha P$
 (which in this case is $\setof{s,ls}$):
\RLEQNS{
   c[\vec e/\vec x]
   &\implies&
   (c * P)[\vec e/\vec x] = P[\vec e/\vec x] ; c * P
\\ \lnot c[\vec e/\vec x]
   &\implies&
   (c * P)[\vec e/\vec x] = \Skip[\vec e/\vec x]
}
\begin{code}
creduceUTCP d (_,PSub w@(_,Comp "Iter" [c,p]) sub)
 | isCondition c        -- true if expr c is a UTP 'condition'
   && beforeSub d sub   -- true if subst-vars are all undashed
 = lcred "loop-substn" [ctrue,cfalse]
 where
   csub = psub c sub            --  psub builds a substitution
   ctrue  = ( psimp d csub          -- psimp runs a simplifier
            , bSeq (psub p sub) w )
   cfalse = ( psimp d $ bNot csub
            , psub bSkip sub )
\end{code}
If this succeeds, the user is presented with two options,
each of the form (side-condition, outcome)
The user can then identify which of those side-conditions is true,
resulting in the appropriate outcome.

We make these two reduction functions ``known'' to the calculator
by adding them into a dictionary.
\begin{code}
lawsUTCPDict
 = makeDict [("laws", LawEntry [reduceUTCP] [creduceUTCP] [])]
\end{code}
We then can take a number of partial dictionaries and use various
dictionary functions,
defined in \texttt{CalcPredicates}, to merge them together.
\begin{code}
dictUTCP = foldl1 dictMrg [ alfUTCPDict, ..., lawsUTCPDict]
\end{code}


\bibliographystyle{splncs03}
\bibliography{UTPCalc-UTP2016-MAIN}

\appendix
%\section{IntroductionX}\label{sec:IntroX}

\section{Design \& ArchitectureX}\label{sec:DesignX}



\subsubsection{Alphabet Entries}~

\begin{code}
Entry s
  = ...
  | AlfEntry  { avars :: [String]}
\end{code}
Most simply put, an alphabet is simply a set of variables.
In any UTP theory we typically have well-defined alphabets,
often with particular subsets of interest,
such as all the ``before'' observations (undashed variables),
or ``after'' observations (dashed variables).
We give these subsets standardised names,
and use the dictionary to list the relevant variables.

\section{Building TheoriesX}\label{sec:TheorisingX}

%Sometimes learning by doing is the best approach,
%so here we sketch out how a theory like UTCP
%is captured in way that this calculator
%can be usefully used to perform validating calculations.

We now present extracts from a theory built using this
approach to illustrate how the calculator is used.
As already stated,
the use of this calculator does require some expertise
in functional programming with languages that support
pattern-matching (e.g. Haskell, ML).
We do this by giving an overview of the Haskell encoding
of the UTCP theory
that was the motivation for this work.
The emphasis here is on how to encode a UTP theory for use with this
calculator,
rather than trying to explain the theory itself, or its design motivation.


\subsection{UTCP Syntax}

We start by defining the syntax of our language
\[
   P ::= A \mid P \pseq P \mid P \parallel P \mid P \pcond P \mid \piter P
\]
and assign them pretty printing precedences,
so they work well with the definitions in modules
\texttt{StdPrecedences} and \texttt{StdUTPPrecedences}.
\begin{code}
precPCond = 5 + precSpacer  1
precPPar  = 5 + precSpacer  2
precPSeq  = 5 + precSpacer  3
precPIter = 5 + precSpacer  6
\end{code}
The \texttt{precSpacer} function is used to leave space between precedence
values so that new constructs can be given an in-between spacing,
if required. Currently it returns its argument times 1.


\subsection{UTCP Alphabet}

As already stated, the theory alphabet is $s,s',ls,ls',g,in,out$.
We declare each as a variable in our expression notation,
noting that Haskell allows identifiers to contain dashes,
which proves very convenient:
\begin{code}
s' = Var "s'"
\end{code}
Note, here \texttt{s'} is a Haskell variable of type \texttt{Expr},
while \texttt{"s'"} is a Haskell literal value of type \texttt{String}.

We have two ways to classify UTP observation variables.
Along one axis we distinguish observations of program variable
values (``script'' variables, e.g. $s$, $s'$) from those that record other
observations such as termination/stability,
or traces/refusals (``model'' variables, e.g. $ls$, $ls'$).
On the other axis we distinguish observations
that are dynamic, whose values change as the program runs
(e.g. $s$, $ls$ with $s'$ and $ls'$)
from those that are static,
unchanged during program execution (e.g. $g$, $in$ and $out$).
We have pre-defined names for these categories,
and a function \texttt{stdAlfDictGen} that
builds all the appropriate entries
given three lists of script, dynamic model and static variable strings.
We also declare that the predicate variables $A$, $B$ and $C$
will refer to atomic state-changes,
and so only have alphabet $\setof{s,s'}$.
\begin{code}
alfUTCPDict
 = dictMrg dictAlpha dictAtomic
 where
   dictAlpha = stdAlfDictGen ["s"] ["ls"] ["g","in","out"]
   dictAtomic = makeDict [ pvarEntry "A" ss'
                         , pvarEntry "B" ss'
                         , pvarEntry "C" ss' ]
   ss' = ["s", "s'"]
\end{code}
(See modules
\texttt{CalcAlphabets}
, \texttt{CalcPredicates}
, \texttt{StdPredicates}.)

\subsection{UTCP Expressions}

We have sets of labels
so we need a way to implement set-expressions.
To avoid long set expressions a number of shorthands are desirable,
so that a singleton set $\setof x$ is rendered as $x$
and the very common idiom $S \subseteq ls$
is rendered as $ls(S)$,
so that for example, $ls(in)$ is short for $\setof{in} \subseteq ls$.
So we encode a set construct as follows:
\begin{code}
set = App "set"                             -- set constructor
showSet d [elm] = edshow d elm      -- drop {,} from singleton
showSet d elms = "{" ++ dlshow d "," elms ++ "}"
\end{code}
We also define an equality tester for sets,
that gets the two element-lists
\begin{code}
eqSet d es1 es2
 = let ns1 = nub $ sort $ es1                -- normalise sets
       ns2 = nub $ sort $ es2
   in if all (isGround d) (ns1++ns2)
      then Just (ns1==ns2)
      else Nothing
\end{code}
The predicate \texttt{isGround} checks to see if an expression has no
dynamic variables.
For the purposes of this theory at least,
we know we can treat these expressions as values.
This is a common feature of encoding theories for this calculator%
---%
knowing when a particular simplification makes sense.
The dictionary entry for the set construct then looks like
\begin{code}
ExprEntry ["*"] showSet none eqSet
\end{code}
where we permit any substitutions directly on the elements,
and we use the special builtin function \texttt{none}
as an evaluator that does not make any changes,
since we regard these sets as evaluated, in this theory.

Similar tricks are used to code a very compact rendering
of a mechanism that involves unique label generators
that can also be split, so that an expression like
\[
 \pi_2(new(\pi_1(new(\pi_1(split(\pi_1(new(g))))))))
\]
can be displayed as $\ell_{g:1:}$, or,
within the calculator, as \texttt{lg:1:} .


\subsection{Coding Atomic Semantics}

Formally, using our shorthand notations, we define atomic behaviour as:
\[
    \A(A) \defs ls(in) \land A \land ls'=ls\ominus(in,out)
\]
where $A$ is a predicate whose alphabet is restricted to $s$ and $s'$.
We code this up as follows:
\begin{code}
patm mpr = comp "A" [mpr] -- we assume mpr has only s, s' free

defnAtomic d [a] = ldefn shPAtm $ mkAnd [lsin,a,ls'eqlsinout]

inp = Var "in" -- 'in' is a Haskell keyword
out = Var "out"
lsin = atm $ App "subset" [inp,ls]
lsinout = App "sswap" [ls,inp,out]
ls'eqlsinout = equal ls' lsinout

patmEntry
 = ( nPAtm
   , PredEntry [] ppPAtm [] defnAtomic (pNoChg nPAtm) )
\end{code}
Here \texttt{atm} lifts an expression to a marked predicate,
while \texttt{"sswap"} names the ternary operation $\_\ominus(\_,\_)$,
and \texttt{equal} is the marked form of \texttt{Equal}.


We won't show the encoding of the composite constructs,
or a predicate transformer called $run$ that actually
enables us to symbolically `execute' our semantics.
We will show how the \texttt{pprint} entry for
sequential composition in UTCP is defined,
just to show how easy support for infix notation is.
\begin{code}
ppPSeq d ms p [mpr1,mpr2]
 = paren p precPSeq -- parenthesise if precedence requires it
     $ ppopen  (pad ";;") [ mshowp d ms precPSeq mpr1
                          , mshowp d ms precPSeq mpr2 ]
\end{code}
Here \texttt{pad} puts spaces around its argument,
while \texttt{ppopen} uses its first argument as a
separator between all the elements of its second list argument.
The function \texttt{mshowp} is the top-level predicate printer.


\subsection{Coding UTCP Laws}

The definition of the semantics of the UTCP language
constructs, and of $run$,
make use of the (almost) standard notions of skip,
sequential composition
and iteration in UTP.
The versions used here are slightly non-standard because we have
non-homogeneous relations,
where the static parameters have no dashed counterparts.
In essence we ignore the parameters as far as flow-of-control is concerned:
\RLEQNS{
   \Skip &\defs& s'=s \land ls'=ls
\\ P ; Q
   &\defs&
   \exists s_m,ls_m @
     P[s_m,ls_m/s',ls']
     \land
     Q[s_m,ls_m/s,ls]
\\ c * P &\defs& \mu L @ (P ; L) \cond c \Skip
\\ P \cond c Q &\defs& c \land P \lor \lnot c \land Q
}
Here, the definition of $\cond\_$ is entirely standard, of course.

What is key here though,
is realising that we do not want to define the constructs
as above and use them directly, as it involves
quantifiers and explicit recursion,
both of which would introduce considerable complexity to the calculator.
Instead, we encode useful laws that they satisfy,
that do not require their definitions to be expanded.
Such laws might include the following:
\RLEQNS{
  \Skip \seq\ P & {} = P = {} & P \seq \Skip
\\ c * P &=& (P \seq c* P ) \cond c \Skip
\\ (c * P)[e/x] &=& P[e/x] \seq c * P, \qquad if c[e/x]
\\ (c * P)[e/x] &=& \Skip[e/x] , \qquad if \lnot c[e/x]
}
These laws need to be proven by hand (carefully),
by the theory developer, and then encoded into Haskell
(equally carefully), as we are about to describe.

We can easily give a definition of $\Skip$,
which is worth expanding.
\RLEQNS{
   \Skip &\defs& s'=s \land ls'=ls
}
\begin{code}
defnUTCPII = mkAnd[ equal s' s, equal ls' ls ]
\end{code}

For more complex laws,
the idea is that we pattern-match on predicate syntax
to see if a law is applicable (we have its lefthand-side),
and if so,
we then build an appropriate instance of the righthand-side.
The plan is that we gather all these pattern/outcome pairs
in one function definition,
which will try them in order.
This is a direct match for how Haskell pattern-matching works.
So for UTCP we have a function called \texttt{reduceUTCP},
structured as follows:
\begin{code}
reduceUTCP :: (Show s, Ord s) => DictRWFun s
reduceUTCP (...1st law pattern...) = 1st outcome
reduceUTCP (...2nd law pattern...) = 2nd outcome
...
reduceUTCP d mpr = lred "" mpr  -- catch-all at end, no change
\end{code}
The last clause matches any predicate
and simply returns it with a null string,
indicating no change took place.
The main idea is find a suitable collection of patterns,
in the right order,
to be most effective in performing calculations.
The best way to determine this is start with none,
run the calculator and when it stalls
(no change is happening for any command),
see what law would help make progress, and encode it.
This is the essence of the agile approach to theory calculator development.

A simple example of such a pattern is the following encoding
of $\Skip;P = P$ :
\begin{code}
reduceUTCP d
 (_,Comp "Seq" [(_,Comp "Skip" []), mpr]) = lred ";-lunit" mpr
\end{code}
The second argument has type marked-predicate (\texttt{MPred})
which is a marking/predicate pair.
We are not interested in the markings
so we use the wildcard pattern '\verb"_"'
for the first pair component.
The sub-pattern in the second pair component,
\verb'Comp "Seq" [(_,Comp "Skip" []), mpr])',
matches a composite called ``Seq'',
with a argument list containing two (marked) predicates.
The first argument predicate pattern \verb'(_,Comp "Skip" [])'
matches a ``Skip'' composite with no further subarguments.
The second argument pattern \verb'mpr' matches an arbitrary predicate
($P$ in the law above).
The righthand-side returns the application \verb'lred ";-lunit" mpr'
which simply constructs a string/predicate pair,
with the string being a justification note that says a reduction-step
using a law called ``$;$-lunit'' was applied.



\subsection{UTCP Recognisers}

Some laws require matching that is a bit more sophisticated.
For example,
consider a useful reduction for tidying up at the end,
assuming that $ls' \notin A$ and $ls \notin B$, and both $k$ and $h$
are ground:
\[
   (A \land ls'=k) ; (B \land ls'= h)
   \quad\equiv\quad
   (A;B) \land ls'=h
   \qquad \elabel{$ls'$-cleanup}
\]
However, we want this law to work when both $A$
and $B$ are themselves conjunctions, with the $ls'=\dots$
as part of the same conjunction, located at some arbitrary position.
We can break the problem into two parts.
First we do a top-level pattern match
to see that we have a top-level sequential composition
of two conjunctions,
then we use a function that will check both conjunction predicate-lists
for the existence of a $ls'=\dots$ component,
and that everything else also satisfies the requirements regarding
the occurrence, or otherwise of $ls$ or $ls'$:
\begin{code}
reduceUTCP d pr@(_,Comp "Seq" [ (_,Comp "And" pAs)
                              , (_,Comp "And" pBs)])
 = case isSafeLSDash d ls' ls' pAs of -- no ls' in rest of pAs
    Nothing -> lred "" pr
    Just (_,restA) ->
     case isSafeLSDash d ls' ls pBs of -- no ls in rest of pBs
      Nothing -> lred "" pr
      Just (eqB,restB)
       -> lred "ls'-cleanup" $   -- build RHS
             comp "And" [ comp "Seq" [ bAnd restA
                                     , bAnd restB ]
                        , eqB ]
\end{code}
The function \texttt{isSafeLSDash}
is designed to
(i) locate the $ls'=e$ conjunct and check that its rhs is a ground expression;
(ii) check that none of the remaining conjuncts make use of the
`unwanted' version of the label-set variable ($ls$ or $ls'$);
and (iii), if all ok, return a pair
whose first component is the ($ls'=\dots$) equality,
and whose second is the list of other conjuncts.
To achieve (i) above,
we make use of two functions provided by the \texttt{CalcRecogniser} module:
\begin{code}
mtchNmdObsEqToConst :: Ord s => String -> Dict s -> Recogniser s
matchRecog :: (Ord s, Show s)
           => Recogniser s -> [MPred s]
           -> Maybe ([MPred s],(MPred s,[MPred s]),[MPred s])
\end{code}
where
\begin{code}
type Recogniser s = MPred s -> (Bool, [MPred s])
\end{code}
A recogniser is a function that takes a predicate
and if it ``recognises'' it, returns \texttt{(True, parts)},
where parts are the subcomponents of the predicate in some order.
The recogniser \texttt{mtchNmdObsEqtoConst v d} matches a predicate of the form
\texttt{Equal (Var v) k}, returning a list with both parts.
The function \texttt{matchRecog} takes a recogniser and list of predicates
and looks in the list for the first predicate to satisfy
the recogniser, returning a triple of the form
(before,satisyingPred,after).
If the recogniser succeeds,
we then check the validity of the expression,
and the absence of the unwanted variable from the
rest of the conjuncts --- using boolean function
 \texttt{dftlyNotInP} (``definitely not in $P$''),
 so handling task (ii) above.
\begin{code}
isSafeLSDash d theLS unwanted prs
 = case matchRecog (mtchNmdObsEqToConst theLS d) prs of
    Nothing -> Nothing
    Just (pre,(eq@(_,Equal _ k),_),post) ->
     if notGround d k
      then Nothing
      else if all (dftlyNotInP d unwanted) rest
       then Just (eq,rest)
       else Nothing
     where rest = pre++post
\end{code}

\subsection{Conditional Reductions}

To avoid having to support a wide range of expression-related theories,
we provide a conditional reducer, that computes
a number of alternative outcomes, each guarded by some predicate
that is hard to evaluate.
The user elects which one to use by checking the conditions manually.
We define a function, similar to reduceUTCP,
that contains a series of patterns for each conditional reduction law.
\begin{code}
creduceUTCP :: (Show s, Ord s) => CDictRWFun s
\end{code}
Provided that $\vec x \subseteq in\alpha P$
 (which in this case is $\setof{s,ls}$):
\RLEQNS{
   c[\vec e/\vec x]
   &\implies&
   (c * P)[\vec e/\vec x] = P[\vec e/\vec x] ; c * P
\\ \lnot c[\vec e/\vec x]
   &\implies&
   (c * P)[\vec e/\vec x] = \Skip[\vec e/\vec x]
}
\begin{code}
creduceUTCP d (_,PSub w@(_,Comp "Iter" [c,p]) sub)
 | isCondition c        -- true if expr c is a UTP 'condition'
   && beforeSub d sub   -- true if subst-vars are all undashed
 = lcred "loop-substn" [ctrue,cfalse]
 where
   csub = psub c sub            --  psub builds a substitution
   ctrue  = ( psimp d csub          -- psimp runs a simplifier
            , bSeq (psub p sub) w )
   cfalse = ( psimp d $ bNot csub
            , psub bSkip sub )
\end{code}
If this succeeds, the user is presented with two options,
each of the form (side-condition, outcome)
The user can then identify which of those side-conditions is true,
resulting in the appropriate outcome.

We make these two reduction functions ``known'' to the calculator
by adding them into a dictionary.
\begin{code}
lawsUTCPDict
 = makeDict [("laws", LawEntry [reduceUTCP] [creduceUTCP] [])]
\end{code}
We then can take a number of partial dictionaries and use various
dictionary functions,
defined in \texttt{CalcPredicates}, to merge them together.
\begin{code}
dictUTCP = foldl1 dictMrg [ alfUTCPDict, ..., lawsUTCPDict]
\end{code}


\section{Implementation}\label{sec:Impl}

\subsection{Before we dive in \dots}

The UTP Calculator is implemented as a series
of Haskell modules,
which are broken into two groups:
\begin{description}
  \item[Infrastructure]
    are modules that implement the calculator mechanics,
    pretty-printing, etc.
    These include \texttt{PrettyPrint},
    and all modules with names starting with \texttt{Calc}.
  \item[Builtin Theories]
    are pre-defined theory modules that cover standard logic,
    whose names start with \texttt{Std}, and modules that cover ``standard''
    UTP, whose names start with \texttt{StdUTP}.
    These theory modules typically come in threes, covering
    \texttt{Predicates}, \texttt{Precedences} and \texttt{Laws}.
\end{description}
All the Haskell modules are found in the \texttt{src} directory
of the repository, with a \texttt{.lhs} file extension
(e.g., \texttt{CalcTypes.lhs}).

\subsection{Development Process}\label{ssec:development}

A very early decision was made to adhere to Agile Software development
principles \cite{Fowl01a}
in developing this calculator
(to the extent possible given that the roles of Software Engineer, Scrum Manager
and Customer were all rolled into one).
In particular we stuck close to the YAGNI (``Ya Ain't Gonna Need It'') principle%
\footnote{More formally, ``Simplicity---the art of maximizing the amount
of work not done---is essential.''}
which requires us to only write software for a function
that is required at that time.
This does not prevent advanced design planning but does keep
the development focussed on immediate needs.
So initially the focus was on being able to use the calculator
to expand the UTP definition of an atomic action.
Once that was working, then the focus shifted to additional code to
support the calculation of the sequential composition of two atomic actions,
and so on.
For example, the feature to take a final calculation and output it to a file
was only developed when this paper was being written,
because there was no need for it until this point.

\subsection{Software Architecture}\label{ssec:architecture}

All the code described here is available online
at
\\\url{https://bitbucket.org/andrewbutterfield/utp-calculator.git}
as Literate Haskell Script files (\texttt{.lhs})
in the \texttt{src} sub-directory.

Taking into account the repetitive nature of the calculations,
as mentioned at the end of \S\ref{ssec:plan},
and the need for shorthand notations we very rapidly converged
on four initial design decisions:
\begin{enumerate}
  \item No parsers! All calculation objects are written
  directly in Haskell.
  \item We would keep the expression and predicate datatype declarations
   very simple, with only equality being singled out.
  \item We would need to have a good way to pretty-print long predicates
    that made it easy to see their overall structure
  \item We would rely on a dictionary based system to
    make it easy to customise how specific constructs
    were to be handled.
\end{enumerate}
From our experience with the \UTP2 theorem-prover we also decided
the following regarding the calculation steps that would be supported:
\begin{itemize}
  \item
    We would not support full  propositional calculus
    or theories of numbers or sets.
    Instead we would support the use of hard-coded relevant laws,
    typically derived from  a handwritten proof.
  \item
    We would avoid, at all costs,
    any use of quantifiers or binding constructs.
  \item
    The calculator user interface would be very simple,
    supporting a few high level commands such as ``simplify''
    or ``reduce''.
    In particular,
    no facility would be provided for the user to identify
    the relevant sub-part of the current goal to which any operation
    should be applied.
\end{itemize}

We will now expand on some of the points above,
and in so doing expose some of the concrete architecture of the calculator code.

\section{Conclusions}\label{sec:Conc}

The calculator was used to develop the semantic definitions
for our UTP theory of concurrent programming (UTCP),
and has proved to be invaluable in checking that all
the definitions led to the correct outcomes.
It continues to be in active use with work in this area
that is extending and improving the theory.

However, there are no guarantees of soundness.
Great care has been taken to ensure that the pre-packaged
dictionaries are correct,
and similar care is needed to ensure that theory-specific
definitions are correct.
But working on a theory by hand faces exactly the same issues
--- a proof or calculation by hand always raises the issue
of the correctness of a law, or the validity of a ``proof-step''
that is really a number of simpler steps all rolled into one.
In either case, by hand or by calculator,
the theory developer has a responsibility to carefully check every line.
This is one reason why so much effort was put into pretty-printing
and marking.
The calculator's real benefit is the ease with which
it can produce a calculation and transcript.

In effect, this UTP Calculator is a tool that assists
with the validation of UTP semantic definitions,
and is designed for use by someone with expertise
in UTP theory building,
and a good working knowledge of Haskell.

A key lesson learnt during the development
of both the calculator, and the UTCP theory,
is the value of the agile approach.
By focussing development of both on what
was the immediate need at any given moment,
we found that the calculator, and its dictionaries,
were prevented from excessive bloat
e.g., coding up a common, useful, obvious law,
that actually wasn't needed.
What was important was
finding the ``sweet spot'' between the use of definition
expansion, and hard-coded laws based on by-hand proofs.

A very interesting observation was that the most effective use
of the calculator seems to involve defining shorthand
combinator like notations, and hand-proving key laws
about how they interact with key operators used
in semantics definitions, such as standard UTP sequential composition.
This drives the development of what is a bespoke semantics algebra.


There is very little work similar to this in the literature.
In \cite{Bird14} we find a calculator
for point-free equational proofs as a final case-study.
Also of interest is the discussion of deep/shallow embedding
in \cite{Gibbons:2014:FDS},
which suggests, that although our calculator is based on deep embedding,
the way it uses hand-coded laws from dictionaries
it more like shallow-embedding in character.


%\subsection{Future Work}\label{ssec:Future}

We plan a formal release of this calculator as a Haskell package.
A key part of this would be comprehensive
user documentation of the key parts of the calculator API,
the standard built-in dictionaries,
as well as a complete worked example of a theory encoding.
There are many enhancements that are also being considered,
that include better transcript rendering options
(e.g. \LaTeX) or ways to customise the REPL
(e.g. always do a simplify step after any other REPL command).
Also of interest would be finding
a way of connecting the calculator
to either the \UTP2 theorem-prover\cite{DBLP:conf/utp/Butterfield10}
or the Isabelle/UPT encoding\cite{DBLP:conf/utp/FosterZW14}
in order to be able to validate the dictionary entries.


\end{document}
