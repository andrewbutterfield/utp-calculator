\section{Laws}\label{sec:Laws}

In addition to the global simplifier and definition expansion facility,
we have three broad classes of laws that can be invoked from the REPL:
Reduce; Conditional Reduce; and Loop Unroll.

The way the latter three laws are applied is somewhat different
to the behaviour of either the simplifier or definition expansion.
Instead the reserved dictionary key \verb$"laws"$ is used
to lookup a special dictionary entry
\begin{verbatim}
 | LawEntry {
     reduce  :: [Dict s -> MPred s -> (String, MPred s)]
   , creduce :: [Dict s -> MPred s -> CRWResult s]
   , unroll  :: [String -> Dict s -> MPred s -> (String, MPred s)] }
type CRWResult s = ( String, [ (Pred s, MPred s) ] )
\end{verbatim}
A list of functions implementing the desired law class laws is then extracted.
The calculator then does a top-down, left-to-right search,
where at each point it tries all the laws in its list, in order.
It terminates at the point of first success.


\subsection{Reduce}

A reduce law is an equation of the form $P = Q$,
where we search for instances of $P$ and replace them with the
corresponding instance of $Q$.
The idea is that we pattern-match on predicate syntax
to see if a law is applicable (we have its lefthand-side),
and if so,
we then build an appropriate instance of the righthand-side.
The plan is that we gather all these pattern/outcome pairs
in one function definition,
which will try them in order.
This is a direct match for how Haskell pattern-matching works.
So for UTCP we have a function called \texttt{reduceUTCP},
structured as follows:
\begin{verbatim}
reduceUTCP (...1st law pattern...) = 1st outcome
reduceUTCP (...2nd law pattern...) = 2nd outcome
...
reduceUTCP d mpr = ("", mpr)  -- catch-all at end, no change
\end{verbatim}
The last clause matches any predicate
and simply returns it with a null string,
indicating no change took place.

A simple example of such a pattern is the following encoding
of $\Skip;P = P$ :
\begin{code}
reduceUTCP d
 (_,Comp "Seq" [(_,Comp "Skip" []), mpr]) = lred ";-lunit" mpr
\end{code}
The second argument has type marked-predicate (\texttt{MPred})
which is a marking/predicate pair.
We are not interested in the markings
so we use the wildcard pattern '\verb"_"'
for the first pair component.
The sub-pattern in the second pair component,
\verb'Comp "Seq" [(_,Comp "Skip" []), mpr])',
matches a composite called ``Seq'',
with a argument list containing two (marked) predicates.
The first argument predicate pattern \verb'(_,Comp "Skip" [])'
matches a ``Skip'' composite with no further subarguments.
The second argument pattern \verb'mpr' matches an arbitrary predicate
($P$ in the law above).
The righthand-side returns the application \verb'lred ";-lunit" mpr'
which simply constructs a string/predicate pair,
with the string being a justification note that says a reduction-step
using a law called ``$;$-lunit'' was applied.


\subsection{Conditional Reduce}

A conditional reduce law is an equation as per reduce,
but with conditional outcomes, e.g. $P = Q_1 \cond{C} Q_2$.
Matching an instance of $P$ with return a list of two pairs,
the first being $(C,Q_1)$, the second $(\lnot C,Q_2)$.
No attempt is made to evaluate $C$, but instead the REPL
asks the user to choose.
This is a key design decision for the calculator.
A general purpose predicate evaluator requires implementing
lots of theories about numbers, sets, lists,
and whatever else might be present.
Given the scope and purpose of this calculator is is
much more effective to let the user choose.

For an example,
here is one pattern of the condtional reduce
function for UTCP, that handles the following useful laws:
Provided that $\vec x \subseteq in\alpha P$
 (which in this case is $\setof{s,ls}$):
\RLEQNS{
   c[\vec e/\vec x]
   &\implies&
   (c * P)[\vec e/\vec x] = P[\vec e/\vec x] ; c * P
\\ \lnot c[\vec e/\vec x]
   &\implies&
   (c * P)[\vec e/\vec x] = \Skip[\vec e/\vec x]
}
\begin{code}
creduceUTCP d (_,PSub w@(_,Comp "Iter" [c,p]) sub)
 | isCondition c        -- true if expr c is a UTP 'condition'
   && beforeSub d sub   -- true if subst-vars are all undashed
 = ("loop-substn" [ctrue,cfalse])
 where
   csub = psub c sub            --  psub builds a substitution
   ctrue  = ( psimp d csub          -- psimp runs a simplifier
            , bSeq (psub p sub) w ) -- bSeq a b builds a;b
   cfalse = ( psimp d $ bNot csub
            , psub bSkip sub )
\end{code}



\subsection{Loop Unroll}

Iteration is typically defined in UTP as the least fixed point
w.r.t to the refinement ordering
that also involves sequential composition,
which itself is defined using existential quantification,
and $\Skip$.
\RLEQNS{
   c * P &\defs& \mu L @ (P ; L) \cond c \Skip
\\ P ; Q
   &\defs&
   \exists s_m,ls_m @
     P[s_m,ls_m/s',ls']
     \land
     Q[s_m,ls_m/s,ls]
\\ \Skip &\defs& s'=s \land ls'=ls
}
We do not want to explicitly handle quantifiers,
or fixed-points.
Instead we prefer to use the loop unrolling law,
as this is much more useful for the kinds of calculations we encounter.
\RLEQNS{
   c * P &=& (P \seq c * P ) \cond c \Skip
}
Even more useful are ones that split the conditional
and unroll a number of times ($\seq$ binds tighter than $\lor$
but looser than $\land$):
\RLEQNS{
   c * P &=& \lnot c \land \Skip \lor c \land P \seq c * P
\\ &=& \lnot c \land \Skip
       \lor c \land P \seq c \land \Skip
       \lor c \land P \seq c * P
\\ &=& \lnot c \land \Skip
       \lor c \land P \seq c \land P \seq c \land \Skip
       \lor c \land P \seq c \land P \seq c * P
\\ &=& \ldots
}
The implementation of unroll used for UTCP
took the simple command `\texttt{l}' to produce the conational form,
and a command \verb@"ln"@, with n a digit, to unroll the disjunctive form $n$ times.

This latter facility was not used with the UTCP theory,
but is being applied to further work that seeks a more compostional
theory of the same language.
\begin{code}
wUnroll :: Ord s => String -> DictRWFun s
wUnroll ns d mw@(_,Comp nm [mpr])
 | nm == nW = ( "W-unroll" ++ ntag ns, wunroll n )
 where

   ntag "" = ""
   ntag ns = '.':ns

   n | null ns = 0
     | isDigit $ head ns = digitToInt $ head ns
     | otherwise = 0

   wunroll 0  =  bCond (bSeq mpr mw) (bNot lsout) bSkip
   wunroll 1  =  bOr [ loopdone
                     , bSeq (loopstep 1) mw]
   wunroll 2  =  bOr [ loopdone
                     , bSeq (loopstep 1) loopdone
                     , bSeq (loopstep 2) mw]
   wunroll 3  =  bOr [ loopdone
                     , bSeq (loopstep 1) loopdone
                     , bSeq (loopstep 2) loopdone
                     , bSeq (loopstep 3) mw]
   wunroll _  =  bCond (bSeq mpr mw) (bNot lsout) bSkip

   loopdone = bD out
   loopstep 1 = bAnd [bM out, mpr]
   loopstep n = bSeq (loopstep (n-1)) (loopstep 1)

wUnroll _ _ mpr = ( "", mpr )
\end{code}
It does up to 3 unrollings---we won't explain all the various bits
but just point out that \texttt{loopdone} is $\lnot c \land \Skip$
(under the hood) and \texttt{loopstep} n is $(c \land P)^n$.


\subsection{Bringing it all together}

We make these two reduction functions ``known'' to the calculator
by adding them into a dictionary.
\begin{code}
lawsUTCPDict
 = makeDict [("laws", LawEntry [reduceUTCP] [creduceUTCP] [])]
\end{code}
We then can take a number of partial dictionaries and use various
dictionary functions,
defined in \texttt{CalcPredicates}, to merge them together.
\begin{code}
dictUTCP = foldl1 dictMrg [ alfUTCPDict, ..., lawsUTCPDict]
\end{code}
The main method of working with dictionaries is constrcut small ones
focussed on some specific area of interest.
These can then be combined in different ways to provide a number of complete
dictionaries that can vary in the order in which things are tried.









%\subsection{UTCP Recognisers}
%
%Some laws require matching that is a bit more sophisticated.
%For example,
%consider a useful reduction for tidying up at the end,
%assuming that $ls' \notin A$ and $ls \notin B$, and both $k$ and $h$
%are ground:
%\[
%   (A \land ls'=k) ; (B \land ls'= h)
%   \quad\equiv\quad
%   (A;B) \land ls'=h
%   \qquad \elabel{$ls'$-cleanup}
%\]
%However, we want this law to work when both $A$
%and $B$ are themselves conjunctions, with the $ls'=\dots$
%as part of the same conjunction, located at some arbitrary position.
%We can break the problem into two parts.
%First we do a top-level pattern match
%to see that we have a top-level sequential composition
%of two conjunctions,
%then we use a function that will check both conjunction predicate-lists
%for the existence of a $ls'=\dots$ component,
%and that everything else also satisfies the requirements regarding
%the occurrence, or otherwise of $ls$ or $ls'$:
%\begin{code}
%reduceUTCP d pr@(_,Comp "Seq" [ (_,Comp "And" pAs)
%                              , (_,Comp "And" pBs)])
% = case isSafeLSDash d ls' ls' pAs of -- no ls' in rest of pAs
%    Nothing -> lred "" pr
%    Just (_,restA) ->
%     case isSafeLSDash d ls' ls pBs of -- no ls in rest of pBs
%      Nothing -> lred "" pr
%      Just (eqB,restB)
%       -> lred "ls'-cleanup" $   -- build RHS
%             comp "And" [ comp "Seq" [ bAnd restA
%                                     , bAnd restB ]
%                        , eqB ]
%\end{code}
%The function \texttt{isSafeLSDash}
%is designed to
%(i) locate the $ls'=e$ conjunct and check that its rhs is a ground expression;
%(ii) check that none of the remaining conjuncts make use of the
%`unwanted' version of the label-set variable ($ls$ or $ls'$);
%and (iii), if all ok, return a pair
%whose first component is the ($ls'=\dots$) equality,
%and whose second is the list of other conjuncts.
%To achieve (i) above,
%we make use of two functions provided by the \texttt{CalcRecogniser} module:
%\begin{verbatim}
%mtchNmdObsEqToConst :: Ord s => String -> Dict s -> Recogniser s
%matchRecog :: (Ord s, Show s)
%           => Recogniser s -> [MPred s]
%           -> Maybe ([MPred s],(MPred s,[MPred s]),[MPred s])
%\end{verbatim}
%where
%\begin{verbatim}
%type Recogniser s = MPred s -> (Bool, [MPred s])
%\end{verbatim}
%A recogniser is a function that takes a predicate
%and if it ``recognises'' it, returns \texttt{(True, parts)},
%where parts are the subcomponents of the predicate in some order.
%The recogniser \texttt{mtchNmdObsEqtoConst v d} matches a predicate of the form
%\texttt{Equal (Var v) k}, returning a list with both parts.
%The function \texttt{matchRecog} takes a recogniser and list of predicates
%and looks in the list for the first predicate to satisfy
%the recogniser, returning a triple of the form
%(before,satisyingPred,after).
%If the recogniser succeeds,
%we then check the validity of the expression,
%and the absence of the unwanted variable from the
%rest of the conjuncts --- using boolean function
% \texttt{dftlyNotInP} (``definitely not in $P$''),
% so handling task (ii) above.
%\begin{code}
%isSafeLSDash d theLS unwanted prs
% = case matchRecog (mtchNmdObsEqToConst theLS d) prs of
%    Nothing -> Nothing
%    Just (pre,(eq@(_,Equal _ k),_),post) ->
%     if notGround d k
%      then Nothing
%      else if all (dftlyNotInP d unwanted) rest
%       then Just (eq,rest)
%       else Nothing
%     where rest = pre++post
%\end{code}
