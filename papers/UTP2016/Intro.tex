\section{Introduction}\label{sec:Intro}


\subsection*{Paper Plan}

\begin{itemize}
  \item motivate by talking about
  concurrency theory calculation travails
  \item explain decision to build a calculator
   (noting emerging tool work on PML as the seed)
   explain why UTP2 was not suitable.
  \item
     explain the key design decisions,
     and how these were motivated by agile software development
     principles (esp. YAGNI).
     Such decisions include
     \begin{itemize}
       \item dictionaries --- originally to support tailored rendering
       \item taking pretty-printing seriously
       \item relying on simple hand-proved (algebraic!) laws
       \item the avoidance of quantifiers,
        and how this forces us to deal with substitutability
     \end{itemize}
  \item
     Present a development of a theory using the calculator,
     step by step.
  \item
    Talk about how an agile approach really is important,
    not just in writing the calculator software and haskell models,
    but also when deciding what definitions and laws to focus on.
    Discuss finding the ``sweet spot'' between the use definition
    expansion, and hard-coded laws based on by-hand proofs.
    Note the key observation that the calculator approach
    leads almost inexorably towards a strong emphasis
    on some kind of (combinator?) algebra.
  \item
    Discuss future plans: formal release as a Haskell library;
    REPL customization; rendering options: \LaTeX, UTP2, Isabelle/UTP.
\end{itemize}
