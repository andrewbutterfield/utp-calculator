\section{Introduction}\label{sec:Intro}


\subsection*{Paper Plan - 20 page limit}

\begin{itemize}
  \item motivate by talking about
  concurrency theory calculation travails (\S\ref{sec:Intro})
  \item explain decision to build a calculator
   (noting emerging tool work on PML as the seed)
   explain why UTP2 was not suitable.(\S\ref{sec:Intro})
\end{itemize}


\subsection{Research Context}

We have recently started to explore using UTP
to do the  formal modelling of a language,
called ``Process Modelling Language'' (PML),
designed to describe software development and similar business
processes \cite{DBLP:journals/infsof/AtkinsonWN07}.
The main objective is to give PML a formal semantics,
as the basis for a number of analysis tools that could be made available
to process designers and users---with one key application
area being the modelling of clinical healthcare pathways.
It quickly became apparent that PML and similar (business) process
notations essentially involve concurrency with global shared mutable state.

There has been work using UTP
to model concurrent programs with shared mutable state,
most notably
with an encoding into actions systems having been done by
Woodcock and Hughes\cite{DBLP:conf/icfem/WoodcockH02}.
We have been looking at adapting this work to provide a UTP semantics
for PML.
In addition, we have started to take an interest in the ``Views'' paper
by Dinsdale-Young and colleagues\cite{conf/popl/Dinsdale-YoungBGPY13},
that provides a framework within which it is possible
to construct instances of many different concurrency theories,
ranging from
type-theory \cite{tal-toplas,Smit00b,journals/fuin/AhmedFM07},
Owicki-Gries\cite{Owicki76},
separation logic\cite{conf/lics/CalcagnoOY07}
and rely-guarantee\cite{Jones83}
approaches,
among others.


\subsection{Theory Construction Difficulties}

\subsection{A Plan}

Haskell\cite{Haskell2010}

A book by Bird \cite{Bird14}

Fun \cite{conf/esop/Rudiak-GouldMJ06}.


\subsection{Structure of this paper}

\begin{itemize}
  \item
     explain the key design decisions,
     and how these were motivated by agile software development
     principles (esp. YAGNI). (\S\ref{sec:Design})
     Such decisions include
     \begin{itemize}
       \item dictionaries --- originally to support tailored rendering
       \item taking pretty-printing seriously
       \item relying on simple hand-proved (algebraic!) laws
       \item the avoidance of quantifiers,
        and how this forces us to deal with substitutability
       \item deciding to add a save-to-file option
        just in time to produce material for this paper
     \end{itemize}
  \item
     Present a development of a theory using the calculator,
     step by step.(\S\ref{sec:Theorising})
  \item
    Talk about how an agile approach really is important,
    not just in writing the calculator software and Haskell models,
    but also when deciding what definitions and laws to focus on.
    Discuss finding the ``sweet spot'' between the use of definition
    expansion, and hard-coded laws based on by-hand proofs.
    Note the key observation that the calculator approach
    leads almost inexorably towards a strong emphasis
    on some kind of (combinator?) algebra. (\S\ref{sec:Conc})
  \item
    Discuss future plans: formal release as a Haskell library;
    REPL customization; rendering options: \LaTeX, UTP2, Isabelle/UTP.
    (\S\ref{sec:Future})
\end{itemize}
