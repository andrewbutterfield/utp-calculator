\section{Types}\label{sec:Types}

We give an overview of key types in the calculator here,
focussing on Expressions, Predicates and the Dictionary.

\subsection{Expressions}

In Fig. \ref{fig:expr-types} we show the Haskell declarations
of the datatypes used to represent expressions and substitution.
\begin{figure}[tb]
\begin{code}
data Expr s
  = St s | B Bool | Z Int | Var String
  | App String [Expr s] | Sub (Expr s) (Substn s) | Undef

type Substn s = [(String,Expr s)]
\end{code}
  \caption{Expression and Sunstitution Datatypes (\texttt{CalcTypes.lhs})}
  \label{fig:expr-types}
\end{figure}
Both datatypes are parameterised on a generic state type \texttt{s},
which allows us to be able to handle different concrete types
of shared state with one piece of code.
Both types have basic values (\texttt{St},\texttt{B},\texttt{Z},\texttt{T},\texttt{F}),
variables (\texttt{Var},\texttt{PVar}),
generic composites (\texttt{App},\texttt{Comp}),
substitution (\texttt{Sub},\texttt{PSub})
and undefined values (\texttt{Undef},\texttt{PUndef}).
The predicate datatype has a way to embed a (boolean-valued)
expression to form an atomic predicate (\texttt{Atm}),
as well as an explicit form for atomic predicates that take
the form of an equality (\texttt{Equal}).
We give a specialised form for equality simply because
its use in laws is widespread and worth optimising.

The abstract syntax is very simple,
and uses names as Haskell Strings to distinguish between
different composites.

\subsection{Predicates}

In Fig. \ref{fig:pred-types} we show the Haskell declarations
of the datatypes to represent predicates.
\begin{figure}[tb]
\begin{code}
data Pred s
  = T | F | PVar String | Equal (Expr s) (Expr s) | Atm (Expr s)
  | Comp String [MPred s] | PSub (MPred s) (Substn s) | PUndef

type MPred s = (Marks, Pred s)
\end{code}
  \caption{Predicate Datatypes (\texttt{CalcTypes.lhs})}
  \label{fig:pred-types}
\end{figure}
Both datatypes are parameterised on a generic state type \texttt{s},
which allows us to be able to handle different concrete types
of shared state with one piece of code.
Both types have basic values (\texttt{St},\texttt{B},\texttt{Z},\texttt{T},\texttt{F}),
variables (\texttt{Var},\texttt{PVar}),
generic composites (\texttt{App},\texttt{Comp}),
substitution (\texttt{Sub},\texttt{PSub})
and undefined values (\texttt{Undef},\texttt{PUndef}).
The predicate datatype has a way to embed a (boolean-valued)
expression to form an atomic predicate (\texttt{Atm}),
as well as an explicit form for atomic predicates that take
the form of an equality (\texttt{Equal}).
We give a specialised form for equality simply because
its use in laws is widespread and worth optimising.

Also important to point out is the fact that \texttt{Pred} is
defined in terms of \texttt{MPred}, which in turn is defined
in terms of \texttt{Pred}.
This is done to facilitate the association of markings (lists of integers)
with predicate constructs.
These markings are used to indicate which parts of a predicate
were changed at each calculation step.
We will not discuss marking further in this document
as it runs completely ``under the hood'',
and its only impact on the users of this calculator
is their need to be aware of the interplay involving \texttt{Pred} and \texttt{MPred}.

The abstract syntax is very simple,
and uses names as Haskell Strings to distinguish between
different composites.
So for example logical operators $\land$ and $\lor$
might be represented as \texttt{Comp "And"} and \texttt{Comp "Or"}
respectively.
We have a default way to pretty-print composites,
so that the predicate $\true \land \false$,
represented in Haskell by \texttt{Comp "And" [bT,bF]},
would render as \texttt{And(true,false)}.
However it is nice to be able to render it as the usual
infix operator, giving \texttt{true /\BS\ false}.
In addition, we need some way to associate laws and definitions,
as appropriate, with composites.

\subsection{Dictionary}





The data-structure that logs the complete calculation outcome
is a triple consisting of the final predicate,
a list of the steps, each with a justification string,
and the dictionary that was used.
\begin{code}
type CalcLog s = (MPred s, [RWResult s], Dict s)
\end{code}
